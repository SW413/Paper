\chapter{Compiler Overview}\label{Chp:CompilerOverview}
Most compilers are either a multi-pass or a single-pass compiler. 
Each have their advantages and disadvantages, however the most significant difference is that in a single-pass compiler, as the name suggests the compiler only passes the source code once. 
This highly limits the information available to the compiler and this decreases the odds of creating efficient programs, because the compiler cannot know what will happen in the following code, which is not wished for in \gls{gamble}. 
As such the \gls{gamble} compiler will be a multi-pass compiler and can therefore be divided into phases.
In this chapter the different phases of the compiler and their goals and tasks will be presented to give an overview for the chapters to come.

The compiler for \gls{gamble} is separated into three phases: Syntax analysis, contextual analysis and code generation.
\myref{fig:phases} shows a state diagram of the phases of the compiler.
Syntax analysis and contextual analysis transforms the source code into an intermediate representation, and verifies the source code according to the syntax specified in the \acrshort{cfg} and also for type as well as scope and type checking.
Java 1.8 is chosen as the language the compiler will be written in, which is the language used in the languages and compiler course aswell.
When making a compiler an object-oriented language simplifies many tasks because of encapsulation, polymorphism and inheritance. 
The paradigm allows for many structural options, and reuse of code when inheriting, which would be difficult using another paradigm like imperative or functional programming.
Java also works across platforms, which is a useful feature.

\begin{figure}[ht]
\centering
\includegraphics[width=0.35\textwidth]{figures/ClassDiagrams/CompilerDiagram.pdf}
\caption{State diagram showing the phases of the compiler when it takes \gls{gamble} source code and compiles it into machine code.}\label{fig:phases}
\end{figure}

In the syntax analysis phase the input source code is parsed and separated into tokens according to the \acrshort{cfg}, this is done by the scanner. 
The parser structures these tokens into a tree structure, which can be traversed in the path the source code is written.
When the source code has been parsed the tree is then simplified to remove unnecessary information such that a tree with less nodes can be traversed.
If a syntactical error is found in this phase the compiler will stop when the parsing is complete and report all found errors to the stdout.

In the contextual analysis phase the tree is used to generate a table, containing all the variables and functions which is declared in the source code.
This is called a symbol table, and it is used to check if the variables and functions called and used in the source code are in scope, and also if they uphold the type rules of \gls{gamble}.
If one or more errors are found in this phase the compiler will stop compilation, and report the errors to the programmer, not just what is wrong but also where the error is located.
If no errors are found it results in the tree now containing additional information about the types of expressions in the source code, and it continues to the next phase.

The last phase of the compiler is code generation.
In the code generation phase the output code is generated from all the information gathered from the previous phases of the compiler.

The target language of this compiler is \gls{opencl} C.
To compile \gls{opencl} C code additional software for the GPU is required, depending on the machine the path to this software may be required when compiling, as such one cannot simply run this on all machines.
By compiling the \gls{opencl} C code, it is translated into machine code and linked with libraries, which the computer then understands.
This is an abstraction made by the project group to simplify the process of generating code, as targeting the \acrshort{gpu} using its specific instruction set, not only gives problems targeting more types of GPUs but is also too demanding for the project group to understand let alone use in just one semester.
\gls{opencl} C is low level compared to Java or C\#, and C has even been characterised as a portable assembly language, as many features of the language translates closely to assembly. \citep{CPort}

\myref{fig:tombstone} shows a tombstone diagram of the translations of \gls{gamble} for this compiler.
The compiler takes the \gls{gamble} source code as input and translates it into \gls{opencl} C using a compiler written in Java, the \gls{opencl} C code is then compiled using a C compiler written in C, and outputs machine code which can be run by the computer. 
This diagram excludes the execution of kernels.
\begin{figure}[!ht]
\centering
\begin{tikzpicture}
\matrix (m) [matrix of nodes,%nodes={minimum width=1em,minimum height=1.7em}
            ]
{
 \gls{gamble}  & $\to$ &  \gls{opencl} C  \\
    &  Java    & \gls{opencl} C & $\to$ & \hspace{1 em} M \hspace{2 em}  \\
    &       &   & C  &         \\
    &       &               \\
  };
 \draw (m-1-1.south west) |- (m-1-3.north east) |- (m-2-2.north east) |- (m-2-2.south west) |- (m-1-1.south west);
\draw (m-2-2.south east) |- (m-2-5.north east) --(m-2-5.south east) -- (m-2-5.south west) |- (m-3-4.south west) |- (m-2-2.south east);

\end{tikzpicture}
\caption{Tombstone diagram for the compiler.}
\label{fig:tombstone}
\end{figure}


The overall structure of the compiler is also implemented using these phases as seen on \myref{fig:compilerOverview} where the \texttt{main} method of the \texttt{Main} class calls the different phases.
This separation helps give an overview when working with the compiler, all the phases are distinctly separated, this separation is helpful when making changes as a change in one phase will not conflict with other phases it also helps locate where in the compiler changes are required.

\begin{figure}[!ht]
	\begin{sideways}
	%\fbox{
		\begin{minipage}{18cm}
			\includegraphics[height=0.42\textheight]{figures/ClassDiagrams/DiagramOfCallsFromMain.pdf}
		\end{minipage}
	%	}
	\end{sideways}
	\centering
	\caption{Diagram showing the structure of the compiler by showing the \texttt{main()} method's method calls. Arguments are omitted for simplicity}\label{fig:compilerOverview}
\end{figure}

\clearpage

\section{Source Code as Trees}\label{SourceCodeAsTrees}
The source code of a program is parsed by the compiler, but should also saved in some way so it is possible to manipulate the source code, and performing the phases of the compiler.
This is often done by using a tree, as is it in the \gls{gamble} compiler.

The tree structure is useful for this purpose because every node of the tree can contains information, and have children which then makes it possible to express the productions of a grammar by following a path on the tree from the root to a leaf.
A parse tree separates the source code into different productions of the grammar it represents, but also contains all of the syntax from the grammar, such as parenthesis.
The tree structure also makes it possible to traverse the tree in the same path as the source code is written, which means that the tree is able to express the structure of the source code as well as the statements which are found in the program.
An example of parse tree from the declaration \texttt{int a = 5;} is shown in \myref{image:PST}

\begin{figure}
    \centering
    \includegraphics[width=0.5\linewidth]{figures/Trees/PST.PNG}
    \caption{A parse tree from the expression \texttt{int a = 5;} using \glspl{gamble} \acrshort{cfg}.} \label{image:PST}
\end{figure}

The following chapter will explain how the compiler creates a parser to produce these parse trees for the source code, and thus making it possible to use the trees in the compiler.


