%Subphases
\chapter{Contextual Analysis}
Once the source code has passed through the syntax analysis, and thereby is validated in regard to the \acrshort{cfg}, it must be checked for contextual errors.

In the contextual analysis phase semantic errors are checked for, this includes type errors such as type mismatch, e.g. adding a boolean to a float, which may well be adhering to the grammar, but is not a valid arithmetic expression in \gls{gamble}.
Scope checking is also done in the contextual analysis. 
The design and implementation of the contextual analysis as well as how such errors are handled are described in this chapter.

The contextual analysis phase has sub-phases for scope- and type-checking and error reporting.
An illustration of the design for this can be seen on \myref{fig:flowContextual}, both sub-phases scope-checker and type-checker will have its own visitor class implemented.
The diagram shows that if variables are out of scope or type mismatches, it results in errors which will stop the compilation, if no errors occur it results in a decorated \acrshort{ast} which is then used in the next phase of the compiler, code generation.

\vspace{10pt}
\begin{figure}[h]
    \centering
    \begin{tikzpicture}[node distance = 3cm, auto]
        \node (invi1) [invi, draw=none] {};
        \node (snyanal) [lille, below right=-2cm and -1.6cm of invi1, minimum width=15.5cm, minimum height=5cm, fill=blue!10, label={[xshift=-5.4cm, yshift=-1cm]Contextual Analysis}] {};
        \node (ast) [lille, below right=-0.35cm and -1cm of invi1] {Abstract Syntax Tree};
        \node (symboltable) [lille, minimum width=6.75cm, minimum height=2.4cm, right=2.8cm of invi1, fill=blue!20, label={[xshift=0cm, yshift=-1cm]Symbol Table}] {};
        \node (scope) [lille, right=1.1cm of ast] {Scopechecker};
        \node (type) [lille, right=0.7cm of scope] {Typechecker};
        \node (dast) [lille, right=1.1cm of type, align=left] {Decorated\\ Abstact Syntax Tree};
        \node (codegen) [lille, right=1.1cm of dast, align=left] {Code \\Generation};

        \node (error) [invi, draw=none, minimum width=2cm, below=1cm of symboltable, label={[xshift=40pt, yshift=-17pt]Error report}] {};

        %\node (error) [draw=none, above=22pt of parser, label={[xshift=0cm, yshift=6 pt]Error report}] {};

        \draw[black,fill=black, above=1cm of parser] (6.35,-2.5) circle (1ex);
        \draw[black, above=1cm of parser] (6.35,-2.5) circle (1.3ex); 

        \draw [arrow] (ast) -- (scope);
        \draw [arrow] (scope) -- (type);
        \draw [arrow] (type) -- (dast);
        \draw [arrow] (dast) -- (codegen);

        \draw [arrow,dashed] (scope) -- (error);
        \draw [arrow,dashed] (type) -- (error);
        \draw [arrow,dashed] (symboltable) -- (error);

    \end{tikzpicture}
    \caption{Diagram showing the modules of the contextual analysis. } 
    \label{fig:flowContextual}
\end{figure}
\vspace{-20pt}

\subsection*{Symbol Table}
In a compiler it is useful to store information about the identifiers, variables and functions in a data structure. 
This information can be useful for scope checking and type checking.

There are two common ways of having a Symbol Table, either have one single table for every identifier or have one for each scope. 
If constrained by memory then having a single symbol table can be beneficial, however having multiple symbol tables can simply the code at a memory cost. 

In \gls{gamble} there exists scopes, and for each of these scopes there is a corresponding symbol table. 
Scopes inherit from each other so a scope can enclose another. 
The outermost scope, the global scope, is where functions are declared, every scope is either directly enclosed by this scope or recursively.
This means that every function can be called from anywhere within the source code. 

In \gls{gamble} the class SymbolTable represents the symbol table.
The core constituent of this class is the ArrayList of the Scope class, called allScopes, meaning that every scope is stored in this ArrayList.
Every scope contains Map of Symbols and strings as keys, and information about the scope such what scope it is enclosed by. 
The key represents the name of the symbol, and must be unique to the scope and not found in an enclosing scope, as this could cause an ambiguity to arise. 
A symbol is either a variable or a function, in the class Symbol its data type, name and scope is stored. 


\section{Scope Checking}
As a part of the contextual analysis the compiler must ensure that every reference is valid in their given scope.
The validity of such reference is specific to the rules of the language, this section describes how the compiler upholds the rules listed in \myref{subsec:Scope}.
\subsection*{Design}
In \myref{subsec:Scope} the scope of identifiers, variables and functions, are defined for \gls{gamble}.
A variable is in scope from its declaration until the end of the block it is declared in.
An inner scope inherits the identifiers declared in the outer scopes. 

During contextual analysis it is important to verify that each variable and function used are in scope and if not it should produce a descriptive error message.
The error message should indicate which identifier is not in scope, and the line this identifier is introduced on.

To check this all references to identifiers must be checked to see if they match an identifier in the symbol table of the current scope, and recursively the scope which encloses it. 
Furthermore it is important that any usage of an identifier only exists after its declaration.
In the compiler for \gls{gamble}, the symbol table is filled while the compiler is also performing the scope checking.
This is because it reduces the amount of traversals through the tree, scope checking and filling the symbol table is also of similar concept, e.g. a declaration creates an entry in the symbol table, while expressions simply use lookups in this table.

The scope checker can produce two errors: redeclaration errors and undeclared errors defined in \myref{subsec:Scope}.\todo{reference nødvendig igen? also: can produce. MP}
A redeclaration error is produced when an attempt to declare a variable while it is already declared in scope is made.\todo{klumset formulering. MP}
An undeclared error is produced when an attempt to use a variable which is not declared in the current scope or any enclosing scopes are made. 
Examples are shown in \myref{lst:scopeErrors}.

\begin{lstlisting}[caption=Examples of scope errors in \gls{gamble}, numbers=none,frame=tlrb,label={lst:scopeErrors}]
/* [...] */
int a = 1;
float a = 2.2;   /* Redeclaration error */
int a = 2;       /* Redeclaration error */ 

b = 2;           /* Undeclared error */   
int b = 0;
b = foo();       /* Redeclaration error and undeclared error */ 
/* [...] */
\end{lstlisting}
\todo{Evt lav en UML diagram der viser de klasser der bruges til at lave symbol tabellet, det gør i hvert fald det nemmere at forstå implementation. Men et diagram hører til i design afsnittet.}

\subsection*{Implementation}
\info[inline]{Todo: Listing or/eller figur i denne del.}
In the \gls{gamble} compiler, the class \texttt{SymbolTable} represents a collection of scopes.
The core constituent of this class is the ArrayList of the \texttt{Scope} class, called \texttt{allScopes}, meaning that every scope is stored in this ArrayList.
Every scope contains a hashed map with \texttt{Symbol}s as values and strings as keys.
Furthermore all scopes contain information about the particular scope such as enclosing scope and a unique id. 
The key in the hashed maps represents the name of the symbol, and must be unique to the scope and not found in an enclosing scope, as this could cause an ambiguity to arise.
This ambiguity is not allowed in \gls{gamble} and therefore throws a redeclaration error.
\todo{Meget forvirrende at holde styr på uden en figur tbh. - Søren}

To determine whether or not a declaration is a redeclaration, the compiler looks up the name of the variable in the \texttt{symbolMap} of the current scope.
If no entry is found, the search continues in the \texttt{symbolMap} of the enclosing scope and so on recursively.\todo{Må man ikke redeclare et variable i et indre scope ??, så hvis jeg bruger tmp ude i global scope, så må jeg ikke bruge den inde i en funktion ?? - Søren - Ikke ifølge hvad der står her - Marc}
The same lookup process is executed when a variable is used e.g in an expression or assignment.
Hereby all enclosing scopes are checked for the declaration of the variable; making redeclaration of variables in scope, and usage of undeclared variables impossible in \gls{gamble}, as defined in \myref{cha:language_design}.
A \texttt{symbol} is an encapsulation of a variable and relevant peripheral data.
The peripheral data consists of a boolean, used for checking if a declared variable is used, and an integer with the line number if the declaration of the variable.
The boolean describing if a declared variable is used, is set to true, if the previously described lookup process for a variable in use, finds a declared variable from the unique id (the name of the variable).
This boolean also makes it possible for the \gls{gamble} compiler to find unused variables and then prompt the user with a relevant warning.

In order for all of the above to be implemented an instance of the \texttt{SymbolTable} class is passed via the constructor, to a visitor which traverses the \acrshort{ast}.
This visitor, the \texttt{SymbolTableFillVisitor} then fills the referenced \texttt{symbolTable} with scopes and their symbols.\todo{Hedder den stadig SymbolTableFillVisitor?}
Every time the visitor meets the start of a new scope e.g. the block of statements within a loop construct.
A new instance of the \texttt{Scope} class is pushed to the \texttt{scopeStack}, hereby making it possible to fill the relevant scope when the block of statements is visited.
At the end of a scope the top element on the \texttt{scopeStack} is popped, and as a consequence of this the top of the \texttt{scopeStack} is back to the enclosing scope.

\section{Type Checking}
Another important part of the contextual analysis phase for the compiler is the type checking, which enforces the type system of \gls{gamble}.
This section will cover the design and implementation of the type checking in the compiler.
\subsection*{Type Checking}
The second important part of the contextual analysis phase for the compiler is the type checking which enforces the type system of \gls{gamble}.
As \gls{gamble} is statically typed it is necessary to check if all references to identifiers and constant values fit into the context they exist in. 
Since implicit conversion between floating point and integer types is not a part of gamble an error must be issued everywhere they are used wrongly. 
It is however possible to convert between integer and floating point types internally e.g. from int16 to int64. 

The symbol table is used as a reference for which type the variable is, and therefore the type checking happens after the symbol table have been filled and therefore after scope checking is completed. 
Type checking is done in many parts of the code, one or more times for each line is common. 
For every operator it must be checked if its types match and if it results in an assignment if that also matches.
Every function call must match the formal parameters of the function. 

The errors produced by the type checker is: ArgumentsError and TypeMismatchError.
An ArgumentsError indicates that the number of arguments does not match the ones given.
A TypeMismatchError is any error which is caused by a value or identifier not being compatible with the function parameters, operator used etc.
Examples are shown in \myref{lst:typeErrors}.

\begin{lstlisting}[caption=Examples of type errors in \gls{gamble},numbers=none,frame=tlrb,label={lst:typeErrors}]
/* [...] */
int a = 1 + 2;      /* Valid */
float b = 2.2 + 1;  /* TypeMismatchError */
float c = 2;        /* TypeMismatchError */ 

a = 2.2;            /* TypeMismatchError */
b = foo(1);         /* ArgumentsError (Takes more or fewer arguments) */ 
/* [...] */
\end{lstlisting}

This check is done as a part of decorating the AST and uses a visitor to traverse the AST. \todo{implementering? ...}

\section{Error Handling}
\subsection*{Design}\label{subsec:DesignErrorHandling}
The source code given to the \gls{gamble} compiler can contain errors and warnings, these are found in the various stages of the compiler.
Most central are scope and type related errors, these are found in the contextual analysis phase of the compiler. 
The following errors and warnings are reported by the \gls{gamble} compiler:
\info[inline]{Itemize yay or nay?}
\begin{itemize}
	\item Arguments error: The arguments of a function call doesn't match the ones used.
	\item Redeclaration error: A variable or function with the same name is already declared.
	\item Type mismatch error: The types used are incompatible with the operator. 
	\item Undeclared error: Attempted use of an undefined variable or function.
	\item Unused variable warning: A warning signaling that a variable or function is declared but never invoked.
\end{itemize} 
The errors in the compiler should give useful information about: Where the error is in the source code and what variable(s) or function(s) was wrongly used.
This include line numbers which then have to be carried over from the parse tree to the abstract syntax tree, this is expanded in \todo{ref til afsnit som ikke findes i denne branch}.

\subsection*{Implementation}\label{subsec:ImplementationErrorHandling}
In the \gls{gamble} compiler this is implemented by having a class \texttt{LanguageError} of which each specific error type inherits from.
The \texttt{LanguageError} superclass has information about what type the error, an enumeration which is either \texttt{Error} or \texttt{Warning}. 
This is so because a program which has warning but no errors, should still compile, however any program with errors should not. 
Compared to simply printing any error when it was discovered and stopping compilation, this model allows the compiler to go through the entire \acrshort{ast} and discover every error. 
It also unifies how errors are printed and shown to the user. \todo{Denne og forrige sætning er måske lidt svage? Skal der være mere ellere mindre om det. På plus siden er det jo en vurdering og det som viser forståelse osv.}
There is also an integer indicating which line number the error is contained on.
This is to enhance the error reporting process, making it easier for users to debug. 
\todo{Måske skriv noget om at dette kræver informationen er med i ASTet fra PTet ?}
As an example the class for \texttt{UnDeclaredError} is shown in \myref{lst:undeclarederrorclass}.
First there are variable which contains information needed to report the error to users. 
Then there is a constructor which simply assigns all values given to it to the new objects fields.
Lastly there is the \texttt{toString} method which is overridden from the implementation in the \texttt{LanguageError} class. 
An example of an \texttt{UnDeclaredError} is: \texttt{Error[line   42]-> Undeclared variable towel in scope Local}. %Sneaky H2G2 reference 

\begin{lstlisting}[caption=The UnDeclaredError class in the \gls{gamble} compiler,numbers=none,frame=tlrb,label={lst:undeclarederrorclass}]
public class UnDeclaredError extends LanguageError {
    private Variable unDeclaredVariable;
    private Scope scope;

    public UnDeclaredError(Variable unDeclaredVariable, Scope scope, int lineNum) {
        this.unDeclaredVariable = unDeclaredVariable;
        this.scope = scope;
        this.lineNum = lineNum;
        this.errorType = ErrorType.ERROR;
    }

    @Override
    public String toString() {
        String type = unDeclaredVariable.isFunction() ? "function " : "variable ";
        return super.toString() + String.format("Undeclared %6$s %1$s%4$s%3$s in scope %2$s%5$s%3$s",
                ANSI_RED, ANSI_BLUE, ANSI_RESET,
                unDeclaredVariable, scope, type);
    }
}
\end{lstlisting}

This error is found in the \texttt{CheckIfUndeclared} method in the \texttt{SymbolTableFillVisitor} shown in \myref{lst:CheckIfUndeclared}.
This method is called whenever a variable or function is invoked, and returns the appropriate information. \todo{Teknisk set returnerer den ikke informationen men sætter det gennem en reference, skal dette skrives eksplicit? Kunne ikke lige formulere det ordenligt.}
If the variable or function invoked is not declared then an error is added to a global list of errors. 
\begin{lstlisting}[caption=The CheckIfUndeclared method in the SymbolTableFillVisitor class in the \gls{gamble} compiler,numbers=none,frame=tlrb,label={lst:CheckIfUndeclared}]
public Void CheckIfUndeclared(Variable variable, StatementNode node) {
    Symbol tmpSymbol = symbolTable.currentScope().resolve(variable.getId());
    if (tmpSymbol != null) {
        /* Sets appropriate information to the 'variable' variable */
    } else {
        errors.add(new UnDeclaredError(variable, symbolTable.currentScope(), node.getLineNumber()));
        symbolTable.currentScope().define(variable);
    }

    return null;
}
\end{lstlisting}
