\chapter{Strategy}\label{Metode}

This chapter will briefly explain the strategy we will use to organise our work to find an answer to our problem statement. 
Inventing a programming language and developing a compiler for it, is a complex and time consuming process, so having a strategy for the development helps, which is why our strategy is discussed in this paper.

\section{Development method}
The development of a language and a compiler for it is not only difficult but the process can also be confusing for newcomers like us. 
In order to develop this, a comprehensive understanding of how programming languages work, and how compilers are structured is needed.
Therefore we need to follow the courses on this semester, acquiring skills and familiarising ourself with the tools needed to develop a compiler.
This makes planning the project very difficult, before we have been following the courses for some time.
Therefore an iterative development method is more useful as we can go back and change previous decisions when further knowledge is obtained, in contrast to a linear development.
One such method is Scrum, which group members has had success with on previous semesters.
Therefore we choose to develop this project using Scrum, but with some changes.
We will make use of scrum's organisational tool, such as the scrum board, daily scrums and sprints with stories.
We will not have the specific roles which you would normally find in scrum, such as the scrum master or the product owner. \citep{Scrum}
The reason for this is, partly due to development methods not being in the curriculum for this project, but also due to the project not having use of these tools.
There is no customer for whom we are developing the compiler nor is there really time for one person in the group to be scrum master, as he would still need to focus a lot on the tasks the rest of the group are also partaking in.

With this method we are able to work in short sprints of one to two weeks, which gives us the opportunity to incorporate the tools, concepts, and methods we learn during the development process.
We do have a hard deadline to take into account, which forces us to have a certain number of sprints, instead of being able to develop on the project eternally.
That is not to say we do not plan ahead, but our long term plans are versatile, i.e. if we learn of something new, it is easier for us to implement this new knowledge in our project by going back to fix mistakes and adding features, compared to if we used linear development method.