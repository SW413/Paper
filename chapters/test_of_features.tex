\chapter{Test of Language Features}
\label{cha:test_of_language_features}
In this chapter \gls{gamble} and the compiler developed for it, will be tested.
The focus of this test will be to test how the compiler reacts to invalid input, both syntactic and contextual.
Also to test if simple operations act as intended and documented in \myref{cha:semantics}. 
As written in our success criteria, \myref{sec:OurCriterias}, it is important to give descriptive error messages, and provide type and scope checking. 

The most optimal solution would be to verify the correctness of the build compiler in according to the language specification in the earlier chapters.
But since formal verification of correctness is a very complex problem, tests is performed to indicate that features of the \gls{gamble} are correctly implemented. \citep{Verification} 

These tests will be performed by using the source code listed in each example as an input for the compiler. 
Then the output from the compiler will be written below the code. 

\section{Syntactic Tests}
\info[inline]{wanan do this? ANTLR errors only...}

\section{Contextual Tests}

\subsection*{Scope Tests}
\subsection*{Type Tests}

\subsection*{Conclusion}
Based on these test it is not possibly to prove the correntnees of the compiler or it's output in general.
This is because testing the compiler is not the same as verification of the compilers implementation of the language specification.

However the aforementioned tests cover a large feature set with some different inputs.
overall implements the language specifications presented in the earlier chapters of the report/paper?.