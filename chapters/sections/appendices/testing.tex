\chapter{Testing}\label{app:testing}
\section{Execution time test}
\begin{lstlisting}[language=bash,caption={Bash script for automating execution time test},frame=tlrb]
#!/bin/bash
make clean
make

ln -s -f ../matrixdata/a.matrix8k4k a.matrix
ln -s -f ../matrixdata/b.matrix8k4k b.matrix
ln -s -f ../matrixdata/res.matrix8k4k res.matrix

min=100
max=8000
step=100

for ((i=0; $min+i<=$max; i+=step))
do
    echo "$((i/step)) out of $(((max-min)/step))"
    sed -i "2s/.*/$((min+i)) $(((min+i)/2))/" "a.matrix"
    sed -i "2s/.*/$(((min+i)/2)) $(((min+i)/2))/" "b.matrix"
    sed -i "2s/.*/$((min+i)) $(((min+i)/2))/" "res.matrix"
    echo -n "$(((min+i)*((min+i)/2)))" >> test.txt    
    (/usr/bin/time -f "\t%e" ./test) 2>> test.txt >/dev/null
done
\end{lstlisting}

\begin{lstlisting}[caption={Sample of file containing a matrix, some data omitted},frame=tlrb]
MATRIX_INT
2000 1000
9 8 0 2 0 6 8 2 7 9 1 5 4 1 4 0 7 5 2 2 2 4 3 1 ... 
3 5 4 2 2 3 0 6 1 3 8 1 4 6 7 8 9 4 5 7 5 3 3 5 ... 
6 0 9 9 8 3 0 7 6 3 0 5 7 1 9 2 2 3 1 0 1 9 1 5 ...
0 2 2 6 4 4 3 4 0 8 0 3 6 7 6 9 3 2 8 9 5 4 4 1 ...
6 4 1 4 5 2 7 9 5 2 5 3 6 9 4 7 1 1 3 5 8 6 0 6 ...
1 4 9 6 0 7 8 7 3 0 8 9 0 0 0 3 8 7 0 8 7 2 2 3 ...
9 9 3 8 2 1 2 5 0 4 4 2 1 6 2 9 2 8 4 6 0 5 0 6 ...
4 4 4 3 8 3 8 4 9 5 0 1 3 6 6 9 2 8 0 9 2 5 5 6 ...
3 4 4 4 1 2 7 4 4 3 8 4 7 8 7 1 4 4 6 1 5 9 3 2 ...
...
\end{lstlisting}

\begin{lstlisting}[language=C,caption={C-code for sequential matrix multiplication using the CPU},frame=tlrb]
#include "complexTypes.h"
#include "gambleStdlib.h"
#define true 1
#define false 0

int main(){
    //MATRIX GENERATION FOR a {rows, cols, dataSize, dataStart}
    matrix a = {0, 0, sizeof(float) * 0 * 0, calloc(0 * 0, sizeof(float))};
    //END MATRIX GENERATION
    loadFromFile("a.matrix", &a);
    //MATRIX GENERATION FOR b {rows, cols, dataSize, dataStart}
    matrix b = {0, 0, sizeof(float) * 0 * 0, calloc(0 * 0, sizeof(float))};
    //END MATRIX GENERATION
    loadFromFile("b.matrix", &b);
    //MATRIX GENERATION FOR res {rows, cols, dataSize, dataStart}
    matrix res = {0, 0, sizeof(float) * 0 * 0, calloc(0 * 0, sizeof(float))};
    //END MATRIX GENERATION
    loadFromFile("res.matrix", &res);
    
    for(int i = 0; i < a.rows ; i++) {
        for(int j = 0; j < b.cols ; j++) {
            for(int k = 0; k < a.cols ; k++) {
                if(i < res.rows && j < res.cols){
                    ((float *) res.dataStart)[i * res.cols + j] += ((float *) a.dataStart)[i * a.cols + k] * ((float *) b.dataStart)[k * b.cols + j];
                } else {
                    printf("\nTrying to access entry out of bounds in %s\n", "res");
                    exit(0);
                }             
            }
        }
    }

    saveToFile("res.matrix", "MATRIX_FLOAT", res);
    return 0;
}
\end{lstlisting}

\begin{lstlisting}[language=C,caption={\gls{gamble} sourcecode for matrix multiplication using the GPU},frame=tlrb]
//Matrix multiplication of large matrices on GPU
matrix<float> a = fileToMatrix("a.matrix");
matrix<float> b = fileToMatrix("b.matrix");
matrix<float> res = fileToMatrix("res.matrix");

res = a * b;

matrixToFile(res, "res.matrix");
\end{lstlisting}

\begin{lstlisting}[language=C,caption={C-code for matrix multiplication using the GPU, compiled from \gls{gamble} sourcecode.},frame=tlrb]
#include "simpleCL.h"
#include "complexTypes.h"
#include "gambleStdlib.h"
#include <math.h>
#define true 1
#define false 0

/*--= MAIN METHOD =--*/
int main(){
/* Simple-OpenCL Hardware setup  */
	sclHard* allHardware;
	sclHard hardware;
	sclSoft software;
	int found = 0;
	allHardware = sclGetAllHardware( &found );
	hardware = sclGetFastestDevice(allHardware, found);

    size_t local_size[2] = {1, 1};
    size_t global_size[2] = {1, 1};

    printf("\n");
/* END Hardware setup */


    //MATRIX GENERATION FOR a {rows, cols, dataSize, dataStart}
    matrix a = {0, 0, sizeof(float) * 0 * 0, calloc(0 * 0, sizeof(float))};
    //END MATRIX GENERATION
    loadFromFile("a.matrix", &a);
    //MATRIX GENERATION FOR b {rows, cols, dataSize, dataStart}
    matrix b = {0, 0, sizeof(float) * 0 * 0, calloc(0 * 0, sizeof(float))};
    //END MATRIX GENERATION
    loadFromFile("b.matrix", &b);
    //MATRIX GENERATION FOR res {rows, cols, dataSize, dataStart}
    matrix res = {0, 0, sizeof(float) * 0 * 0, calloc(0 * 0, sizeof(float))};
    //END MATRIX GENERATION
    loadFromFile("res.matrix", &res);
    if(a.cols == b.rows && a.rows == res.rows && b.cols == res.cols){
        //MATRIX MULTIPLICATION OF a AND b INTO res
        global_size[0] = a.rows;
        local_size[0] = 1;
        global_size[1] = b.cols;
        local_size[1] = 1;
        software = sclGetCLSoftware("matrixMul.cl", "matrixMul", hardware);
        sclManageArgsLaunchKernel(hardware, software, global_size, local_size, "%r %r %R %a %a",
        	a.dataSize, a.dataStart, b.dataSize, b.dataStart, res.dataSize, res.dataStart,
        	sizeof(unsigned int), &a.cols, sizeof(unsigned int), &b.cols);
        global_size[0] = 1;
        global_size[1] = 1;
        //END OF MATRIX MULTIPLICATION
        
    } else {
        printf("\nMatrix %s and %s not compatible for matrix multiplication into %s \n", "a", "b", "res");
        exit(0);
    }
    
    saveToFile("res.matrix", "MATRIX_FLOAT", res);
    return 0;
}
\end{lstlisting}
