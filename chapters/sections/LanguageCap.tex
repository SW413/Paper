\section{Capabilities of \gls{gamble}}
\myref{sec:OurCriterias} discusses what criteria \gls{gamble} should have in order to be successful for this project.
\myref{cha:test_of_language} shows that the gamble code executing on the \acrshort{gpu} for matrix multiplications has a faster execution time for big matrices than performing the same calculation sequentially in C on the \acrshort{cpu}.
This is done without any overhead explicitly stating that the computation should be run on the \acrshort{gpu}.
If a computer has no \acrshort{gpu}, the OpenCL generated code may also run on Intel processors, and this is done without any change to the \gls{gamble} program.
The basic operations which are possible to perform on matrices and vectors in \gls{gamble} makes it possible to do many more complex algorithms, and with the speed up provided by the \acrshort{gpu}, we think the execution time of these should be shorter than their sequential counterparts.
An example of a more complex algorithm could be an algorithm for finding the singular value decomposition.
A function for finding eigen values and transforming matrices into reduce row echelon form etc, would greatly improve on the capabilities of a \gls{gamble} program in a Linear algebra environment, and it is also possible to construct these functions using the tools \gls{gamble} has available.