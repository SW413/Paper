\section{Parallel computing platform}
In \myref{CUDAvsOpenCL} it was decided whether to use parallel computing platform developed by Nvidia, CUDA, or the one to developed by The Khronos Group, OpenCL.
The choice fell on OpenCL due to its cross platform capabilities.
In this section we will discuss how this choice affected the project and whether or not it was the best choice, the perks as well as the flaws encountered whilst using OpenCL.

The significant factor that made the choice fall upon OpenCL was its cross platform abilities; once the OpenCL code generation started complications in the cross platform abilities was met.
While indeed OpenCL can run cross platform, this feature is not without its faults.
This especially became evident as the different \acrshort{gpu}s support different versions of OpenCL, thus having different features implemented.
Furthermore, as mentioned in \myref{subsec:runtime} different processor platforms also perform different on the same code, even if the platforms have different specifications.
This as a result means that to really

One of these faults the project group encountered quite early was that Windows is not the most cooperative operating system when it comes to linking the required libraries for OpenCL to run.
Throgh Visual Studio the project group was able to run OpenCL C example programs, thus proving that it could work on the Windows operating system, however doing so through a terminal and a GCC compiler proved to be quite difficult.
In an attempt to solve this issue two things also became clear to the project group, just about all help available was aimed at the OS X and Linux operating systems, as a result the project group proceeded development using these operating systems rather than windows.

Another thing realised by the project group while learning about OpenCL was that help and web ressources for OpenCL was represented by a small margin of what was available for CUDA.
As a result this led to many a trial and error testing to figure out the workings of OpenCL and debugging, where as if CUDA had been the platform worked on, it may have been easier