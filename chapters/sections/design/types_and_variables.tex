\section{Types and Variables}\label{sec:Types}
%http://www.informit.com/articles/article.aspx?p=2103809&seqNum=3
\gls{gamble} has three primitive data types; integers (int), floating points (float) and boolean values (bool), as well as two complex data types; matrices and vector. 
In \gls{gamble} any variable must be initialised when it is declared. \todo{Er dette stadig sandt? fx matricer -- TK}
This avoids any null references which could otherwise cause runtime errors.
In \gls{gamble} all variables are statically stored, there are several reasons for this and therefore the language does not implement neither explicit nor dynamic memory handling. \todo{Det er ikke dynamic at vi skriver matrix[a,b] hvor a og b er variabler som bestemmes på baggrund af alt muligt ?? - Søren}
The first reason for this is that the purpose of the language is to perform arithmetic operations on the \acrshort{gpu}, so by excluding this the language's simplicity increases, and therefore its read- and writability increases.
A second reason being that the size of every variable in \gls{gamble} is unchangeable.
Any vector and matrix will be allocated on the heap by the compiler. \todo{Compiler igen. - Søren ... Det er heller ikke noget compileren gør xD Men den sørger for at det sker der. -- TK}

\subsection*{Integers and floating points}
An \texttt{integer}\todo{texttt{int} eller integer som plantxt MP} in \gls{gamble} is either a positive or negative integer, represented by the two's complement method. 
Meaning that it ranges from $-(2^{n-1}) $ to $2^{n-1} - 1 $, where n is the number of bits used.
A floating point number is an approximation to a real number. 
It is separated in 3 parts, the sign bit, the mantissa and the exponent. 
Floats in \gls{gamble} are in base 2, the value of a float is calculated by $ {(-1)}^{sign bit} \cdot mantissa \cdot 2^{exponent} $. \todo{alt dette er vel kun på grund af vi codegen'er til c? MP -- Ikke rigtigt, det skal være sandt for alle gyldige GAMBLE compilers. -- TK}
Integers and floating point numbers default to 32 bits with the keywords \texttt{int} and \texttt{float}. 
It is possible to declare 16 and 64 bit integers and floating point numbers by postfixing their size to their respective keywords, e.g. \texttt{float64} or \texttt{int16}. \todo{Skal det med i semantikken at man kan skrive 16 og 64 ? Det skal det vel ? - Søren}
It is useful to be able to specify the size of an integer or a floating point number, as data transfer can cause a lot of overhead particularly when transferring data to and from the \acrshort{gpu}. \todo{Evt. og max størrelse ? -- TK}

\subsection*{Boolean}
A boolean value is either true or false. 
These are used for control structures in \gls{gamble} which takes a boolean value as a parameter. 
But can also be used for any means which the programmer finds useful. 

\subsection*{Vectors and Matrices}\todo{Hvordan tilgås matricer?}
A vector in \gls{gamble} is a sequence of entries which are each the same primitive data type. 
The dimensions of a vectors and matrices are immutable, meaning that its dimensions cannot be changed after its declaration. \todo{That is not what immutable means? immutable == const -- TK}
A matrix consists of a number of row vectors inside a column vector, each row vector is the same dimension. 
As for other variables both vectors and matrices must be initialised when they are declared. \todo{Stadig sandt? ;) -- TK}
This is done with the syntax shown in \myref{lst:matrix}.
As seen there are two ways of declaring a matrix and a vector, one can either provide the values for each entry, or one can provide the dimensions required of a matrix or vector in which case all values are initialised as zero.
The syntax for the latter does not adhere to the concept of orthogonality mentioned in \myref{cha:language_criteria} although the use of this syntax should be held to a minimum, as it primarily exists such that functions in the standard library for \gls{gamble} can be created.\todo{Det har vi ikke lavet ? Har vi - Søren - Dette skulle bruges til MatrixAdd og de funktioner så det er vel egentlig ikke sandt længere, std lib eller ej - Marc}
What these functions entail is described in \myref{sec:funcs}.

\begin{lstlisting}[caption={Syntax for creating a matrix or vector},label={lst:matrix},numbers=none]
//Makes a 2x2 matrix by providing entry values
matrix<int> m = [1, 0; 
                 0, 1];
//Makes a 100x100 matrix with all values set to 0
matrix<int>[100, 100] mm;

//Making a vector by providing values
vector<int> v = [1, 0];
//Makes a vector with 100 entries all set to 0
vector<int>[100] vv;
\end{lstlisting}

\subsection*{Operators}\label{subsec:operators}
Available operators for simple data types in the language can be found on \myref{tbl:operators} with a short description.  

%It is not possible to use operators with matrices or vectors, functions which support different operations on these types will be included in a library.
\begin{table}[h]
    \centering
    \colorlet{shadecolor}{gray!40}
    \rowcolors{1}{white}{shadecolor}
    \begin{tabular}{|c|l|l|l|l|l|}
    \hline
    \textbf{Operator}  & \textbf{Integer}                   & \textbf{Floating point number}    & \textbf{Boolean}       \\ \hline
    +                  & Addition (infix)                   & Addition (infix)                  & Type error             \\ \hline 
    -                  & Subtraction (infix)                & Subtraction (infix)               & Type error             \\ \hline 
    ++                 & Increment (unary,pre-, postfix)    & Increment(unary,pre-, postfix)    & Type error             \\ \hline    
    - -                  & Decrement (unary,pre-, postfix)    & Decrement(unary,pre-, postfix)    & Type error             \\ \hline
    *                  & Multiplication (infix)             & Multiplication(infix)             & Type error             \\ \hline
    \%                 & Remainder (infix)                  & Modulo(infix)                     & Type error             \\ \hline
    /                  & Division (infix)                   & Division (infix)                  & Type error             \\ \hline
    \&\&               & Type error                         & Type error                        & Logical and (infix)    \\ \hline 
    ||                 & Type error                         & Type error                        & Logical or (infix)     \\ \hline 
    ==                 & Equality (infix)                   & Equality* (infix)                 & Type error             \\ \hline 
    !=                 & Inequality (infix)                 & Inequality* (infix)               & Type error             \\ \hline
    <=                 & Less than or equal to (infix)      & Less than or equal to* (infix)    & Type error             \\ \hline
    >=                 & More than or equal to (infix)      & More than or equal to* (infix)    & Type error             \\ \hline
    <                  & Less than (infix)                  & Less than (infix)                 & Type error             \\ \hline
    >                  & More than (infix)                  & More than (infix)                 & Type error             \\ \hline
    \end{tabular}
    \caption[List of operators on primitive data types in \gls{gamble}.]{List of operators on primitive data types in \gls{gamble}.\@*Equality comparisons on floating point number may be incorrect.}\label{tbl:operators}
\end{table}
%\vspace{-20pt}
Some operators will work with vectors and matrices.
Since the elements in vectors and matrices are one of the simple types of \gls{gamble}, the operands on both sides of an infix operator must adhere to the rules of type compatibility.
The list of possible operands on these complex data types can be seen on \myref{tbl:matOps}.
It is not shown in the table but the $+$ and $-$ operations in \myref{tbl:matOps} requires the matrices and vectors to have the same dimensions.\todo{Enten vis det i table't eller blot nævn det i brødtxt MP}
The $*$ operation requires one matrix to have the same row size as the other matrix' column size. Ex. a matrix [j,k] and a matrix [k,h] can be multiplied because they have the same size k.
This will result in a [j,h] sized matrix.
The use of operators between matrix and vector instances is influenced by matlab which uses the same concept with great success.
% Please add the following required packages to your document preamble:
% \usepackage{multirow}
% \usepackage[table,xcdraw]{xcolor}
% If you use beamer only pass "xcolor=table" option, i.e. \documentclass[xcolor=table]{beamer}
\begin{table}[ht]
\centering
\begin{tabular}{|c|l|l|l|l|}
\hline
\textbf{Operator}   & \textbf{Operation}                        & \textbf{Operand}                 & \textbf{Operand}                  & \textbf{Result}                 \\ \hline
                    & \cellcolor{gray!40}Matrix multiplication  & \cellcolor{gray!40}Matrix        & \cellcolor{gray!40}Matrix         & \cellcolor{gray!40}Matrix       \\ \cline{2-5} 
                    & Matrix scale                              & Matrix                           & Numeral                           & Matrix                          \\ \cline{2-5}
                    & \cellcolor{gray!40}Dot product            & \cellcolor{gray!40}Vector        & \cellcolor{gray!40}Vector         & \cellcolor{gray!40}Vector       \\ \cline{2-5} 
\multirow{-4}{*}{*} & Vector scale                              & Vector                           & Numeral                           & Vector                          \\ \hline
                    & \cellcolor{gray!40}Matrix addition        & \cellcolor{gray!40}Matrix        & \cellcolor{gray!40}Matrix         & \cellcolor{gray!40}Matrix       \\ \cline{2-5}
                    & Add numeral to each index                 & Matrix                           & Numeral                           & Matrix                          \\ \cline{2-5}
                    & \cellcolor{gray!40}Vector addition        & \cellcolor{gray!40}Vector        & \cellcolor{gray!40}Vector         & \cellcolor{gray!40}Vector       \\ \cline{2-5} 
\multirow{-4}{*}{+} & Add numeral to each index                 & Vector                           & Numeral                           & Vector                          \\ \hline
                    & \cellcolor{gray!40}Matrix substraction    & \cellcolor{gray!40}Matrix        & \cellcolor{gray!40}Matrix         & \cellcolor{gray!40}Matrix       \\ \cline{2-5}
                    & Substract numeral from each index         & Matrix                           & Numaral                           & Matrix                          \\ \cline{2-5}
                    & \cellcolor{gray!40}Vector substraction    & \cellcolor{gray!40}Vector        & \cellcolor{gray!40}Vector         & \cellcolor{gray!40}Vector       \\ \cline{2-5} 
\multirow{-4}{*}{-} & Substract numeral from each index         & Vector                           & Numeral                           & Vector                          \\ \hline
\end{tabular}
\caption{List of operators on complex data types in \gls{gamble}. \todo[inline]{Ingen division? -- TK}}\label{tbl:matOps}
\end{table}
%\subsection*{Memory handling}
%Locally declared and small and simple variables will be written to the stack, while large quantities of data must be written to the heap. 
%To manage the heap in a programming language one must decide how the management must be implemented.
%At the topmost level there exist to different approaches, the first being explicit memory handling as seen in C with function calls like \texttt{alloc()} and \texttt{free()}. 
%The second approach is dynamic memory handling, often called garbage collection.
%While  manual memory handling is a very low level concept, it has some advantages over dynamic memory handling, for instance manual handling gives the users full control over a programs resources and while this can be hard o get right, can both increase the simplicity of the compiler and increase a programs runtime execution speed because garbage collection can be a complex runtime process.\citep{Sebesta, gribble}          
