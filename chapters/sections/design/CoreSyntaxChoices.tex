\section{Core syntax choices}
The source code must be encoded in UTF-8, but only the ASCII alphabet and numbers are allowed for identifiers and values.
The language is case sensitive, because the most used languages seen on TIOBEs index are also case sensitive. \citep{TIOBE}
When a language is case sensitive it is possible for the programmer to name different parts of the code like he pleases, resulting in higher readability. 
Some languages, such as Python, uses whitespace and indentation to indicate scope, in \gls{gamble} scope is indicated by curly brackets like many existing languages\texttt{\{\}}.

\subsection*{Comments}
It is useful to be able to write comments in code to annotate the meaning of some code.
If a language were not to have the possibility to write comments, then the source code could be hard to understand for those not involved in the code writing process.
Furthermore source code without comments might be difficult to return to after some time away from it, since the programmer would have to reread the entire code, instead of a few well written comments. \citep{Commenting}
Therefore both single line and multi line comments can be used in \gls{gamble}. 
``//'' is used for single line comments, meaning that everything after the ``//'' until the next newline is ignored by the compiler, also known as C++ style comments. 
``/* */'' is used for multi line comments, meaning that everything between ``/*'' and ``*/'' is ignored by the compiler, also known as C style comments. \citep{C_comment,Cplus_comment}

\subsection*{End of statement terminator}
When choosing an end of statement terminator there are two common choices.
The newline character as a terminator or another symbol, often semicolon (\texttt{;}).
The use of newline is often used in high level languages e.g. MATLAB or other mathematical oriented languages.
We deem the primary value of using the newline is simplicity and a decrease in errors due to forgotten semicolons. 
Semicolon is as mentioned the other choice, this choice has been used in many programming languages throughout the history. 
The main advantage of an end of statement terminator like semicolon is that it allows freeform code. 
This freeform allows to have multiple statements in one line or allows long statements to span across multiple lines.
We deem this to enhance the readability of the code.
Furthermore the semicolon terminator allows for a simpler syntax and translator design.
As mentioned before forgotten semicolons can result in parsing errors, we deem this as the biggest disadvantage of this end of statement terminator.

Based on this analysis we have chosen to use the semicolon as end of statement terminator, for easier parsing and allowing freeform coding style.

\subsection*{Scope}\label{subsec:Scope}
Scope rules define what variables and functions are available at a given point in a program.
A single scope would mean that any variable or function can be used form any place in the code. 
Scope rule gives the programmer more control of the code and reduces the chance of changing variables by accident.
There are two dominantly used methods of scoping, static and dynamic scoping.
Most of the popular languages on TIOBE's list of popular programming languages use static scoping. \citep{TIOBE}
\gls{gamble} will use this form of scoping because we find it to be the easiest to work with and less prone to errors than dynamic scoping.
The scope rules of \gls{gamble} are as follows:

A variable is in scope from its declaration until the end of the block it is declared in.
An inner block, such as a control flow structure, see \myref{subsec:control-flow} for more, has access to the variables from the outer scope, above the inner block. 
It is also not allowed to redeclare a variable which is already declared in the current scope or any enclosing scopes. 
Additionally any function is in scope of any other function including itself, this also allows recursion. 
This is static scoping without information hiding. 

\subsection*{Structure}\label{subsec:Struc}
The structure of the source code of a \gls{gamble} program is shown in \myref{lst:Structure}.
To increase readability the source code is split up in segments.
A single segment contains only a small part of source code.
The inclusion of other files are, like in many other languages, written in the top segment, this is to make it easy to get an overview of any and all included libraries. 
All function declarations are written in the next segment followed by the main statements which is the last segment.
This collects the main statements, which makes it easier for a programmer to get an overview of the order of execution, compared to if the main statements were allowed between the function declarations.
This sectioning also makes it easier to write a context free grammar for the language. 
A context free grammar is essentially the grammar which specifies what is and what is not an acceptable statement in a language. This will further be explained \myref{chap:syntax}.

\begin{lstlisting}[caption={Source code file layout in \gls{gamble}},frame=tlrb,label={lst:Structure}, numbers=none]
Libary inclusions

Function declarations

Statements
\end{lstlisting}                           