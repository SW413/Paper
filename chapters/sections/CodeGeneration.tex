\chapter{Code Generation}
Code generation is the phase in which the object code is generated, the process and considerations this entails are covered in this chapter.
The object code is the output of the compiler.
Once the source code has passed through syntax analysis and contextual analysis without errors it has been validated and the compiler can proceed with generating the object code.
In the case of this compiler the object code is OpenCL C as a result one or more compilers must use this code as source code to eventually end up with machine code that can be executed.
Since the object code is OpenCL C means that tasks such as instruction selection and scheduling as well as register allocation will be handled by the compiler which will compile the OpenCL C code rather than \gls{gamble}.
Furthermore in the code generation phase optimisation of the object code also takes place, for \gls{gamble} this means to create object code which is quickly executed and utilises the \acrshort{gpu} for calculation that benefit from its use.
A diagram showing the phases of the code generation can be seen in \myref{fig:flowCodegen}.

\vspace{10pt}
\begin{figure}[h]
    \centering
    \begin{tikzpicture}[node distance = 3cm, auto]
        %\node (invi1) [invi, draw=none] {};
        %\node (ast) [lille, below=-0.35cm of invi1] {Abstract Syntax Tree};
        %\node (symboltable) [lille, minimum width=6.75cm, minimum height=2.4cm, right=2cm of invi1, fill=blue!10, label={[xshift=0cm, yshift=-1cm]Symbol Table}] {};
        %\node (scope) [lille, right=1.1cm of ast] {Scopechecker};
        %\node (type) [lille, right=0.7cm of scope] {Typechecker};
        %\node (dast) [lille, right=1.1cm of type] {Decorated Abstact Syntax Tree};

        %\node (error) [cloud, below=1cm of symboltable] {Error report};

        %\draw [arrow] (ast) -- (scope);
        %\draw [arrow] (scope) -- (type);
        %\draw [arrow] (type) -- (dast);
        %\draw [arrow,dashed] (scope) -- (error);
        %\draw [arrow,dashed] (type) -- (error);
        %\draw [arrow,dashed] (symboltable) -- (error);

        \node (dast) [lille, align=left] {Contextual \\Analysis Phase};
        \node (cgv) [lille, right=0.7cm of dast, align=left] {Code Generation \\Visitor};
        \node (copy) [lille, right=0.7cm of cgv] {Output \gls{opencl} C code};
        \node (error) [invi, draw=none, minimum width=2cm, right=1cm of copy, label={[xshift=40pt, yshift=-17pt]Finished Compilation}] {};

        \draw[black,fill=black, above=1cm of parser] (10.9,0) circle (1ex);
        \draw[black, above=1cm of parser] (10.9,0) circle (1.3ex); 

        \draw [arrow] (dast) -- (cgv);
        \draw [arrow] (cgv) -- (copy);
        \draw [arrow] (copy) -- (error);
    \end{tikzpicture}
    \caption{State diagram showing the modules of the code generation. } 
    \label{fig:flowCodegen}
\end{figure}
\vspace{-20pt}
