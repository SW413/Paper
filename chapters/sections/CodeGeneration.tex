\chapter{Code Generation}
Code generation is the phase in which the object code is generated, the process and considerations this entails are covered in this chapter.
Once the source code has passed through syntax analysis and contextual analysis without errors it has been validated and the compiler can proceed with generating the object code.
In the case of this compiler the object code is OpenCL C as a result one or more compilers must use this code as souce code to eventually end up with machine code that can be executed.
Having the object code be OpenCL C means that tasks such as instruction selection and scheduling as well as register allocation will be handled by the compiler which will compile the OpenCL C code rather than \gls{gamble}.
Furthermore it is also in the code generation phase that optimisation commonly takes place, for \gls{gamble} this means to create object code which is quickly executed and utilises the \acrshort{GPU} for calculation that benefit from its use.