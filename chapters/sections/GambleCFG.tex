\section{Context Free Grammar in \gls{gamble}}
\gls{gamble}'s grammar is written in \acrfull{ebnf}, and uses regular expressions to define terminals such as numbers and identifiers.
The full \acrshort{cfg} of \gls{gamble} can be found in \myref{app:CFG} alongside the lexing rules.
This section only presents a small selection of the \acrshort{cfg}.
\gls{gamble}'s \acrshort{cfg} is used in the compiler for creating a parser for the translator.
The production rules are used to check for any syntactical errors which might occur.

A production rule from the \acrshort{cfg} of \gls{gamble} is the statement rule. 
\myref{lst:statements} shows that a statement can be either an assignment, declaration, functioncall, controlblock or a loop construct. 
The Statement's productions are the building blocks of any source code written in \gls{gamble}.

\begin{lstlisting}[caption={\acrshort{cfg} for Statements in \gls{gamble}},frame=tlrb,label={lst:statements},numbers=none]
statement
    : assignment ';'
    | declaration ';'
    | functioncall ';'
    | controlblock 
    | loop
    ;
\end{lstlisting}

All production rules in the statement each expand to their own production hence they are non-terminals.
Looking at the declaration production on \myref{lst:declaration} observe that it has three production rules.
The first two end on an expression non-terminal.

\begin{lstlisting}[caption={\acrshort{cfg} for Declarations in \gls{gamble}},frame=tlrb,label={lst:declaration},numbers=none]
declaration                                           
    : valueType ID '=' expression                     
    | complexdatatype ID '=' expression               
    | complexdatatype '[' entranceCoordinate ']' ID   
    ; 
\end{lstlisting}

Further expanding into the expression production rules, it can be seen that it has six different production rules. 
An expression has production rules expanding the expression into a construct containing more expressions.
This is done because an expression can derive to a value, and several values are needed for multiple arithmetic operations.
This construct also causes left-recursion, something often attempted to be avoided in \acrshort{cfg}.
However some parser generators can implicitly rewrite the \acrshort{cfg} to not use left recursion.
\begin{lstlisting}[caption={\acrshort{cfg} for Expressions in \gls{gamble}},frame=tlrb,label={lst:expression},numbers=none]
expression                                            
    : expression '^' expression                       
    | expression ( '*' | '/' | '%' | '#' ) expression 
    | expression ( '+' | '-' ) expression             
    | '(' expression ')'                              
    | value                                           
    | ID postUnaryOperator                            
    ; 
\end{lstlisting}
  
The value production rule can expand into what is seen on \myref{lst:value}.
Some of the right side elements of the production evaluates to a terminal, e.g. ID evaluates to a terminal.
Terminals are elements that have no further production rules, leading to a ``dead end''.
When a terminal is reached the derivation of the production ends and the terminal becomes a leaf on the parse tree, also seen in \myref{sec:AST}.
\begin{lstlisting}[caption={\acrshort{cfg} for Values in \gls{gamble}},frame=tlrb,label={lst:value},numbers=none]
value                                      
    : ID                                   
    | constant                             
    | '[' valueList ( ';' valueList )* ']' 
    | functioncall                         
    | collectionEntrance                   
    | BOOLVAL                              
    ;                        
\end{lstlisting}