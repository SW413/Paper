\section{Context free grammar in gamble}
Our grammar is written in \acrfull{ebnf}, and uses \acrfull{regex}s to define terminals such as numbers and identifiers.
In this section there will be a short traversal of a branch from the parse tree.
The full \acrshort{cfg} of \gls{gamble} can be found on \todo{ref til appendix} alongside the lexing rules.
This section only presents a small selection of the \acrshort{cfg}.

A production rule from the \acrshort{cfg} of \gls{gamble} is the statement. 
Looking at \myref{lst:statements} it is seen that a statement can be both an assignment, declaration, functioncall, controlblock or a loop construct. 
Statement productions are the building blocks of any sourcecode written in \gls{gamble}.

\begin{lstlisting}[caption={\acrshort{cfg} Statement},frame=tlrb,label={lst:statements},numbers=none]
statement
    : assignment ';'
    | declaration ';'
    | functioncall ';'
    | controlblock
    | loop
    ;
\end{lstlisting}

All production rules in the statement each expand to their own production hence they are non-terminals.
Looking at the declaration production on \myref{lst:declaration} observe that it has two production rules, ``datatype '=' expression'' and ``complexdatatype ID '=' expression''. 
They both end on the expression production which can be seen on \myref{lst:expression}.

\begin{lstlisting}[caption={\acrshort{cfg} Declaration},frame=tlrb,label={lst:declaration},numbers=none]
declaration
    : datatype ID '=' expression                        #primitiveDecl
    | complexdatatype ID '=' expression                 #complexDecl
    ; ;
\end{lstlisting}

Further expanding into the expression production, it can be seen that it has five different production rules.
An expression has production rules expanding the expression into a construct containing more expressions.
This is done because an expression can derive to a value, and several values are needed for multiple arithmetic operations.
This construct also causes left-recursion, something often attempted to be avoided in \acrshort{cfg}.
The parse generator ANTLR can handle left-recursion and as such it is acceptable in the \acrshort{cfg} for \gls{gamble}.
\begin{lstlisting}[caption={\acrshort{cfg} Expression},frame=tlrb,label={lst:expression},numbers=none]
expression
    : expression ( '*' | '/' | '%' ) expression     #mulExpr
    | expression ( '+' | '-' ) expression           #addExpr
    | '(' expression ')'                            #parenExpr
    | value                                         #valueExpr
    | ID postUnaryOperator                          #postIDExpr
    ;
\end{lstlisting}
  
The value production expands into what is seen on \myref{lst:value}.
In this production both terminals and non-terminals are included.
The terminals are elements that have no further production rules, leading to a ``dead end''.
When a terminal is reached it is used and is as a leaf on a tree, also seen in .\todo{Kilde til afsnittet med billeder Søren har lavet.}
\begin{lstlisting}[caption={\acrshort{cfg} Value},frame=tlrb,label={lst:value},numbers=none]
value
    : ID                                     #valID
    | constant                               #valConstant
    | '[' valueList ( ';' valueList )* ']'   #valList
    | functioncall                           #valFuncCall
    | collectionEntrance                     #valCollectionEntrance
    | BOOLVAL                                #valBool
    ;
\end{lstlisting}

Going through this small part of the \acrshort{cfg} one single branch of the parse tree have been unfolded.
\gls{gamble}s \acrshort{cfg} is then used in the parser.
The parser takes the input from a lexical analyzer, the source code in the form of a token stream.
The parser uses the \acrshort{cfg}s production rules to determine syntax errors in the source code.
The parser then outputs a parse tree.
How this is done is described further in \myref{sec:syntaxAnalysis} and \myref{subsec:parser}.
