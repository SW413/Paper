\subsection*{Parser}
The parser is based on formal syntax specification such as a CFG. 
It reads tokens and groups them into phrases according to the syntax specification.
The Parser verifies the syntax, and if an syntax error is found it issues an corresponding error message.
 \citep{Crafting}
By using a parser generator like ANTLR or SapleCC handling of syntactic error and repairs can be done automatically.
An parser can also be written manual but doing so can result in syntactic errors that is hard to locate or solve.
We have been using ANTLR as our parser generator tool, for ease of use.

There are different kind of parsers, both bottom-up and top down parsers.
We have chosen to make a top- down parser, more specific a recursive descent parser.
A recursive decent parser is a kind of top-down parser build from a set of mutually recursive procedures where each such procedure usually implements on of the productions of the grammar.
The structure of the resulting program closely mirrors the grammar it recognizes.\citep{Recursive_programming}




\subsubsection*{Visitor Pattern}
We have chosen the visitor pattern which is a alternative to the listener pattern.
Both of the patterns are parse-tree listeners and visitors to build language applications.
The biggest difference between listeners and visitors is that listener methods are not explicitly responsible for calling methods to walk every children.
The visitor pattern on the other hand must walk the children in the parse-tree to keep the tree traversal going.
Visitor get to control the tree traversal and in what order the tree is visited because of these explicit calls to visit children.\citep{ANTLR4_Book}

Using the visitor pattern we can keep application-specific code out of out grammar.
It is possible, with the visitor pattern, add functionality to pass around return values and arguments.
