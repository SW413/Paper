\subsubsection*{From OpenCL to execution}\label{makefile}
After the \gls{gamble} compiler have compiled the source code; the compiler outputs OpenCL C code as its object code.
The OpenCL C code must be compiled by a C-compiler such as the GCC (GNU Compiler Collection) compiler to produce code the \acrshort{cpu} and \acrshort{gpu} can execute.
During the compilation the OpenCL C headers must be accessable, to provide the functions and types used. 
Aditionally the \gls{gamble} standard libary, containing functions such as \texttt{matrixToFile} and \texttt{fileToMatrix} must also be included and compiled. 
This produces an executable file for the platform targetted by the compiler, this will most often be the platform running, such as x86--64 on Linux.
The same executeable file is not usable for another platform, e.g. ARM, nor on another operating system, e.g. Windows or OS X. 
To maintain portability a ``MakeFile'' is used; a MakeFile is a file in the syntax which the GNU make utility can execute. 
The make utiliry is a set of rules which ensures that the correct compiler, path, headers, etc. are used during compilation on one of multiple platforms. 
This means that after running the \gls{gamble} compiler, on a configured computer, e.g. has OpenCL C headers, libary, a compiler etc., the make command can be used to build a binary file to be executed.\todo{der skal nok et kilde på det her}

The kernel can either be compiled at compile-time or Just-In-Time (JIT) compiled at runtime.
The JIT compilation, called online compilation under OpenCL, allows the system to adapt with platform specific optimisations for the running platform.
If the kernel is compiled at compile-time, called offline compilation under OpenCL, then it might support fewer devices, or will take a long time and increase the size of the executable as it has to contain versions for each hardware for which support is requested. \citep{openclbookjit}
For \gls{gamble} the online compilation is chosen for its simplicity as it lets the driver of the running system compile the kernel. 

A typical compilation on Ubuntu 14.04 configured with OpenCL headers located at \path{/opt/opencl-headers/include/} will look like \myref{lst:makecommands}, where \texttt{code.c} is the output of the \gls{gamble} compiler.
After the execution of these, an output file called \texttt{run} is made and can be executed with the command \texttt{./run}. 

\begin{lstlisting}[caption=The commands executed by the make command according to the rules of the MakeFile,numbers=none,frame=tlrb,label={lst:makecommands}]
gcc -Wall -Wextra -pedantic -O3 -std=c99 -I/opt/opencl-headers/include/ -c code.c simpleCL.c gambleStdlib.c
gcc -Wall -Wextra -pedantic -O3 -std=c99 *.o -o run -lm -lOpenCL -lrt
rm -f *.o
\end{lstlisting}
