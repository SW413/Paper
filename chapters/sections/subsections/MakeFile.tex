\subsubsection*{From OpenCL to execution}
After the \gls{gamble} compiler have completed the output is OpenCL C code.
This is compiled by a C-compiler such as GCC to produce code the \acrshort{cpu} and \acrshort{gpu} can execute.
During the compilation the OpenCL headers must be accessable, this is to provide the functions and types used. 
Aditionally the \gls{gamble} standard libary, containing functions such as \texttt{matrixToFile} and \texttt{fileToMatrix} must also be included and compiled. 
This produces an executable for the platform targetted by the compiler, this will most often be the platform running, such as x86-64 on Linux.
The same executeable is not usable for another platform, e.g. ARM, or on another operating system, e.g. Windows or OS X. 
However to maintain some crossplatform portability a ``MakeFile'' is used. 
A MakeFile is a file in the syntax which the make utility can execute. 
It is a set of rules which ensures that the correct compiler, path, headers, etc. are used during compilation on one of multiple platforms. 

This means that after running the \gls{gamble} compiler, on a configured computer, e.g. has a OpenCL compadable unit, a compiler etc. , the make command can be used to build a binary to execute.

The compilation of the kernels are done at runtime, this is in order to optimise them for the unit used for the execution. 
This unit might be unknown at the compile-time of the rest. 