\subsection*{GPU Usage}\label{GPUCode}
Since \gls{gamble} distances its programmers from directly controlling which computations are performed on the \acrshort{gpu}, determing what code to perform on the \acrshort{gpu} becomes a problem for the compiler to solve.\todo{task, lyder bedre eller hvad? MP. - tasks synes jeg ikke rigtigt passer en når det er hvilke kodesegmenter, måske operations hvis det skal ændres? - Marc}
To make this decision the compiler must know what kind of code performs faster on a \acrshort{gpu} than a \acrshort{cpu}.

From \myref{sec:comparch} it is clear that for it to make sense to move any computation to the \acrshort{gpu} it must be of significant size to make up for the overhead of moving data, and be executable in parallel.
If a computation is reliant on the outcome of other computations, the Fibonacci function as an example, moving it to the \acrshort{gpu} would be a significant decrease in performance compared to a \acrshort{cpu}.

Any code written in a recursive format will not be run on the \acrshort{gpu}. 
Furthermore due to the overhead in data transfer, only computations requiring a significant amount of operations to be performed should be executed on the \acrshort{gpu} as \myref{image:benchmark} shows.
Therefore statements which only contain simple data types, i.e. integers, floats and booleans, are performed on the \acrshort{cpu}.
An example could be \texttt{value = value1 + value2}, where all types are integers.
%Therefore statements not containing complex data types, i.e. statements with no vector or matrix arithmetics, are also performed on the \acrshort{gpu}.

However statements that do include matrix or vector arithmetics will be performed on the \acrshort{gpu}.
An example could be matrix multiplication.
Now it is entirely possible to make a matrix multiplication of a $2\times2$ matrix, which would be so small that the overhead of data transfer is more expensive than simply computing on the \acrshort{cpu}. \todo{Det er jo ikke kune data overførslen som koster tid? Der er også JIT af kernel i nogle cases (se alle). -- Troels}
However to simplify the code generation it has been chosen that all matrix or vector calculations, are to be done on the \acrshort{gpu}.
This is not always the best choice, as \myref{sec:comparch} clearly shows, but it requires less analysis of the code given, furthermore as mentioned in \myref{sec:phil} \gls{gamble} uses the \acrshort{gpu} to gain computational power for performing already developed algorithms on data sets big enough to see an improvement in execution time.
Also even if smaller matrices are being computed, the increase in runtime is significantly less than the decrease for just a single bigger matrix operation, as can be seen on \myref{fig:test_results}.
\todo[inline]{Måske nævne vi hellere vil flytte små matricer på gpu, end at beholde store på cpuen? MP - God ide, den tid det tager at flytte en lille matrix til GPU er heller ikke så stor som det man spare på at flytte store matricer til gpu'en. - Søren - Som ovenstående? - Marc}

