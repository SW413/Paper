\subsection*{GPU Usage}\label{GPUCode}
Since \gls{gamble} distances its programmers from directly controlling which computations are performed on the \acrshort{gpu}, determining what code to perform on the \acrshort{gpu} becomes a problem for the compiler to solve.
To make this decision the compiler must know what kind of code performs faster on a \acrshort{gpu} than a \acrshort{cpu}.

From \myref{sec:comparch} it is clear that to make sense in moving any computation to the \acrshort{gpu} it must be of significant size to make up for the overhead of moving data, and be executable in parallel.
If a computation is reliant on the outcome of other computations, the Fibonacci function as an example, moving it to the \acrshort{gpu} would be a significant decrease in performance compared to on a \acrshort{cpu}.

Any code written in a recursive format will not be run on the \acrshort{gpu}. 
Furthermore due to the overhead in data transfer, only computations requiring a significant amount of operations to be performed should be executed on the \acrshort{gpu} as \myref{image:benchmark} shows.
Therefore statements which only contain simple data types, i.e. integers, floats and booleans, are performed on the \acrshort{cpu}.
An example could be \texttt{value = value1 + value2}, where all types are integers.
%Therefore statements not containing complex data types, i.e. statements with no vector or matrix arithmetics, are also performed on the \acrshort{gpu}.

However statements that do include matrix or vector arithmetics will be performed on the \acrshort{gpu}.
An example could be matrix multiplication.
Now it is entirely possible to make a matrix multiplication of a $2\times2$ matrix, which would be so small that the overhead of data transfer is more expensive than simply computing on the \acrshort{cpu}.
However to simplify the code generation it has been chosen that all matrix or vector calculations, are to be done on the \acrshort{gpu}.
This is not always the best choice, as \myref{sec:comparch} clearly shows, but it requires less analysis of the code given, furthermore as mentioned in \myref{sec:phil} \gls{gamble} uses the \acrshort{gpu} to gain computational power for performing already developed algorithms on data sets big enough to see an improvement in execution time.

