\subsection*{Abstract Syntax Tree}
ANTLR creates a parse tree, which contains a node for all the productions of \gls{gamble}'s grammar.
This tree simply has too much information, and it makes it difficult to traverse the tree and check for \gls{gamble}'s semantics.
Therefore the tree will be transformed into what is called an \acrfull{ast}.
The \acrfull{ast} should contain the same information as a parse tree, and therefore one should be able to make a pretty printer from it.
A pretty printer, is a program which takes the \acrfull{ast} as input, and prints the sourcecode again.
If this is possible the \acrfull{ast} has all the information it needs.
A transformation from a parse tree to an \acrfull{ast}, on the declaration: \texttt{int a = 5;} can be seen on \myref{image:AST}

\begin{figure}
		\centering
	 	\includegraphics[width=0.8\linewidth]{figures/Trees/AST.PNG}
		\caption{The tree on the left is the parse tree, and the tree on the right is the AST, which still contains all the information from the parse tree.} \label{image:AST}
\end{figure}

As mentioned the parse tree is generated by ANTLR, which means we cannot decide what information is contained on the nodes of the tree.
The nodes of the \acrfull{ast} can therefore be made to contain information for type and scope checking, which makes those phases a lot easier.
All that is needed to make this transformation, is to make classes for all the nodes of the tree, and create a way to traverse the tree.
The next section will present ways of traversing trees, while performing different computations like scope and type checking on the tree.


