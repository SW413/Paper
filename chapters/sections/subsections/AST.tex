\subsection*{Abstract Syntax Tree}\label{sec:AST}
The parser creates a parse tree which contains a node for each production of \gls{gamble}'s grammar.
This tree contains unneeded information, which makes it difficult to traverse the tree and check for the semantics of \gls{gamble}.
Therefore the tree will be transformed into what is called an \acrfull{ast}.
An \acrshort{ast} should be able to express the same source code as its parse tree, and therefore one should be able to make a pretty printer from it.

The compiler developed by the project group implements a pretty printer.
A pretty printer is used to ensure that no information is lost in the process of converting the source code into a parse tree followed by the conversion to an \acrshort{ast}.
It does so by traversing the \acrshort{ast} and outputting the original source code except for code which is not parsed, e.g. whitespace and comments.
This output should then be a valid input for the compiler, if this holds true it also serves to prove that the pretty printer indeed working as intended.
\todo[inline]{Dette prettyprinter er noget sjovt placeret. MP - Sýnes nu det er meget fint, vi nævner at man skal kunne lave en pretty printer lige inden, så bør vi vel også forklare hvad det er :) ? - Søren}
A transformation from a parse tree to an \acrshort{ast}, on the declaration: \texttt{int a = 5;} (using the grammar of \gls{gamble}) can be seen on \myref{image:AST}

\begin{figure}
		\centering
	 	\includegraphics[width=0.8\linewidth]{figures/Trees/AST.PNG}
		\caption{The tree on the left is the parse tree, and the tree on the right is the AST, which still contains all the information from the parse tree.}\label{image:AST}
\end{figure}

The parse tree is generated using a parser produced by compiling the grammar with ANTLR, which means we cannot decide which information is contained on the nodes of the tree.
The nodes of the \acrshort{ast} can therefore be made to contain information for type and scope checking, which helps in the contextual analysis phase.
To make this transformation the following is needed: classes for all the nodes of the \acrshort{ast}, a way to traverse the parsetree, and then using the traversal to create instances of the \acrshort{ast} nodes and binding them together to create a structured \acrshort{ast}.



