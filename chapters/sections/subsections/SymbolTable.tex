\section{Symbol Table}
To conduct scope checking and type checking in the compiler; the compiler needs to store data about the identifiers, variables and functions in a data structure.
An excellent data structure for this is a symbol table.
Storing all this information in a symbol table makes it possible for the compiler to conduct the contextual analysis.

A symbol table is a table-like data structure which contains data regarding which contexts variables and constants are used in.
While different ways of implementing symbol tables exist, there exist no viable alternative to this data structure in our context.\todo{Der blev ellers introduceret en masse i SPO ? Hvorfor er de ikke viable ? -  Søren}

Two common ways of implementing a symbol table is as follows; either there is one table which contains every identifier or there is a symbol table for each scope in the program.
If constrained by memory, having a single symbol table can be beneficial, however having multiple symbol tables can simplify the code at a memory cost. 

In \gls{gamble} there exist scopes, and for each of these scopes there is a corresponding symbol table. 
Scopes inherit from each other so a scope can enclose another scope. 
The outermost scope, the global scope, is where functions are declared; every scope is either directly enclosed by this scope or recursively.
This means that every function can be called from anywhere within the source code, including within its own function declarations allowing for recursive functions.

