\section{Symbol Table}
In a compiler it is useful to store information about the identifiers, variables and functions in a data structure. 
This information can be useful for scope checking and type checking.

There are two common ways of implementing a Symbol Table; either you have one table which contains every identifier or you have a Symbol table for each scope. 
If constrained by memory, having a single symbol table can be beneficial, however having multiple symbol tables can simplify the code at a memory cost. 

In \gls{gamble} there exists scopes, and for each of these scopes there is a corresponding symbol table. 
Scopes inherit from each other so a scope can enclose another. 
The outermost scope, the global scope, is where functions are declared; every scope is either directly enclosed by this scope or recursively.
This means that every function can be called from anywhere within the source code --- including within its own function declarations; allowing for recursive functions.

