\section{Symbol Table}
To conduct scope checking and type checking in the compiler, the compiler need to store data about the identifiers, variables and functions in a data structure.
An excellent data structure for this is a symbol table.
Storing all this information in a symbol table makes it possibly for the compiler to conduct the contextual analysis.

An symbol table is an table like data structure which contains data regarding which context variable and constants is used in.
While there exist different ways of implementing symbol tables, there exist no viable alternative to this data structure in our context.

Two common ways of implementing a Symbol Table is as follows; either you have one table which contains every identifier or you have a Symbol table for each scope. 
If constrained by memory, having a single symbol table can be beneficial, however having multiple symbol tables can simplify the code at a memory cost. 

In \gls{gamble} there exists scopes, and for each of these scopes there is a corresponding symbol table. 
Scopes inherit from each other so a scope can enclose another. 
The outermost scope, the global scope, is where functions are declared; every scope is either directly enclosed by this scope or recursively.
This means that every function can be called from anywhere within the source code --- including within its own function declarations; allowing for recursive functions.

