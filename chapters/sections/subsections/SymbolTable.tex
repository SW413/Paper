\subsection*{Symbol Table}
In a compiler it is useful to store information about the identifiers, variables and functions in a data structure. 
This information can be useful for scope checking and type checking.

There are two core ways of having a Symbol Table, either have one single table for every identifier or have one for each scope.
It is in many cases more simple to use if every scope has its own symbol table. \todo{godt argument}

In \gls{gamble} there exists scopes, and for each of these scopes there is a corresponding symbol table. 
Scopes inherit from each other so a scope can enclose another. 
The outermost scope, the global scope, is where functions are declared, every scope is either directly enclosed by this scope or recursively.
This means that every function can be called from anywhere within the source code. 

In \gls{gamble} the class SymbolTable represents the symbol table.
The core constituent of this class is the ArrayList of the Scope class, called allScopes, meaning that every scope is stored in this ArrayList.
Every scope contains Map of Symbols and strings as keys, and information about the scope such what scope it is enclosed by. 
The key represents the name of the symbol, and must be unique to the scope and not found in an enclosing scope, as this could cause an ambiguity to arise. 
A symbol is either a variable or a function, in the class Symbol its data type, name and scope is stored. 

