\subsection*{Scope Checking}
In \myref{subsec:Scope} the scope of identifiers, variables and functions, are defined for \gls{gamble}.
A variable is in scope from its declaration until the end of the block it is declared in.
And an inner scope inherits the identifiers declared in the outer scopes. 

In the contextual analysis part of the compiler it is central to verify that each variable and function used are in scope, and it should produce a useful error message.
The error message should indicate which identifiers is not in scope, and what line this identifiers is used wrongly.

To check this every reference to identifiers must be checked to see if they matches an identifiers in the symbol table of the current scope, and the scopes which enclose it. 
Furthermore if it important that any usage of a identifiers is after its declaration.

The scope checker produces two errors: redeclaration error and undeclared error.
A redeclaration error is an attempt to declare a variable while it is already in scope.
A undeclared error is the attempt to use a variable which is not declared in the current scope. 
Examples shown in \myref{lst:scopeErrors}.

\begin{lstlisting}[caption=Examples of scope errors in \gls{gamble}, numbers=none,frame=tlrb,label={lst:scopeErrors}]
/* [...] */
int a = 1;
float a = 2.2;   /* Redeclaration error */
int a = 2;       /* Redeclaration error */ 

b = 2;           /* Undeclared error */
b = foo();       /* Redeclaration error and undeclared error */ 
/* [...] */
\end{lstlisting}

These errors are found when the symbol table is being filled by the SymbolTableFillVisitor ... \todo{Er det her implementering?}