\subsection*{Type Checking}
The second important part of the contextual analysis phase for the compiler is the type checking which enforces the type system of \gls{gamble}.
As \gls{gamble} is statically typed it is necessary to check if all references to identifiers and constant values fit into the context they exist in. 
Since implicit conversion between floating point and integer types is not a part of gamble an error must be issued everywhere they are used wrongly. 
It is however possible to convert between integer and floating point types internally e.g. from int16 to int64. 

The symbol table is used as a reference for which type the variable is, and therefore the type checking happens after the symbol table have been filled and therefore after scope checking is completed. 
Type checking is done in many parts of the code, one or more times for each line is common. 
For every operator it must be checked if its types match and if it results in an assignment if that also matches.
Every function call must match the formal parameters of the function. 

The errors produced by the type checker is: ArgumentsError and TypeMismatchError.
An ArgumentsError indicates that the number of arguments does not match the ones given.
A TypeMismatchError is any error which is caused by a value or identifier not being compatible with the function parameters, operator used etc.
Examples are shown in \myref{lst:typeErrors}.

\begin{lstlisting}[caption=Examples of type errors in \gls{gamble},numbers=none,frame=tlrb,label={lst:typeErrors}]
/* [...] */
int a = 1 + 2;      /* Valid */
float b = 2.2 + 1;  /* TypeMismatchError */
float c = 2;        /* TypeMismatchError */ 

a = 2.2;            /* TypeMismatchError */
b = foo(1);         /* ArgumentsError (Takes more or fewer arguments) */ 
/* [...] */
\end{lstlisting}

This check is done as a part of decorating the AST and uses a visitor to traverse the AST. \todo{implementering? ...}