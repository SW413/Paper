\subsection*{Design}
The second important part of the contextual analysis phase for the compiler is the type checking, which enforces the type system of \gls{gamble}.
As \gls{gamble} is statically typed it is necessary to check if all references to identifiers and constant values fit into the context they exist in. 
Since implicit conversion between floating point and integer types is not a part of \gls{gamble} an error must be issued everywhere they are used wrongly. 
It is however possible to implicitly convert between integer and floating point types internally e.g. from int16 to int64, as long as the source variable is of larger bit-wise.
This is also the case for complex datatypes, matrices and vectors, e.g. from \texttt{matrix<float>} to \texttt{matrix<float64>}. 

The symbol table is used as a reference for which type the variable is, and therefore the type checking happens after the symbol table has been filled, hence after scope checking is completed. 
Type checking is done in many parts of the code, one or more times for each line is common. 
For every operator it must be checked if its types match and if it results in an assignment if that also matches.
Every function call must match the formal parameters of the function. 

The errors produced by the type checker are: Argument errors and type mismatch errors.
An argument error indicates that the number of arguments in the function declaration does not match the number of arguments provided in the function call.
A type mismatch error is caused by a value or identifier not being compatible (type safe) with the function parameters, operator used etc.
Examples are shown in \myref{lst:typeErrors}.

\begin{lstlisting}[caption=Examples of type errors in \gls{gamble},numbers=none,frame=tlrb,label={lst:typeErrors}]
/* [...] */
int a = 1 + 2;      /* Valid */
float b = 2.2 + 1;  /* Type mismatch error */
float c = 2;        /* Type mismatch error */

a = 2.2;            /* Type mismatch error */
b = foo(1);         /* Argument error (Takes more or fewer arguments) */ 
/* [...] */
\end{lstlisting}

\subsection*{Implementation}
To implement type checking \gls{gamble} the a visitor is used during the contextual analysis.
As described in the previous section, the checker checks for type validity in both both assignments and function calls.
the class \textt{ASTTypeCheckVisitor} visit every node relevant node in the \acrshort{ast}. 
The functionality of the class consist of checks whether the actual arguments is of the type expected by the compiler, this happens i a couple of different ways described in the following text.