\section{Control Flow}\label{subsec:control-flow}
Control flow is any part of the language in which a branch occurs. 
A branch is in this case a place in the code where multiple codepaths are possible. 
These come in three variations: primitives, choice and loops.
C, C\#, Java etc. contain some or all of these elements to control the flow of a program:

\textbf{Primitive:}
\begin{itemize}[noitemsep,topsep=-5pt] %removes whitespace before and between points.
    \item \texttt{goto}
\end{itemize}

\textbf{Choice:}
\begin{itemize}[noitemsep,topsep=-5pt] %removes whitespace before and between points.
    \item \texttt{if/if-else}
    \item \texttt{?:} (Ternary operator)
    \item \texttt{switch}
\end{itemize}

\textbf{Loops:}
\begin{itemize}[noitemsep,topsep=-5pt] %removes whitespace before and between points.
    \item \texttt{while}
    \item \texttt{for}
    \item \texttt{foreach}
    \item \texttt{do..while}
\end{itemize}

From the elements if, goto and label the other constructs above can be made.
For example a \texttt{while} loop can be translated to a combination of the if and goto statements, this is shown in \myref{ifgotowhile1} and \myref{ifgotowhile2} written in C. 

\noindent\begin{minipage}{.45\textwidth}
\begin{lstlisting}[caption=Loop made with while.,frame=tlrb, label=ifgotowhile1, numbers=none]{Name}
while ( condition )
{
    statement();
}
\end{lstlisting}
\end{minipage}\hfill
\begin{minipage}{.45\textwidth}
\begin{lstlisting}[caption=The same loop with if and goto.,frame=tlrb, label=ifgotowhile2, numbers=none]{Name}
label startOfLoop:
if ( condition )
{
    statement();
    goto startOfLoop;
}
\end{lstlisting}
\end{minipage}

The same can be done with the \texttt{for} and \texttt{do..while} loops. 
This means that the only needed control structures should be \texttt{if} and \texttt{goto}. 
However the use of goto is considered harmful in many cases, because it is easy to make mistakes when using it. \citep{DijkstraGoto}
Even though \texttt{for} and \texttt{while} are similar, they are often used in different contexts.
Therefor \texttt{loop} is useful for iterating over a collection or knowing exactly how many time the code is executed. 
Therefore we have decided to include both \texttt{for} and \texttt{while} in \gls{gamble}, but not \texttt{foreach} and \texttt{do..while}, this is to simplify the number of control structures.

\subsection{Choice structure}
We have decided to implement the traditional \texttt{if} and \texttt{if-else} constructs. 
These are useful, and can achieve the same as a \texttt{switch}, by chaining if-else statements, therefore we have excluded the switch construct to simplify the implement. 
The ternary operator, also known as the inline if, written as show in \myref{terlst} in C and many other languages, is useful for compacting code to make it more readable, however this is not implemtented in \gls{gamble}, because the functionality is redundant with the if-else structure and it is not important for the goal of this project.

Examples of if/if-else \gls{gamble} are shown in \myref{iflst} and \myref{ifelselst}. 
The condition is enclosed in a parenthesis this is to make the if construct similar to the function call, in which the parameters are in a parenthesis. 
All the conditional code must be enclosed in a curly bracket, this is to make the code more readable. 
The use of curly brackets is to make them similar to functions and the explicit declaring the scope if the choice structure.

\noindent\begin{minipage}{.45\textwidth}
\begin{lstlisting}[caption=An if construct in \gls{gamble}.,frame=tlrb, label=iflst, numbers=none]{Name}
if ( condition )
{
    statement();
}

\end{minipage}\hfill
\begin{minipage}{.45\textwidth}
\begin{lstlisting}[caption=An if-else construct in \gls{gamble}.,frame=tlrb, label=ifelselst, numbers=none]{Name}
if ( condition )
{
    statement();
} else {
    alternative_statement();
}
\end{lstlisting}
\end{minipage}


\subsection{while loop}
An example of the while loop in \gls{gamble} is shown in \myref{whilelst}. 
The syntax is identical to the if statement, which also makes it look like a function in certain ways and explicit express the scope.

\vspace{10pt}
\noindent\begin{minipage}{.40\textwidth}
\begin{lstlisting}[caption=A while loop in \gls{gamble}.,frame=tlrb, label=whilelst, numbers=none]{Name}
while ( condition )
{
    statement();
}
\end{lstlisting}
\end{minipage}\hfill
\begin{minipage}{.50\textwidth}
\begin{lstlisting}[caption=A for loop in \gls{gamble}.,frame=tlrb, label=forlst, numbers=none, language=C]{Name}
for ( initialize; condition; update )
{
    statement();
}
\end{lstlisting}
\end{minipage}

\subsection{for loop}
An example of the \texttt{for} loop in \gls{gamble} is shown in \myref{forlst}.
An alternative to this syntax is one which allows an iteration over a collection such as a foreach or one that iterates over a range.
Examples of how such a syntax should be is shown in \myref{forextlst}.

\begin{lstlisting}[caption=Alternative syntaxes for the for loop.,frame=tlrb, label=forextlst, numbers=none, language=C]{Name}
// iterates over every int in the range.
for ( int i in range 0:10 )
{
    statement(i);
}
// iterates over every other int in the range.
for ( int i in range 0:10:2 )
{
    statement(i);
}
// iterates over each element as a reference to the element in the collection
foreach ( item i in collection )
{
    statement(i);
}
// iterates over each index in a collection
foreach ( index i in collection )
{
    statement(collection[i]);
}
\end{lstlisting}

These each have the strength of expressing clearly what they do, however they are easily expressed in the form expressed in \myref{forlst}.
