\subsection{Scanner}
The primary function of a scanner is to transform a sequence of chracters into a sequence of tokens.
A scanner is sometime called a lexer og lexical anayzer.
The scanner is generally combined with a parser, which together analyze the syntax of the language.
With the scanner we can specify how our language should be witting. 
An example of this, would be whatever you could use the notation .1 or 0.1 for a decimal number.
The scanner syntax is usually regular language, whose alphabet consists of the individual characters of the source code.
A scanners job is to convert a stream of characters into a stream of tokens.
A token is a character string that is used when writing source code in the programming language.

%We have been using ANTLR to generate our token stream from the context-free grammar.
%Using ANTLR is saving us time, but writing a scanner in the hand is possible.
%Writing a scanner from scratch means reimplementing components that are common to all scanners.
%This leads to a heavy reproduction of effort.
Some example of a token stream from GAMBLE can be seen on \myref{lst:token}.
Looking at the integer it is seen  the int must start with a number 1-9 and then have all the 0-9 that is needed.
When the formal structure of tokens and program structure is given, it is possible to go through the language and look for design flaws.
This allows for flaws to be discovered before the design is complete, and therefore provide a better flow in the creation of a compiler or interpretor. \citep{Crafting}

\begin{lstlisting}[caption=Example from out Scanner,label={lst:token},numbers=none]
\\Integers
INT: 'int' | 'int16' | 'int32' | 'int64' ; // Integers
INTNUM: '0' | SIGN? [1-9][0-9]* ;

\\Matrices and vectors
MATRIX: 'matrix' ;
ROWVECTOR: 'rowvector' | 'rvec' ;
COLVECTOR: 'colvector' | 'cvec' ;  

\\Whitespace and comments
WS: [ \t ]+ -> skip;
NL: [ \r \n | \n ] -> skip;

COMMENT
    :   '/*' .*? '*/' -> skip
    ;

LINE_COMMENT
    :   '//' ~[\r\n]* -> skip
    ;
\end{lstlisting}