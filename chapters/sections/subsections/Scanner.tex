\subsection{Scanner}
The primary function of a scanner is to transform a sequence of characters into a sequence of tokens.
A scanner is sometime called a lexer og lexical analyser.
The scanner is generally combined with a parser, which together analyse the syntax of the language.
With the scanner we can specify how our language should be witting. 
An example of this, would be whatever you could use the notation .1 or 0.1 for a decimal number.
The scanner syntax is usually regular language, whose alphabet consists of the individual characters of the source code.
A scanners job is to convert a stream of characters into a stream of tokens.
A token is a character string that is used when writing source code in the programming language.

%We have been using ANTLR to generate our token stream from the context-free grammar.
%Using ANTLR is saving us time, but writing a scanner in the hand is possible.
%Writing a scanner from scratch means reimplementing components that are common to all scanners.
%This leads to a heavy reproduction of effort.
Some examples of our lexer rules written for ANTLR4 from GAMBLE can be seen on \myref{lst:token}.
The definition of a integer number on line 3 states that an integer is either a zero or a negative sign followed by a single digit from one to nine followed by zero or more numbers from zero to nine.
It is necessary to clearly define tokens for the lexer to read in order to read source code correctly. \citep{Crafting_book}

\begin{lstlisting}[caption=Example of our lexer rules for ANTLR4,frame=tlrb,label={lst:token}]
// Integers
INT: 'int' | 'int16' | 'int32' | 'int64' ; // Integers
INTNUM: '0' | SIGN? [1-9][0-9]* ;

// Matrices and vectors
MATRIX: 'matrix' ;
ROWVECTOR: 'rowvector' | 'rvec' ;
COLVECTOR: 'colvector' | 'cvec' ;  

// Whitespace and comments
WS: [ \t ]+ -> skip;
NL: [ \r \n | \n ] -> skip;

COMMENT
    :   '/*' .*? '*/' -> skip
    ;

LINE_COMMENT
    :   '//' ~[\r\n]* -> skip
    ;
\end{lstlisting}