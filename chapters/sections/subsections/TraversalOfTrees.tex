\subsubsection*{Traversal of Trees}
When working with trees, traversing those trees is an important part of the process.
For this task different approaches can be taken, a common way is to implement a design pattern, the visitor pattern is particularly popular for tree traversal.
Alternatively one can implement the composite pattern or choose to implement no pattern at all, but simply create a case analysis for each object.
The use of a design pattern is not a requirement for the creation of a compiler.

Design patterns provide solution templates for software problems, each pattern providing its own benefits.
In OOP design patterns are typically aimed at helping object generation and interaction between these objects.
However the most important thing to keep in mind when using a design pattern, is not its exact implementation of classes and methods, but the concept the pattern describes.

The two aforementioned patterns are classified under two different branches of patterns.
The composite pattern is a structural pattern where the visitor pattern is a behavioural one.
A structural pattern provides a way of defining the relations between objects, the composite pattern is used to create a hierarchical recursive tree structure of related objects that may be accessed in a standardised manner.
A behavioural pattern is instead used to define how the objects communicate, the visitor pattern is used to separate a set of structured classes from any functionality that should be performed upon them.
For the compiler the visitor pattern have been implemented for the traversal of trees as such the pattern is described further in \myref{subs:visit}.  
