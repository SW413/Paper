\section{Reserved Words}

% (fold)
\label{sec:reserved_words}
In \gls{gamble} there exists a number of reserved words used for types and language constructs among others.
These reserved words, also known as language keywords, cannot be used as identifiers in the language. However, keywords are not always reserved words.
This is because keywords such as \texttt{if} and \texttt{for} often are followed by a specific construct.
In fact most cases of keyword usage require some sort of special construct, hereby rendering the keyword unambiguous.
As en example, this makes it possible to use \texttt{if}, along with all other keywords, as identifiers in FORTRAN, but this is not the case in \gls{gamble}.\citep{fortran_identifiers}
Keywords in \gls{gamble} are considered reserved, because it minimises the lookahead needed, and therefore simplifying the lexing process.
All the reserved words of \gls{gamble} can be seen in \myref{res:words} .
\begin{table}[h!]
	\centering
	\def\arraystretch{1.5} \setlength{\tabcolsep}{2em}
	\begin{tabular}{l l l l l}
        \texttt{int}     & \texttt{int16}     & \texttt{int32}     & \texttt{int64}     & \texttt{bool}    \\
        \texttt{float}   & \texttt{float16}   & \texttt{float32}   & \texttt{float64}   & \texttt{void}    \\
        \texttt{matrix}  & \texttt{vector}    & \texttt{true}      & \texttt{false}     &                  \\
        \texttt{if}      & \texttt{else}      & \texttt{for}       & \texttt{while}     &                  \\
        \texttt{print}   & \texttt{rows}      & \texttt{cols}      & \texttt{return}    & \texttt{include} \\
    \end{tabular}\caption{Reserved words of \gls{gamble}}\label{res:words}
	\def\arraystretch{1}
\end{table}

Besides the reserved words \gls{gamble} contains literals, which also cannot be used as identifiers.
The literals in \gls{gamble} are \texttt{true} and \texttt{false}.
If \gls{gamble} were to provide a \texttt{null} value, this would be a literal as well.

% section reserved_words (end)
