\subsubsection*{The Code Generator Visitor}
When the code generator is called, the \acrshort{ast} is used to accept a \texttt{CodeGeneratorVisitor} as seen on \myref{fig:CodeGeneratorVisitor}.

\begin{figure}[!ht]
\centering
 \includegraphics[width=0.8\textwidth]{figures/ClassDiagrams/CodeGeneratorCall.pdf}%trim=4cm 0cm 0cm 0cm, clip
\caption{A class diagram of \texttt{CodeGeneratorVisitor} showing the call from the code generator to the visitor which makes the string from the decorated \acrshort{ast}.}\label{fig:CodeGeneratorVisitor}
\vspace{-15pt}
\end{figure}

The visitor builds a string\todo{of object code -- Troels} by traversing the nodes of the tree which will in the end be put into a file by the \texttt{CodeGenerator} class.\todo{Hvad menes der med at det blive puttet i en fil? Det står måske lidt uklart. - Corlin}
Some visit methods like a visit method for a \texttt{ConstantExpressionNode} returns a string while other methods instead directly appends to the string the visitor will return to the class \texttt{CodeGenerator}.
This string is then used as an argument in other visit methods e.g. \texttt{AssignmentNode} which then produces a string which is a statement that can be executed in C. \todo{C eller OpenCL C ? -- Troels}
This string from \texttt{AssignmentNode} is one of the strings appended to the string containing the fully compiled program.
%%%%%%%%%%%%%%%%%%%%%%%%%%%%%%%%%%%%%%%%%%%%%%%%%%%%%%%%%%%%%%%%%%%%%%%%%%%%%%%%%%%%%%%%%%%%%%%%%%

As a result the information bubbles upwards from the leafs of the tree to the statement-nodes.
The information is put in the correct statement-nodes because the visitor pattern makes it possible to specify the route of traversal.
The \texttt{CodeGeneratorVisitor} also contains other methods for producing certain C constructions using the similar \gls{gamble} constructions found in the \acrshort{ast}.
These methods can also be seen on \myref{fig:CodeGeneratorVisitor}.
Some of these methods also make runtime checks of matrices, as matrices needs to be of compatible sizes for some of the operations which can be performed on a matrix, furthermore an index check is implemented such that an out of bounds error will occur if one tries to access memory beyond the bounds of the matrix.
When multiplying two matrices the left matrix of the multiplication has to be a $ N \times M $ while the right one has to be a $ M \times P $ matrix.
So the right matrix must have the same number of columns as the right matrix has rows.
If performing a matrix index multiplication, where every index is multiplied with the corresponding index of the the other matrix, the matrices have to have the same number of rows and columns. \todo{De 3 linjer her virker lidt random. Det hele er bare en stor uorganiseret brødtekst. -- Troels}

These checks will be inserted as a surrounding if statement around the matrix calculations, so they have to pass these checks before the calculation will be done, if it does not pass an error will be printed.\todo{Der kommer da en fejl right ? - Søren}
The reason this is done at run-time instead of at compile-time is because the sizes of matrices can be dynamic, which results in making it impossible to check for this\todo{Fjern ``for this`` ? - Corlin} at compile-time.

When the string is complete it is written to a file called code.c, along with all the other files needed for running the code, e.g. the kernels used for performing computations on the \acrshort{gpu}.
The code.c file is structured so that it is a valid C program.
First off are the libraries included that C uses, then follows a list of prototypes which are all the \gls{gamble} functions translated into C code, both the user made and the ones from libraries.
After the prototypes the main function is made where the body consists of all the statements in the \gls{gamble} sourcecode translated into C.
After the main method all the implementations of the function prototypes are made.
