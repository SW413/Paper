\subsection*{Optimisation/Runtime Efficiency}
As code is generated considerations of efficiency can be made such that the object code becomes more efficient than other alternatives.
Such optimisations consists of considerations like memory allocation, pipelining, parallelisation and other considerations that may have an impact on execution time.
To do an efficient and complete analysis of how to optimise code, information is required.
This is where \gls{gamble} is at a disadvantage due to \gls{gamble} promoting its seamless use of the \acrshort{gpu}.
As this is seamless information is lackluster all computations may will not be fully optimised, this is one of the tradeoffs \gls{gamble} makes.
If \gls{gamble} were to require more information, the simplicity and seamless use of the \acrshort{gpu} would be lost.

As previously mentioned due the object code being OpenCL C certain considerations pertaining to instruction handling are not a point for optimisation for this compiler.
Instead the optimisation here resides in when to use the \acrshort{gpu}, optimising the generated C code as well as the OpenCL kernals.

%Knowing when to use GPU

%Optimising C code(CPU)

%Optimising OpenCL Kernals

