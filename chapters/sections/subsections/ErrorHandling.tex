This section describes how the compiler handles errors, and how error handling is implemented.
Error handling is referred to as detection and reporting of programming errors. 
Both errors at compile time and at run time will be handled.
\subsection*{Design}\label{subsec:DesignErrorHandling}
The source code given to the \gls{gamble} compiler may contain errors, these errors are found and uncovered in various stages of the compiler.   
In the syntax analysis, all syntax errors are found, and the compilation is stopped if errors are found as it would not be advised to further compile source code which contains fundamental syntax errors.
However it should be noted that the entire syntax analysis is completed before stopping the compiler, hence all syntax errors are collected and reported to the user.
This makes the process of resolving syntax errors less tedious for the programmer, than if the compiler would stop every time a single error was detected. 
A core element of error handling is scope and type related errors, these are found in the previously explained subphases of the contextual analysis: Scope- and typechecking.
As with the syntax analysis, the compiler collects all errors before stopping the compilation of the source code. 
The following errors and warnings are reported by the \gls{gamble} compiler:
\begin{description}
	\item[Argument error]\hfill\\ 
	The arguments required in a function call doesn't match the ones used.
	\item[Redeclaration error]\hfill\\ 
	A variable or function with the same name is already declared.
	\item[Type mismatch error]\hfill\\ 
	The type or types used are incompatible with each other and/or the operator. 
	\item[Undeclared error]\hfill\\ 
	Attempted use of an undefined variable or function.
	\item[Unused variable warning]\hfill\\ 
	A warning signaling that a variable or function is declared but never used.
\end{description} 
The errors in the compiler should give useful information about where the error is located in the source code and what variable(s) or function(s) were used incorrectly.
This includes line numbers which means that the creation of the \acrshort{ast} have to include the line numbers from the source code.

\subsection*{Implementation}\label{subsec:ImplementationErrorHandling}
In the \gls{gamble} compiler the error handling is implemented by having a class \texttt{LanguageError} which each specific error type inherits from.
The \texttt{LanguageError} superclass has information about the type of error, an enumeration which is either \texttt{Error} or \texttt{Warning}. 
This is because a program which has warnings but no errors, should still compile, however any program with one or more errors should not. 
Compared to printing an error once discovered and stopping compilation, this model allows the compiler to go through the entire \acrshort{ast} and discover every error as described above. 
It also unifies how errors are printed and shown to the user.
There is also an integer indicating which line the error is contained on.
This enhances the error reporting process, making it easier for users to debug. 
As an example the class for \texttt{UnDeclaredError} is shown in \myref{lst:undeclarederrorclass}.
Firstly there are variables which contain information needed to report the error to the programmer. 
Then there is a constructor which simply assigns all values given to it to the new object's fields.
Lastly there is the \texttt{toString} method which overriddens the implementation in the \texttt{LanguageError} superclass. 
An example of an \texttt{UnDeclaredError} when printed is: \texttt{Error[line   42]-> Undeclared variable towel in scope Local}, if the variable's ID was \texttt{towel}. %Sneaky H2G2 reference 

\begin{lstlisting}[caption=The UnDeclaredError class in the \gls{gamble} compiler,numbers=none,frame=tlrb,label={lst:undeclarederrorclass}]
public class UnDeclaredError extends LanguageError {
    private Variable unDeclaredVariable;
    private Scope scope;

    public UnDeclaredError(Variable unDeclaredVariable, Scope scope, int lineNum) {
        this.unDeclaredVariable = unDeclaredVariable;
        this.scope = scope;
        this.lineNum = lineNum;
        this.errorType = ErrorType.ERROR;
    }

    @Override
    public String toString() {
        String type = unDeclaredVariable.isFunction() ? "function " : "variable ";
        return super.toString() + String.format("Undeclared %6$s %1$s%4$s%3$s in scope %2$s%5$s%3$s",
                ANSI_RED, ANSI_BLUE, ANSI_RESET,
                unDeclaredVariable, scope, type);
    }
}
\end{lstlisting}

An undeclared variable error is discovered by the \texttt{CheckIfUndeclared} method in the \texttt{SymbolTableFillVisitor} shown in \myref{lst:CheckIfUndeclared}.
The method is called whenever a variable or function is invoked.
If the variable or function invoked is not declared, an error is added to a global list of errors. 
\begin{lstlisting}[caption=The CheckIfUndeclared method in the SymbolTableFillVisitor class in the \gls{gamble} compiler,numbers=none,frame=tlrb,label={lst:CheckIfUndeclared}]
public Void CheckIfUndeclared(Variable variable, StatementNode node) {
    Symbol tmpSymbol = symbolTable.currentScope().resolve(variable.getId());
    if (tmpSymbol != null) {
        /* Sets appropriate information to the 'variable' variable */
    } else {
        errors.add(new UnDeclaredError(variable, symbolTable.currentScope(), node.getLineNumber()));
        symbolTable.currentScope().define(variable);
    }

    return null;
}
\end{lstlisting}
