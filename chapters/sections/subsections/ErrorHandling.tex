\subsection*{Design}\label{subsec:DesignErrorHandling}
The source code given to the \gls{gamble} compiler can contain errors and warnings, these are found in the various stages of the compiler.
Most central are scope and type related errors, these are found in the contextual analysis phase of the compiler. 
The following errors and warnings are reported by the \gls{gamble} compiler:
\info[inline]{Itemize yay or nay?}
\begin{itemize}
	\item Arguments error: The arguments of a function call doesn't match the ones used.
	\item Redeclaration error: A variable or function with the same name is already declared.
	\item Type mismatch error: The types used are incompatible with the operator. 
	\item Undeclared error: Attempted use of an undefined variable or function.
	\item Unused variable warning: A warning signaling that a variable or function is declared but never invoked.
\end{itemize} 
The errors in the compiler should give useful information about: Where the error is in the source code and what variable(s) or function(s) was wrongly used.
This include line numbers which then have to be carried over from the parse tree to the abstract syntax tree, this is expanded in \todo{ref til afsnit som ikke findes i denne branch}.

\subsection*{Implementation}\label{subsec:ImplementationErrorHandling}
In the \gls{gamble} compiler this is implemented by having a class \texttt{LanguageError} of which each specific error type inherits from.
The \texttt{LanguageError} superclass has information about what type the error, an enumeration which is either \texttt{Error} or \texttt{Warning}. 
This is so because a program which has warning but no errors, should still compile, however any program with errors should not. 
Compared to simply printing any error when it was discovered and stopping compilation, this model allows the compiler to go through the entire \acrshort{ast} and discover every error. 
It also unifies how errors are printed and shown to the user. \todo{Denne og forrige sætning er måske lidt svage? Skal der være mere ellere mindre om det. På plus siden er det jo en vurdering og det som viser forståelse osv.}
There is also an integer indicating which line number the error is contained on.
This is to enhance the error reporting process, making it easier for users to debug. 
\todo{Måske skriv noget om at dette kræver informationen er med i ASTet fra PTet ?}
As an example the class for \texttt{UnDeclaredError} is shown in \myref{lst:undeclarederrorclass}.
First there are variable which contains information needed to report the error to users. 
Then there is a constructor which simply assigns all values given to it to the new objects fields.
Lastly there is the \texttt{toString} method which is overridden from the implementation in the \texttt{LanguageError} class. 
An example of an \texttt{UnDeclaredError} is: \texttt{Error[line   42]-> Undeclared variable towel in scope Local}. %Sneaky H2G2 reference 

\begin{lstlisting}[caption=The UnDeclaredError class in the \gls{gamble} compiler,numbers=none,frame=tlrb,label={lst:undeclarederrorclass}]
public class UnDeclaredError extends LanguageError {
    private Variable unDeclaredVariable;
    private Scope scope;

    public UnDeclaredError(Variable unDeclaredVariable, Scope scope, int lineNum) {
        this.unDeclaredVariable = unDeclaredVariable;
        this.scope = scope;
        this.lineNum = lineNum;
        this.errorType = ErrorType.ERROR;
    }

    @Override
    public String toString() {
        String type = unDeclaredVariable.isFunction() ? "function " : "variable ";
        return super.toString() + String.format("Undeclared %6$s %1$s%4$s%3$s in scope %2$s%5$s%3$s",
                ANSI_RED, ANSI_BLUE, ANSI_RESET,
                unDeclaredVariable, scope, type);
    }
}
\end{lstlisting}

This error is found in the \texttt{CheckIfUndeclared} method in the \texttt{SymbolTableFillVisitor} shown in \myref{lst:CheckIfUndeclared}.
This method is called whenever a variable or function is invoked, and returns the appropriate information. \todo{Teknisk set returnerer den ikke informationen men sætter det gennem en reference, skal dette skrives eksplicit? Kunne ikke lige formulere det ordenligt.}
If the variable or function invoked is not declared then an error is added to a global list of errors. 
\begin{lstlisting}[caption=The CheckIfUndeclared method in the SymbolTableFillVisitor class in the \gls{gamble} compiler,numbers=none,frame=tlrb,label={lst:CheckIfUndeclared}]
public Void CheckIfUndeclared(Variable variable, StatementNode node) {
    Symbol tmpSymbol = symbolTable.currentScope().resolve(variable.getId());
    if (tmpSymbol != null) {
        /* Sets appropriate information to the 'variable' variable */
    } else {
        errors.add(new UnDeclaredError(variable, symbolTable.currentScope(), node.getLineNumber()));
        symbolTable.currentScope().define(variable);
    }

    return null;
}
\end{lstlisting}
