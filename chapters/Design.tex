\chapter{Design}
\label{cha:Design}
%Metatext for design chapter
In this chapter the design of the language will be presented.
First the philosophy for the language will be introduced to clarify what this language is attempting to achieve.
The philosophy will lead to some values being more important than others, the choices we deem impactful will be documented in this chapter.

\input{chapters/sections/philosophy.tex}

In the following sections different parts of \gls{gamble} will be investigated.

\section{Core syntax choices}
The source code must be written in Unicode, but only the ASCII alphabet and numbers are allowed for names, this is to simplify the parsing. 
The language is case sensitive, this is done because the most used languages seen on TIOBEs index\citep{TIOBE} also are case sensitive
When a language is case sensitive it is possible for the programmer to name different parts of the code like he pleases, resulting in a higher readability.
Some languages, such as Python, uses whitespace and indentation to indicate scope, in our language scope is indicated by curly brackets like many exciting languages\texttt{\{\}}.  

\subsection*{Comments}
It is useful to be able to write comments in code for annotate meaning.
If a language were not to have the possibility to write comments, then the source code could be hard to work co-operative on.
A source code whiteout comments would be difficulty to return to after some time away from it, since the programmer would have to reread the entire code, instead of a few good comments provided by himself.\citep{Commenting}
Therefore both single line and multi line comments can be used in \gls{gamble}. 
``//'' is used for single line comments, meaning that everything after the ``//'' until the next newline is ignored by the compiler, also known as C++ style comments. 
``/* */'' is used for multi line comments, meaning that everything between ``/*'' and ``*/'' is ignored by the compiler, also known as C style comments. 

\subsection*{End of statement terminator}
When choosing an end of state terminator there is two main choices.
Either one can use the newline character as a terminator or one can use another symbol, often semicolon (;).
The use of newline is often used in simple languages i.e. MATLAB or other mathematical oriented languages. 
We deem the main value of using the newline are simplicity and a decrease in errors due to forgotten semicolons.
Semicolon is as mentioned the other main choice, this choice has been used through many programming languages throughout the history. 
The main advantage of a end of statement terminator  like semicolon is that it allows freeform code. 
This freeform allows to have multiple statements in one line or allows long statements to span across multiple lines.
We deem this to enhance the readability of the code.
Furthermore the semicolon terminator allows for a simpler syntax and compiler design.
As mentioned before forgotten semicolons can result in compiler errors, we deem this as the biggest disadvantages of this end of statement terminator.

Based on this analysis we have chosen to use the semicolon as end of statement terminator.

\subsection*{Scope}
If a programming language contain no implementation of scoping rules, the one must assume that all the source code in a given program would be in the same scope.
Such a single scope would mean that any variable can be used form any place in the code. 
Mostly, as briefly described in \myref{sec:state_of_the_art}, a programming language implements some kind of scoping rules.
There are several reasons for this, the main reason being it gives the programmer more control her code and reduced the change of chancing variables by fault.
There exist two main methods of scoping, static and dynamic scoping.
As seen in \myref{sec:state_of_the_art} most of the languages examined in this section used a form a static scope. 
\gls{gamble} will inherit this form of scoping because we find it to be the easiest to work with and less prone to errors than dynamic scoping.
This exact details of the scoping rules are described in the following  lines.

A variable is in scope from its deceleration until the end of the block it is declared in.
An inner block such as a control flow structure, see \myref{subsec:control-flow} for more, inherits the variables from the outer scope. 
It is also not allowed to redeclare a variable while it is in scope. 
This is to increase the readability of the programs written in \gls{gamble}.
Additionally any function is in scope of any other function including itself, this also allows recursion. 

This is static scoping without information hiding. 

\subsection*{Structure} 
\todo{compare with existing languages and present alternatives, pro's and con's}
The structure of the source code to a \gls{gamble} program is shown in \myref{lst:Structure}.
For a high level of readability we have decided  to split the source code up in segments.
A single segment contains only a small part of source code.
Inclusions are gathered in the top segment, this is to make it easy to get an overview of any and all included libraries. 
Then all function declarations are written in the next segment.
Then all the main statements are written in the last segment. 
This is also easier to parse and write context free grammar for.

An alternative to the strictly split segments, is i.e C.
C can use functions before they are declared, but it can't only be done with prototypes in the top of the source code.
We find the feature very good, but we limit our self to the more simple structure hence it is a lot easier to implement.

\begin{lstlisting}[caption={Source code file layout in \gls{gamble}},frame=tlrb,label={lst:Structure}, numbers=none]
Libary inclusions

Function declarations

Statements
\end{lstlisting}

\section{Types and Variables} \label{sec:Types}
%http://www.informit.com/articles/article.aspx?p=2103809&seqNum=3
\gls{gamble} uses three primitive data types; integers (int), floating points (float) and boolean values (bool), as well as two composite data types; matrices and vector. 
In \gls{gamble} any variable must be initialised when it is declared. 
This avoids any null reference which can cause runtime errors in many languages such as C, C\# and Java. 

\subsection*{Integers and floating points}
An integer is a whole number either positive or negative, represented by the two's compliment method. 
Meaning that it ranges from $-(2^{n-1}) $ to $2^{n-1} - 1 $, where n is the number of bits used.
A floating point number is an approximation to a real number. 
It is separated in 3 parts, the sign bit, the mantissa and the exponent. 
Floats in \gls{gamble} are in base 2, the value of a float is calculated by $ (-1)^{sign bit} \cdot mantissa \cdot 2^{exponent} $. 
Integers and float point numbers default to 32 bits with the keywords \texttt{int} and \texttt{float}. 
It is possible to declare 16 and 64 bit integers and floating point numbers by postfixing their size to their respective keywords, i.e. \texttt{float64} or \texttt{int16}. 
It is useful to be able to specify the size of an integer or a floating point number, as data transfer can cause a lot of overhead particularly when transferring data to and from the \acrshort{gpu}. 

\subsection*{Boolean}
A boolean value is either true or false. 
These are used for control structures in \gls{gamble} which takes a boolean value as a parameter. 
But can also be used for any means which the programmer finds them useful. 

\subsection*{Vectors and Matrices}
A vector in \gls{gamble} is a sequence of entries which are each the same primitive data type. 
The dimension of a vector is immutable, meaning that its dimension cannot be changed after its deceleration. 
A matrix is multiple row vectors inside a column vector, each row vector is the same dimension. 
As for other variables both vectors and matrices must be initialised when they are declared. 
This is done with the syntax shown in \myref{lst:matrix}.

\begin{lstlisting}[caption={Creating a matrix},label={lst:matrix},numbers=none]
matrix<int> m = [1, 0; 0, 1];

rowvector<int> rv = [1, 0];
\end{lstlisting}

\subsection*{Operators}
Available operators in the language can be found on \myref{tbl:operators} with a short description.  
It is not possible to use operators with matrices or vectors, functions which support different operations on these types will be included in a library.
\begin{table}[h]
    \centering
    \colorlet{shadecolor}{gray!40}
    \rowcolors{1}{white}{shadecolor}
    \begin{tabular}{|c|l|l|l|l|l|}
    \hline
    \textbf{Operator}  & \textbf{Integer}                   & \textbf{Floating point number}    & \textbf{Boolean}      & \textbf{Matrix}       & \textbf{Vector}     \\ \hline
    +                  & Addition (infix)                   & Addition (infix)                  & Type error            & Addition (infix)      & Addition (infix) \\ \hline 
    -                  & Subtraction (infix)                & Subtraction (infix)               & Type error            & Subtraction (infix)   & Subtraction (infix) \\ \hline 
    ++                 & Increment (unary,pre-, postfix)    & Increment(unary,pre-, postfix)    & Type error            & Type error            & Type error \\ \hline    
    --                 & Decrement (unary,pre-, postfix)    & Decrement(unary,pre-, postfix)    & Type error            & Type error            & Type error \\ \hline
    *                  & Multiplication (infix)             & Multiplication(infix)             & Type error            & Type error            & Type error      \\ \hline
    \%                 & Remainder (infix)                  & Modulo(infix)                     & Type error            & Type error            & Type error  \\ \hline
    /                  & Division (infix)                   & Division (infix)                  & Type error            & Type error            & Type error \\ \hline
    \&\&               & Type error                         & Type error                        & Logical and (infix)   & Type error            & Type error \\ \hline 
    ||                 & Type error                         & Type error                        & Logical or (infix)    & Type error            & Type error \\ \hline 
    ==                 & Equality (infix)                   & Equality* (infix)                 & Type error            & Type error            & Type error \\ \hline 
    !=                 & Inequality (infix)                 & Inequality* (infix)               & Type error            & Type error            & Type error \\ \hline
    <=                 & Less than or equal to (infix)      & Less than or equal to* (infix)    & Type error            & Type error            & Type error \\ \hline
    >=                 & More than or equal to (infix)      & More than or equal to* (infix)    & Type error            & Type error            & Type error \\ \hline
    <                  & Less than (infix)                  & Less than (infix)                 & Type error            & Type error            & Type error \\ \hline
    >                  & More than (infix)                  & More than (infix)                 & Type error            & Type error            & Type error\\ \hline
    \end{tabular}
    \caption[List of operators on primitive data types in GAMBLE.]{List of operators on primitive data types in GAMBLE.\@*Equality comparisons on floating point number may be incorrect.}\label{tbl:operators}
\end{table}
\vspace{-20pt}

\section{Functions}
Functions are blocks of code which can be called to do a specific computation, often but not always, such a function takes some variables as inputs.

This section will describe \gls{gamble}'s use of functions. 
Like in many other languages it is possible to declare your own functions in \gls{gamble}.
This is useful for organising code, and reusing parts of the source code.
Functions in \gls{gamble} can take inputs, but are not required to, the inputs can be any of the types the languages supports.
Because of high implementation cost in the compiler, it is not possibly to the use a function as an input in function.


\subsection*{Function identifiers}
When declaring a \gls{gamble} function it is required to write the body of the function immediately after.
This is to avoid lookahead which function prototypes requires.
A function identifier is exactly like C, with the \texttt{return type} then the \texttt{functionname}, \texttt{(formal parameters)} and finally \texttt{\{body\}}.
The function body can contain a number of statements and various control flows.
Furthermore it can contain calls to other functions or calls to the function itself, and hereby recursion is a possibility in \gls{gamble}
An example of a function identifier can be seen on \myref{lst:functionID}.
Other syntaxes could have been used, but not only is this the way C does it, it is also widely used in many other languages i.e. C\# and Java.

\begin{lstlisting}[caption={Function Identifier},label={lst:functionID},numbers=none]                                                        
int add(int a, int b)
{
	return a + b;
}
\end{lstlisting}

\subsection*{Function calls}
After a function has been declared it can be invoked also known as called. 
A function call contains the identifier and parameters. 
The syntax in \gls{gamble} is the same as in many other programming language and shown in \myref{lst:functionCall}. 

\begin{lstlisting}[caption={A function call in \gls{gamble}},label={lst:functionCall},numbers=none]
add(4, 3);
\end{lstlisting}


\subsection*{Return value}
Functions have a return value which can be seen on \myref{lst:functionID}.
The return values are all the types in \gls{gamble} mentioned in \myref{sec:Types} and void. 
A void function will not return anything, but instead can be seen as a procedure.
This choice was made because \gls{gamble} should to be able to return all the different types in the language from a function, but also be able to perform these procedures which a void function may do.
The value or type to be returned is preceded by the keyword \texttt{return}, just like it is in C and other C-like languages.
An example can be seen on \myref{lst:returnFunction}.

\begin{lstlisting}[caption={Return Function},label={lst:returnFunction}]
int a = 0;

a = add(4, 3);
\end{lstlisting}

%STD LIB EVT?!
%\subsection*{Premade functions (rename)}\todo{This} 
%\section{Control Flow}
%\textbf{For-loop}
%\textbf{While-loop}
%\textbf{If \& If else}

%% OLD
%
\section{Control Flow}
Control flow is the parts of the language which will make ``decitions''.
Often these are separated into two primary groups: iteration and condition. 
An iteration is a statement which gives the program the possibility to iterate over the same code a given number of times. 
Common loops in programming languages are: do..while, while, for and foreach. 
The iteration of a loop can be broken in our language, to jump out of the loop, by a statement often called \texttt{break}, and the ability to skip to start of the next iteration by the keyword \texttt{continue}.

The conditionals in many programming languages are: if, ternary operator (?:) and switch.
The combination of a conditional and a jump, often seen as a \texttt{goto} can produce an iteration. 
The goto statement is not in our language, it is considered harmful by many, and in all but a few cases useless, if other conditionals are in the programming language. \citep{DijkstraGoto}
The switch statement is not included in our language. \todo{Skal vi have switch med?}

\subsection{While-loop}
The while-loop is a very common language construct, it iterates once or more times \textit{while} a given boolean condition is true. 
In our language blocks are delimited by curly brackets (``\texttt{\{}'' and ``\texttt{\}}''), and so is the body of the while-loop. 
Unlike languages like C, C\# and Java etc. in our language it is not allowed to omit the curly brackets if one wishes to only to have a single statement in the while-loop, this is a syntax error. 
This is to increase the consistency of the code thus making it more readable. 
In \myref{lst:whileExample} is an example of a while-loop in our language. 
An variation of the while-loop called the do..while-loop, in which a single execution a done before checking the boolean condition is not included in our language. 
This is to reduce the amount of built-in control structures. 

\begin{figure}[h]
\begin{lstlisting}[caption=An example of a while-loop, label=lst:whileExample]
while ( boolean_condition )
{
    example_statement();
}
\end{lstlisting}
\end{figure}
\subsection{For-loop}
A for-loop is a loop like the while-loop, with additional features. 
The traditional for-loop has three fields, the initialization, the condition and the increment/decrement. 
Any of these can be omitted, if the conditional is omitted then the loop is an infinite loop, but it can still be broken by the \texttt{break} statement.
Like for while-loops, curly brackets around the body of for-loops are enforced. 
Any traditional for-loop can be translated into a while-loop, by putting the initialization before the while-loop and the increment/decrement in the end of the body. 
We have the tradition for-loop in our language, an example of it is shown in \myref{lst:whileExample}. 
For-loops are often used to iterate over a collection, and it is easier to see what the for-loop does than a while-loop, as the three fields are in one place. 
This can be further simplified by having a foreach-loop, which iterate over each element in a collection.
However this is not included in our language, in order to reduce the scope of the language and simplify its development. \todo{Skal vi have foreach?}
\begin{figure}[h]
\begin{lstlisting}[caption=An example of a for-loop, label=lst:forExample]
for ( initialization ; boolean_condition ; increment/decrement )
{
    example_statement();
}
\end{lstlisting}
\end{figure}


\subsection{Conditionals}
The if-statement is a very common language construct, it is just like a while-loop except it does not iterate and only executes once if the condition is true.
Like for while-loops, curly brackets around the if-statement are enforced. 
Optionally an else statement can be added to the if-statement, this occurs only if the boolean condition of the previous if was false. 
A grouping of several if and else statements are allowed and is called an if-else-chain. 

The \texttt{?:} operator known as the conditional-operator or ternary if, is also in our language. 
Before the question mark is a boolean condition if it is true, then the statement after the question mark is executed otherwise the one after the colon. 
An example of both the if, if-else and ternary if are shown in \myref{lst:condExample}.
\begin{figure}[h]
\begin{lstlisting}[caption=An example of the conditional operators., label=lst:condExample]
if ( boolean_condition )
{
    example_statement();
} else {
    // executed only if the boolean_condition is false
    other_statement;
}

// ternary example, if the boolean_condition is true,
// value is alternative_one, else it is alternative_two
value = boolean_condition ? alternative_one : alternative_two;

\end{lstlisting}
\end{figure}

%% New
\section{Control Flow}\label{subsec:control-flow}\todo{Troels: Imo er hele dette afsnit på en måde for langt. }
Control flow is any part of the language in which a branch occurs. 
A branch is in this case a change in which instruction happens next. 
These come in three variations: primitives, choice and loops.
C, C\#, Java etc. contain some or all of these elements to control the flow of a program:

\textbf{Primitive:}
\begin{itemize}[noitemsep,topsep=-5pt] %removes whitespace before and between points.
    \item \texttt{goto}
\end{itemize}

\textbf{Choice:}
\begin{itemize}[noitemsep,topsep=-5pt] %removes whitespace before and between points.
    \item \texttt{if/if-else}
    \item \texttt{?:} (Ternary operator)
    \item \texttt{switch}
\end{itemize}

\textbf{Loops:}
\begin{itemize}[noitemsep,topsep=-5pt] %removes whitespace before and between points.
    \item \texttt{while}
    \item \texttt{for}
    \item \texttt{foreach}
    \item \texttt{do..while}
\end{itemize}

If a programming language has if, goto and label then the other constructs can be made.
For example a while-loop can be translated to a combination of the if and goto statements, this is shown in \myref{ifgotowhile1} and \myref{ifgotowhile2} written in C. 

\noindent\begin{minipage}{.45\textwidth}
\begin{lstlisting}[caption=Loop made with while.,frame=tlrb, label=ifgotowhile1, numbers=none]{Name}
while ( condition )
{
    statement();
}
\end{lstlisting}
\end{minipage}\hfill
\begin{minipage}{.45\textwidth}
\begin{lstlisting}[caption=The same loop with if and goto.,frame=tlrb, label=ifgotowhile2, numbers=none]{Name}
label startOfLoop:
if ( condition )
{
    statement();
    goto startOfLoop;
}
\end{lstlisting}
\end{minipage}

The same can be done with the \texttt{for} and \texttt{do..while} loops. 
This means that the only need control structures should be \texttt{if} and \texttt{goto}. 
However the use of goto is considered harmful, because it is easy to make mistakes when using it. \citep{DijkstraGoto}
Even though \texttt{for} and \texttt{while} are similar, they are often used in different contexts.
The for loop is useful for iterating over a collection or knowing exactly how many time the code is executed. 
Therefore we have decided to include both \texttt{for} and \texttt{while} in \gls{gamble}, but not \texttt{foreach} and \texttt{do..while}, this is to simplify the number of control structures.

We have decided to keep the traditional \texttt{if} and \texttt{if-else} constructs. 
These are useful, and can achieve the same as a \texttt{switch}, by chaining if-else statements, so we have excluded the switch construct. 
The ternary operator, also known as the inline if, written as \texttt{?:} in C and many other languages, is useful for compacting code to make it more readable, therefore we include it in \gls{gamble}.

\subsection{if, if-else and ?:} \todo{Jeg blander meget rundt i statement og construct, kan ikke helt beslutte mig. }
Examples of if/if-else \gls{gamble} are shown in \myref{iflst} and \myref{ifelselst}. 
The condition is enclosed in a parenthesis this is to make the if construct similar to the function call, in which the parameters are in a parenthesis. 
All the conditional code must be enclosed in a curly bracket, this is to make the code more readable. 
The use of curly brackets is to make them similar to functions to make the language more alike \todo{mangler et godt fagord her. marc: more familiar/relateable/orthogonal?}.

\noindent\begin{minipage}{.45\textwidth}
\begin{lstlisting}[caption=An if statement in \gls{gamble}.,frame=tlrb, label=iflst, numbers=none]{Name}
if ( condition )
{
    statement();
}
\end{lstlisting}
\begin{lstlisting}[caption=A use of \texttt{?:} in \gls{gamble}.,frame=tlrb, label=terlst, numbers=none]{Name}
condition ? ifTrue() : ifFalse();
\end{lstlisting}

\end{minipage}\hfill
\begin{minipage}{.45\textwidth}
\begin{lstlisting}[caption=An if-else statement in \gls{gamble}.,frame=tlrb, label=ifelselst, numbers=none]{Name}
if ( condition )
{
    statement();
} else {
    alternative_statement();
}
\end{lstlisting}
\end{minipage}

An example of the ternary operator \texttt{?:} is shown in \myref{terlst}. 
If the boolean statement before the question mark is true, then the first statement is executed otherwise the second is. 
The ternary operator can also be used for an assignment, only if the value getting assigned to is left of the question mark.
As the ternary operator allows any statement in its middle body, it is possible to chain them together, just like a if-else chain. 

\subsection{while}
An example of the while loop in \gls{gamble} is shown in \myref{whilelst}. 
The syntax is identical to the if statement, which also makes it look like a function in certain ways. 
It's also identical to the syntax found in many other programming languages and is therefore similar to many programmers. \todo{Skal sådan et argument være med?}

\noindent\begin{minipage}{.40\textwidth}
\begin{lstlisting}[caption=A while loop in \gls{gamble}.,frame=tlrb, label=whilelst, numbers=none]{Name}
while ( condition )
{
    statement();
}
\end{lstlisting}
\end{minipage}\hfill
\begin{minipage}{.50\textwidth}
\begin{lstlisting}[caption=A for loop in \gls{gamble}.,frame=tlrb, label=forlst, numbers=none, language=C]{Name}
for ( initialize; condition; update )
{
    statement();
}
\end{lstlisting}
\end{minipage}

\subsection{for}
An example of the for loop in \gls{gamble} is shown in \myref{forlst}.
An alternative to this syntax is one which allows an iteration over a collection such as a foreach or one that iterates over a range.
Examples of how such a syntax should be is shown in \myref{forextlst}.

\begin{lstlisting}[caption=Alternative syntaxes for the for loop.,frame=tlrb, label=forextlst, numbers=none, language=C]{Name}
// iterates over every int in the range.
for ( int i in range 0:10 )
{
    statement(i);
}
// iterates over every other int in the range.
for ( int i in range 0:10:2 )
{
    statement(i);
}
// iterates over each element as a reference to the element in the collection
foreach ( item i in collection )
{
    statement(i);
}
// iterates over each index in a collection
foreach ( index i in collection )
{
    statement(collection[i]);
}
\end{lstlisting}

These each have the strength of expressing clearly what they do, however they are easily expressed in the form expressed in \myref{forlst}.


