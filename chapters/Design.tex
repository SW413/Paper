\chapter{Design}
\label{cha:Design}
%Metatext for design chapter
In this chapter the design of the language called \gls{gamble} will be presented.\todo{Hvad står Gamble for?}
First with a brief description of our design philosophy for \gls{gamble} will be introduced to clarify what this language is attempting to achieve and why.
The design philosophy will address some of the criterias found in \myref{sec:OurCriterias}.
Furthermore the features and design of \gls{gamble} will be documented later in this chapter.

%\gls{gamble} is used to refer to our languange
\section{Design Philosophy}\label{sec:phil}

The focus for \gls{gamble} is to use the computational powers of the \acrshort{gpu} to handle such computations without it being inconvenient for the programmer.
As described in \myref{sec:state_of_the_art} several libraries and languages allows the programmer to explicitly designate workloads to the \acrshort{gpu} however this often requires explicit memory handling as well.
Keeping the rules and syntax of \gls{gamble} familiar to other languages makes it more accessible and reduces the time required for users to familiarize themselves with it.
This is done by using a C-like syntax, while stripping \gls{gamble} of features we deem not necessary for a language which main focus lies in using the \acrshort{gpu} for  linear algebra calculations.
The C-like syntax is chosen because the top 5 languages on Tiobe's list of most popular programming languages are C-like, and therefore if \gls{gamble} resembles these languages, it will be more familiar to start programming in \gls{gamble} which in turn makes it easier to implement ones algorithms.\citep{TIOBE}
As the \acrshort{gpu} is the resource being used to achieve more computational power, the data computed must also be applicable to the niche of the GPU, i.e. the data must be parallelisable as explained in \myref{sec:comparch}.
This basic need will influence how data is represented, and also put focus onto matrices and vector calculations, which as mentioned before can often be parallelised.

\textbf{Allow the programmer to use the \acrshort{gpu} without it being inconvenient}

Due to \gls{gamble} being focused on numerical computations, allowing the programmer to focus on managing the mathematical aspects is the main focus.
Therefore having the programmer control the runtime architecture seems an unnecessary distraction.
\gls{gamble} takes care of designating the computations to the best suited processing unit, whether it be the \acrshort{gpu} or the \acrshort{cpu}, as such any inconvenience in that process is removed from the programmer whose focus can be solely on the mathematics.
This abstraction therefore results in better writability.

\info[inline]{Maybe we should add automatic memory management here, it is *HARD* in OpenCL, CUDA etc. so it would make a nice point? -- Troels}

\textbf{Avoid implementing unnecessary data types and features}

As the purpose of \gls{gamble} is to use the \acrshort{gpu} for calculations which can be parallelised implementing features or data types, that do not hold any regard to this aspect would clutter the language.
Additionally \gls{gamble} should not try to adapt itself towards purposes for which it is not designed, an example of excluding such features is the fact that strings are not part of the language, this choice and others like it are further documented later in this chapter.
This makes \gls{gamble} simpler, and therefore both easier to read and write.
\todo{Denne udtalelse gør at vi nok bør lytte til Thomas og helt slette afsnit 14.6 om general purpose Gamble.. - Søren}

\textbf{Let the language be somewhat familiar to read and use}

As mentioned the main purpose of \gls{gamble} is to use the \acrshort{gpu} for computations, and is focused on doing computations, not developing new software.
As such \gls{gamble} would most often be used where this niche is required.
It may even be likely that it is not used when developing an algorithm to do computations, but first used once the algorithm is to be implemented during the algorithm design process, and can then be applied to bigger sets of data.\citep{AlgorithmDesign}
This is because of the large overhead as explained in \myref{sec:comparch}, it will be faster to perform a test of an algorithm with smaller datasets in other languages, like C.
Therefore to use the niche that \gls{gamble} proclaims, having the language be familiar makes it easier to use for its pure computational aspect, and improves on \gls{gamble}'s read- and writability.

In the following sections different parts of \gls{gamble} will be investigated.

\section{Core syntax choices}
The source code must be encoded in UTF-8, but only the ASCII alphabet and numbers are allowed for identifiers and values, this is to simplify the parsing. 
The language is case sensitive, because the most used languages seen on TIOBEs index\citep{TIOBE} are also case sensitive.
When a language is case sensitive it is possible for the programmer to name different parts of the code like he pleases, resulting in higher readability.
Some languages, such as Python, uses whitespace and indentation to indicate scope, in our language scope is indicated by curly brackets like many existing languages\texttt{\{\}}.  

\subsection*{Comments}
It is useful to be able to write comments in code to annotate the meaning of some code.
If a language were not to have the possibility to write comments, then the source code could be hard to understand.
Source code without comments might be difficult to return to after some time away from it, since the programmer would have to reread the entire code, instead of a few good comments.\citep{Commenting}
Therefore both single line and multi line comments can be used in \gls{gamble}. 
``//'' is used for single line comments, meaning that everything after the ``//'' until the next newline is ignored by the compiler, also known as C++ style comments. 
``/* */'' is used for multi line comments, meaning that everything between ``/*'' and ``*/'' is ignored by the compiler, also known as C style comments. 

\subsection*{End of statement terminator}
When choosing an end of state terminator there are two main choices.
The newline character as a terminator or another symbol, often semicolon (;).
The use of newline is often used in simple languages e.g. MATLAB or other mathematical oriented languages. 
We deem the main value of using the newline are simplicity and a decrease in errors due to forgotten semicolons.
Semicolon is as mentioned the other main choice, this choice has been used through many programming languages throughout the history. 
The main advantage of a end of statement terminator  like semicolon is that it allows freeform code. 
This freeform allows to have multiple statements in one line or allows long statements to span across multiple lines.
We deem this to enhance the readability of the code.
Furthermore the semicolon terminator allows for a simpler syntax and compiler design.
As mentioned before forgotten semicolons can result in compiler errors, we deem this as the biggest disadvantages of this end of statement terminator.

Based on this analysis we have chosen to use the semicolon as end of statement terminator, for easier parsing.

\subsection*{Scope}
If a programming language contain no implementation of scoping rules, then one must assume that all the source code in a given program would be in the same scope.
Such a single scope would mean that any variable can be used form any place in the code. 
Mostly, as briefly described in \myref{sec:state_of_the_art}, a programming language implements some kind of scoping rules.
There are several reasons for this, the main reason being it gives the programmer more control of the code and reduces the chance of changing variables by accident.
There exist two main methods of scoping, static and dynamic scoping.
As seen in \myref{sec:state_of_the_art} most of the languages examined in this section used a form of static scope. 
\gls{gamble} will use this form of scoping because we find it to be the easiest to work with and less prone to errors than dynamic scoping.
The scope rules of \gls{gamble} are as follows:

A variable is in scope from its declaration until the end of the block it is declared in.
An inner block such as a control flow structure, see \myref{subsec:control-flow} for more, has access to the variables from the outer scope. 
It is also not allowed to redeclare a variable which is already declared in the current scope. 
Additionally any function is in scope of any other function including itself, this also allows recursion. 
- This is static scoping without information hiding. 

\subsection*{Structure} 
The structure of the source code to a \gls{gamble} program is shown in \myref{lst:Structure}.
For a high level of readability we have decided  to split the source code up in segments.
A single segment contains only a small part of source code.
Including of other files are, like in many other languages, written in the top segment, this is to make it easy to get an overview of any and all included libraries. 
All function declarations are written in the next segment followed by the main statements which is the last segment.
This collects the main statements, which makes it easier for a programmer to get an overview over the order of execution, compared to if the main statements were allowed between the function declarations.
This sectioning also makes it easier to write a context free grammar for the language.

\begin{lstlisting}[caption={Source code file layout in \gls{gamble}},frame=tlrb,label={lst:Structure}, numbers=none]
Libary inclusions

Function declarations

Statements
\end{lstlisting}

\section{Types and Variables} \label{sec:Types}
%http://www.informit.com/articles/article.aspx?p=2103809&seqNum=3
\gls{gamble} uses three primitive data types; integers (int), floating points (float) and boolean values (bool), as well as two composite data types; matrices and vector. 
In \gls{gamble} any variable must be initialised when it is declared. 
This avoids any null reference which can cause runtime errors in many languages such as C, C\# and Java.
In \gls{gamble} all variables are statically stored, there are several reasons for this and therefore the language does not implement neither explicit or dynamic memory handling.
The first reason for this is that the purpose of the language is to perform arithmetic operations on the \acrshort{gpu}, so by excluding this the language's simplicity increases, and therefore its read- and writability increases.
A second reason being that the size of every variable in \gls{gamble} is unchangeable.
Memory handling could increase the number of possible programs in the language and is therefore seen as a viable extension to future versions of \gls{gamble}.


\subsection*{Integers and floating points}
An integer is a whole number either positive or negative, represented by the two's compliment method. 
Meaning that it ranges from $-(2^{n-1}) $ to $2^{n-1} - 1 $, where n is the number of bits used.
A floating point number is an approximation to a real number. 
It is separated in 3 parts, the sign bit, the mantissa and the exponent. 
Floats in \gls{gamble} are in base 2, the value of a float is calculated by $ (-1)^{sign bit} \cdot mantissa \cdot 2^{exponent} $. 
Integers and floating point numbers default to 32 bits with the keywords \texttt{int} and \texttt{float}. 
It is possible to declare 16 and 64 bit integers and floating point numbers by postfixing their size to their respective keywords, e.g. \texttt{float64} or \texttt{int16}. 
It is useful to be able to specify the size of an integer or a floating point number, as data transfer can cause a lot of overhead particularly when transferring data to and from the \acrshort{gpu}. 

\subsection*{Boolean}
A boolean value is either true or false. 
These are used for control structures in \gls{gamble} which takes a boolean value as a parameter. 
But can also be used for any means which the programmer finds useful. 

\subsection*{Vectors and Matrices}
A vector in \gls{gamble} is a sequence of entries which are each the same primitive data type. 
The dimension of a vector is immutable, meaning that its dimension cannot be changed after its declaration. 
A matrix is multiple row vectors inside a column vector, each row vector is the same dimension. 
As for other variables both vectors and matrices must be initialised when they are declared. 
This is done with the syntax shown in \myref{lst:matrix}.
Here it is also seen why the use of ``;'' is a smart choice for ending statements.

\begin{lstlisting}[caption={Creating a matrix},label={lst:matrix},numbers=none]
matrix<int> m = [1, 0; 
                 0, 1];

rowvector<int> rv = [1, 0];
\end{lstlisting}

\subsection*{Operators}
Available operators in the language can be found on \myref{tbl:operators} with a short description.  
It is not possible to use operators with matrices or vectors, functions which support different operations on these types will be included in a library.
\begin{table}[h]
    \centering
    \colorlet{shadecolor}{gray!40}
    \rowcolors{1}{white}{shadecolor}
    \begin{tabular}{|c|l|l|l|l|l|}
    \hline
    \textbf{Operator}  & \textbf{Integer}                   & \textbf{Floating point number}    & \textbf{Boolean}      & \textbf{Matrix}       & \textbf{Vector}     \\ \hline
    +                  & Addition (infix)                   & Addition (infix)                  & Type error            & Addition (infix)      & Addition (infix) \\ \hline 
    -                  & Subtraction (infix)                & Subtraction (infix)               & Type error            & Subtraction (infix)   & Subtraction (infix) \\ \hline 
    ++                 & Increment (unary,pre-, postfix)    & Increment(unary,pre-, postfix)    & Type error            & Type error            & Type error \\ \hline    
    --                 & Decrement (unary,pre-, postfix)    & Decrement(unary,pre-, postfix)    & Type error            & Type error            & Type error \\ \hline
    *                  & Multiplication (infix)             & Multiplication(infix)             & Type error            & Type error            & Type error      \\ \hline
    \%                 & Remainder (infix)                  & Modulo(infix)                     & Type error            & Type error            & Type error  \\ \hline
    /                  & Division (infix)                   & Division (infix)                  & Type error            & Type error            & Type error \\ \hline
    \&\&               & Type error                         & Type error                        & Logical and (infix)   & Type error            & Type error \\ \hline 
    ||                 & Type error                         & Type error                        & Logical or (infix)    & Type error            & Type error \\ \hline 
    ==                 & Equality (infix)                   & Equality* (infix)                 & Type error            & Type error            & Type error \\ \hline 
    !=                 & Inequality (infix)                 & Inequality* (infix)               & Type error            & Type error            & Type error \\ \hline
    <=                 & Less than or equal to (infix)      & Less than or equal to* (infix)    & Type error            & Type error            & Type error \\ \hline
    >=                 & More than or equal to (infix)      & More than or equal to* (infix)    & Type error            & Type error            & Type error \\ \hline
    <                  & Less than (infix)                  & Less than (infix)                 & Type error            & Type error            & Type error \\ \hline
    >                  & More than (infix)                  & More than (infix)                 & Type error            & Type error            & Type error\\ \hline
    \end{tabular}
    \caption[List of operators on primitive data types in GAMBLE.]{List of operators on primitive data types in GAMBLE.\@*Equality comparisons on floating point number may be incorrect.}\label{tbl:operators}
\end{table}
\vspace{-20pt}

%\subsection*{Memory handling}
%Locally declared and small and simple variables will be written to the stack, while large quantities of data must be written to the heap. 
%To manage the heap in a programming language one must decide how the management must be implemented.
%At the topmost level there exist to different approaches, the first being explicit memory handling as seen in C with function calls like \texttt{alloc()} and \texttt{free()}. 
%The second approach is dynamic memory handling, often called garbage collection.
%While  manual memory handling is a very low level concept, it has some advantages over dynamic memory handling, for instance manual handling gives the users full control over a programs resources and while this can be hard o get right, can both increase the simplicity of the compiler and increase a programs runtime execution speed because garbage collection can be a complex runtime process.\citep{Sebesta, gribble}

\section{Functions}
\todo{Forklar print funktionen}
Functions are blocks of code which can be called to do a specific computation, often but not always, a function takes some variables as inputs.

This section will describe \gls{gamble}'s use of functions. 
Like in many other languages it is possible to declare your own functions in \gls{gamble}.
This is useful for organising code, and reusing parts of the source code.
Functions in \gls{gamble} can take input, but are not required to, the inputs can be any of the types the languages support.
Because of high implementation cost in the compiler, it is not possible to use a function as an input for a function.


\subsection*{Function identifiers}
When declaring a \gls{gamble} function it is required to write the body of the function immediately after.
This is to avoid lookahead which function prototypes in C requires.
A function identifier is exactly like C, with the \texttt{return type} then the \texttt{functionname}, \texttt{(formal parameters)} and finally the \texttt{\{body\}}.
The function body can contain a number of statements and various control flows.
Furthermore it can contain calls to other functions or calls to the function itself, and hereby recursion is a possibility in \gls{gamble}
An example of a function identifier can be seen on \myref{lst:functionID}.
Other syntaxes could have been used, but not only is this the way C does it, it is also widely used in many other languages i.e. C\# and Java.

\begin{lstlisting}[caption={Function Identifier},label={lst:functionID},numbers=none]                                                        
int add(int a, int b)
{
    return a + b;
}
\end{lstlisting}

\subsection*{Function calls}
After a function has been declared it can be called. 
A function call contains the identifier and parameters for the function. 
The syntax in \gls{gamble} is the same as in many other programming language and shown in \myref{lst:functionCall}. 

\begin{lstlisting}[caption={A function call in \gls{gamble}},label={lst:functionCall},numbers=none]
add(4, 3);
\end{lstlisting}


\subsection*{Return value}
Functions have a return value which can be seen on \myref{lst:functionID}.
The return values are all the types in \gls{gamble} mentioned in \myref{sec:Types} and void. 
A void function will not return anything, but instead can be seen as a procedure.
This choice was made because \gls{gamble} should be able to return all the different types in the language from a function, but also be able to perform these procedures which a void function may do.
The value or type to be returned is preceded by the keyword \texttt{return}, just like it is in C and other C-like languages.
An example can be seen on \myref{lst:returnFunction}.

\begin{lstlisting}[caption={Return Function},label={lst:returnFunction}]
int a = 0;

a = add(4, 3);
\end{lstlisting}


\subsection*{Print function}\todo{rename maybe}
In \gls{gamble} we have one function that is ready-made. It is a print function.
The function is special because it can both take variables, numerical values and strings as input parameter.
Since string is not a valid data type in \gls{gamble} this function is special. 
With the print function it is possible to give the user a visual representation of the computations result, in the console.
A simple print function which take string inputs increases both readability and writability.
Strings in the print function allows the programmer to write easier understandable and readable program output. 
%STD LIB EVT?!
\section{Control Flow}\label{subsec:control-flow}\todo{Troels: Imo er hele dette afsnit på en måde for langt. }
Control flow is any part of the language in which a branch occurs. 
A branch is in this case a change in which instruction happens next. 
These come in three variations: primitives, choice and loops.
C, C\#, Java etc. contain some or all of these elements to control the flow of a program:

\textbf{Primitive:}
\begin{itemize}[noitemsep,topsep=-5pt] %removes whitespace before and between points.
    \item \texttt{goto}
\end{itemize}

\textbf{Choice:}
\begin{itemize}[noitemsep,topsep=-5pt] %removes whitespace before and between points.
    \item \texttt{if/if-else}
    \item \texttt{?:} (Ternary operator)
    \item \texttt{switch}
\end{itemize}

\textbf{Loops:}
\begin{itemize}[noitemsep,topsep=-5pt] %removes whitespace before and between points.
    \item \texttt{while}
    \item \texttt{for}
    \item \texttt{foreach}
    \item \texttt{do..while}
\end{itemize}

If a programming language has if, goto and label then the other constructs can be made.
For example a while-loop can be translated to a combination of the if and goto statements, this is shown in \myref{ifgotowhile1} and \myref{ifgotowhile2} written in C. 

\noindent\begin{minipage}{.45\textwidth}
\begin{lstlisting}[caption=Loop made with while.,frame=tlrb, label=ifgotowhile1, numbers=none]{Name}
while ( condition )
{
    statement();
}
\end{lstlisting}
\end{minipage}\hfill
\begin{minipage}{.45\textwidth}
\begin{lstlisting}[caption=The same loop with if and goto.,frame=tlrb, label=ifgotowhile2, numbers=none]{Name}
label startOfLoop:
if ( condition )
{
    statement();
    goto startOfLoop;
}
\end{lstlisting}
\end{minipage}

The same can be done with the \texttt{for} and \texttt{do..while} loops. 
This means that the only need control structures should be \texttt{if} and \texttt{goto}. 
However the use of goto is considered harmful, because it is easy to make mistakes when using it. \citep{DijkstraGoto}
Even though \texttt{for} and \texttt{while} are similar, they are often used in different contexts.
The for loop is useful for iterating over a collection or knowing exactly how many time the code is executed. 
Therefore we have decided to include both \texttt{for} and \texttt{while} in \gls{gamble}, but not \texttt{foreach} and \texttt{do..while}, this is to simplify the number of control structures.

We have decided to keep the traditional \texttt{if} and \texttt{if-else} constructs. 
These are useful, and can achieve the same as a \texttt{switch}, by chaining if-else statements, so we have excluded the switch construct. 
The ternary operator, also known as the inline if, written as \texttt{?:} in C and many other languages, is useful for compacting code to make it more readable, therefore we include it in \gls{gamble}.

\subsection{if, if-else and ?:} \todo{Jeg blander meget rundt i statement og construct, kan ikke helt beslutte mig. }
Examples of if/if-else \gls{gamble} are shown in \myref{iflst} and \myref{ifelselst}. 
The condition is enclosed in a parenthesis this is to make the if construct similar to the function call, in which the parameters are in a parenthesis. 
All the conditional code must be enclosed in a curly bracket, this is to make the code more readable. 
The use of curly brackets is to make them similar to functions to make the language more alike \todo{mangler et godt fagord her. marc: more familiar/relateable/orthogonal?}.

\noindent\begin{minipage}{.45\textwidth}
\begin{lstlisting}[caption=An if statement in \gls{gamble}.,frame=tlrb, label=iflst, numbers=none]{Name}
if ( condition )
{
    statement();
}
\end{lstlisting}
\begin{lstlisting}[caption=A use of \texttt{?:} in \gls{gamble}.,frame=tlrb, label=terlst, numbers=none]{Name}
condition ? ifTrue() : ifFalse();
\end{lstlisting}

\end{minipage}\hfill
\begin{minipage}{.45\textwidth}
\begin{lstlisting}[caption=An if-else statement in \gls{gamble}.,frame=tlrb, label=ifelselst, numbers=none]{Name}
if ( condition )
{
    statement();
} else {
    alternative_statement();
}
\end{lstlisting}
\end{minipage}

An example of the ternary operator \texttt{?:} is shown in \myref{terlst}. 
If the boolean statement before the question mark is true, then the first statement is executed otherwise the second is. 
The ternary operator can also be used for an assignment, only if the value getting assigned to is left of the question mark.
As the ternary operator allows any statement in its middle body, it is possible to chain them together, just like a if-else chain. 

\subsection{while}
An example of the while loop in \gls{gamble} is shown in \myref{whilelst}. 
The syntax is identical to the if statement, which also makes it look like a function in certain ways. 
It's also identical to the syntax found in many other programming languages and is therefore similar to many programmers. \todo{Skal sådan et argument være med?}

\noindent\begin{minipage}{.40\textwidth}
\begin{lstlisting}[caption=A while loop in \gls{gamble}.,frame=tlrb, label=whilelst, numbers=none]{Name}
while ( condition )
{
    statement();
}
\end{lstlisting}
\end{minipage}\hfill
\begin{minipage}{.50\textwidth}
\begin{lstlisting}[caption=A for loop in \gls{gamble}.,frame=tlrb, label=forlst, numbers=none, language=C]{Name}
for ( initialize; condition; update )
{
    statement();
}
\end{lstlisting}
\end{minipage}

\subsection{for}
An example of the for loop in \gls{gamble} is shown in \myref{forlst}.
An alternative to this syntax is one which allows an iteration over a collection such as a foreach or one that iterates over a range.
Examples of how such a syntax should be is shown in \myref{forextlst}.

\begin{lstlisting}[caption=Alternative syntaxes for the for loop.,frame=tlrb, label=forextlst, numbers=none, language=C]{Name}
// iterates over every int in the range.
for ( int i in range 0:10 )
{
    statement(i);
}
// iterates over every other int in the range.
for ( int i in range 0:10:2 )
{
    statement(i);
}
// iterates over each element as a reference to the element in the collection
foreach ( item i in collection )
{
    statement(i);
}
// iterates over each index in a collection
foreach ( index i in collection )
{
    statement(collection[i]);
}
\end{lstlisting}

These each have the strength of expressing clearly what they do, however they are easily expressed in the form expressed in \myref{forlst}.

\section{Reserved Words}
% (fold)
\label{sec:reserved_words}
In \gls{gamble} there exists a number of reserved words used for types and language constructs among others.
These reserved words, also known as language keywords, cannot be used as identifiers in the language. 
However, not all keywords are reserved words.
This is because keywords such as \texttt{if} and \texttt{for} are often followed by a specific construct.
In fact most cases of keyword usage require some sort of special construct, hereby rendering the keyword unambiguous.\todo{Det ved vi vist ikke hvad er på det her tidspunkt?  - Søren, Nu er unambigious jo ikke noget fagterm så det kan vi sagtens bruge - Marc | enig med Marcus - msm}
As an example, this makes it possible to use \texttt{if}, along with all other keywords, as identifiers in FORTRAN, but this is not the case in \gls{gamble}. \citep{fortran_identifiers}
Keywords in \gls{gamble} are considered reserved, because it minimises the lookahead needed, and therefore simplifying the lexing process while also increasing readability.\todo{Man ved ikke hvad en lexer er på dette tidspunkt. - Corlin}
All the reserved words of \gls{gamble} can be seen in \myref{res:words}.
\begin{table}[h!]
	\centering
	\def\arraystretch{1.5} \setlength{\tabcolsep}{2em}
	\begin{tabular}{l l l l l}
        \texttt{int}     & \texttt{int16}     & \texttt{int32}     & \texttt{int64}     & \texttt{bool}    \\
        \texttt{float}   & \texttt{float16}   & \texttt{float32}   & \texttt{float64}   & \texttt{void}    \\
        \texttt{matrix}  & \texttt{vector}    & \texttt{true}      & \texttt{false}     & \texttt{fileToMatrix} \\
        \texttt{if}      & \texttt{else}      & \texttt{for}       & \texttt{while}     & \texttt{matrixToFile} \\
        \texttt{print}   & \texttt{rows}      & \texttt{cols}      & \texttt{return}    & \texttt{include} \\
    \end{tabular}
    \caption{Reserved words of \gls{gamble}}\label{res:words}
	\def\arraystretch{1}
\end{table}

Besides the reserved words \gls{gamble} contains literals, which also cannot be used as identifiers.
The literals in \gls{gamble} are \texttt{true} and \texttt{false}.\todo{Er tal ikke også literals :p ?}
If \gls{gamble} were to provide a \texttt{null} value, this would be a literal as well.

% section reserved_words (end)
