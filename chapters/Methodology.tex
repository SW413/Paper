\chapter{Methodology}\label{chap:methdology}

This chapter will briefly explain the development method we will use to organise our work to find an answer to our problem statement. 
Inventing a programming language and developing a compiler for it, is a complex and time consuming process, so having a method for the development helps, which is why our methodology is discussed in this paper.

\section{Development Method}
In order to develop a language and its compiler, a comprehensive understanding of how programming languages work, and how compilers are structured is required.
Therefore we need to use the tools taught in the courses of this semester.
As these tools are acquired over a period of time the planning and development of a language cannot be done beforehand; with this in mind,, it makes sense to use an iterative development process.
One such method is Scrum, which group members have had success with on previous semesters.
Therefore we choose to develop this project using an adaptation Scrum.
In our adaptation we will make use of Scrums organisational tool, such as the Scrum board, daily Scrums and sprints.

The Scrum board brings an overview of active subproblems to be solved for the project.
To solve these the daily Scrums are used to partition workloads to each member of the project as well as acknowledge potential issues.
The sprints keeps the development iterative which in turn allows us to use newfound knowledge that we may not have had in previous sprints.
We will not have the specific roles which you would normally find in Scrum, such as the Scrum master or the product owner. \citep{Scrum}
The reason for this is, partly due to development methods not being in the curriculum for this project, but also due to the project not having use of these tools.
There is no customer for whom we are developing the compiler nor is there really time for one person in the group to be Scrum master, as he would still need to focus a lot on the tasks the rest of the group are also partaking in.
With this method we are able to work in short sprints of one to two weeks, which gives us the opportunity to incorporate the tools, concepts, and methods we learn during the development process.

We have a hard deadline to take into account, which forces us to have a certain number of sprints, instead of being able to develop on the project eternally.
That is not to say we do not plan ahead, but our long term plans are versatile, i.e. if we learn of something new, it is easier for us to implement this new knowledge in our project by going back to fix mistakes and adding features, compared to if we used linear development method.

% ALTERNATIVE?!!?
%\section{Development Method}
%We have decided to use parts of the agile development method Scrum. 
%Our development method is not Scrum but iterative with parts of Scrum to organise group work. 
%The parts of Scrum which we are using is the Scrum board, daily Scrum and sprints with user stories. 
%This is to make communication, organisation, and teamwork easier. 