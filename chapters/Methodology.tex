\chapter{Methodology}\label{cha:methdology}   

This chapter will briefly explain the development method used to organise the work to find an answer to the problem statement. \todo{kluntet slutning MP}
Designing a programming language and developing a compiler for it, is a complex and time consuming process, so having a method for the development helps, which is why methodology is discussed in this paper.

\section{Development Method}
In order to develop a language and its compiler, a comprehensive understanding of how programming languages work, and how compilers are structured is required.
Therefore the tools taught in the courses of this semester will be used.
As these tools are acquired over a period of time the planning and development of a language cannot be done beforehand.
With this in mind, it makes sense to use an iterative development process.
One such method is Scrum, which group members have had success with in previous semesters.
Therefore this project will be developed using an adaptation of Scrum.
The adaptation we will make use of Scrums organisational tool, such as the the Scrum board and daily Scrums.

The Scrum board brings an overview of active subproblems to be solved for the project.
To solve these subproblems, the daily Scrums are used to partition workloads to each member of the project as well as acknowledging potential issues.
The sprints keeps the development iterative which in turn allows the use of newfound knowledge that may not have been available in previous sprints.
We will not have the specific roles which are normally found in Scrum, such as the Scrum master or the product owner. \citep{Scrum}
The reason for this is, partly due to development methods not being in the curriculum for this project, but also due to the project not having any use for these tools.
There is no customer for whom the project is being developed nor is there really time for one person in the group to be Scrum master, as he would still need to focus a lot on the tasks the rest of the group are also partaking in.\todo{Man har vel tid til det ? Har man ikke ?}\todo{Er vores kunde ikke defineret udfra problemformuleringen? MP}
With this method we are able to work in short sprints of one to two weeks, which gives the opportunity to incorporate the tools, concepts, and methods we learn during the development process.

The project has a hard deadline to take into account, which forces the project to only iterate over the work a certain number of times, instead of being able to develop on the project eternally.
That is not to say we do not plan ahead, but rather that the long term plans are versatile, i.e. if we learn of something new, it is easier to implement this new knowledge in the project by going back to fix mistakes and adding features, compared to if something new was learned using a linear development method, like the waterfall method.
\todo[inline]{Måske nævne af rapportens opbygning i høj grad afspejler denne arbejdsmetode? er dette relevant? MP}
\todo[inline]{Er dette afsnit brugbart? Giver det noget positivt til rapporten? -- TK}

% ALTERNATIVE?!!?
%\section{Development Method}
%We have decided to use parts of the agile development method Scrum. 
%Our development method is not Scrum but iterative with parts of Scrum to organise group work. 
%The parts of Scrum which we are using is the Scrum board, daily Scrum and sprints with user stories. 
%This is to make communication, organisation, and teamwork easier. 