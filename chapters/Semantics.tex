\chapter{HER DIN NOOB}
\subsection*{Arithmetic}
The semantics of statements and declarations both depend on the semantics of arithmetic expressions.
Arithmetic expressions may contain variables.

\subsubsection{Addition.}
\begin{equation}
	\frac { { e }_{ v },st\vdash { a }_{ 1 }{ \rightarrow  }_{ A }{ v }_{ 1 }\quad { e }_{ v },st\vdash { a }_{ 2 }{ \rightarrow  }_{ A }{ v }_{ 2 } }{ { e }_{ v },st\vdash { a }_{ 1 }+{ a }_{ 2 }{ \rightarrow  }_{ A }{ v } } , v = { v }_{ 1 }+{ v }_{ 2 }
\end{equation}


\subsubsection{Subtraction.}
\begin{equation}
	\frac { { e }_{ v },st\vdash { a }_{ 1 }{ \rightarrow  }_{ A }{ v }_{ 1 }\quad { e }_{ v },st\vdash { a }_{ 2 }{ \rightarrow  }_{ A }{ v }_{ 2 } }{ { e }_{ v },st\vdash { a }_{ 1 }-{ a }_{ 2 }{ \rightarrow  }_{ A }{ v } } ,\quad v={ v }_{ 1 }-{ v }_{ 2 }
\end{equation}


\subsubsection{Multiplication.}
\begin{equation}
	\frac { { e }_{ v },st\vdash { a }_{ 1 }{ \rightarrow  }_{ A }{ v }_{ 1 }\quad { e }_{ v },st\vdash { a }_{ 2 }{ \rightarrow  }_{ A }{ v }_{ 2 } }{ { e }_{ v },st\vdash { a }_{ 1 }\ast { a }_{ 2 }{ \rightarrow  }_{ A }{ v } } ,\quad v={ v }_{ 1 }\ast { v }_{ 2 }
\end{equation}


\subsubsection{Boolean $\top$}
\begin{equation}
	\frac { { e }_{ v },st\vdash { a }_{ 1 }{ \rightarrow  }_{ A }{ v }_{ 1 }\quad { e }_{ v },st\vdash { a }_{ 2 }{ \rightarrow  }_{ A }{ v }_{ 2 } }{ { e }_{ v },st\vdash { a }_{ 1 }={ a }_{ 2 }{ \rightarrow  }true } ,{ v }_{ 1 }={ v }_{ 2 }\rightarrow true
\end{equation}


\subsubsection{Boolean $\bot$}
\begin{equation}
	\frac { { e }_{ v },st\vdash { a }_{ 1 }{ \rightarrow  }_{ A }{ v }_{ 1 }\quad { e }_{ v },st\vdash { a }_{ 2 }{ \rightarrow  }_{ A }{ v }_{ 2 } }{ { e }_{ v },st\vdash { a }_{ 1 }!={ a }_{ 2 }{ \rightarrow  }false } ,{ v }_{ 1 }={ v }_{ 2 }\rightarrow false
\end{equation}


\subsubsection{Parentheses.}
\begin{equation}
	\frac { { e }_{ v },st\vdash { a }_{ 1 }{ \rightarrow  }_{ A }{ v } }{ { e }_{ v },st\vdash { (a }_{ 1 }){ \rightarrow  }_{ A }{ v } }
\end{equation}


\subsubsection{Variables}
\begin{equation}
	{ e }_{ v },st\vdash n{ \rightarrow  }_{ a }v, { e }_{ v }(x)=\iota\quad and\quad st(\iota )=v
\end{equation}


\subsection*{Declaring Variables}
Any non-empty variable declaration will modify the variable environment since the new variables will be bound to the new location.
A variable declaration will also modify the store, since the new location will be initialised to contain the initial values of the new variables.
The transition relation describing variable declarations defines a big-step semantic, since the allocation of new address space for newly declared variables is an invisible operation.

\subsubsection{Transition.}
\begin{equation}
<{ D }_{ v },{ e }_{ v },st>{ \rightarrow  }_{ Dv }<{ e }_{ v }^{ i },{ st }^{ i }>
\end{equation}


\subsubsection{Empty Variables.}
\begin{equation}
	<\varepsilon,{ e }_{ v },st>{ \rightarrow  }_{ DV }<{ e }_{ v },st>
\end{equation}


\subsubsection{Variable declaration.}
\begin{equation}
	\frac { <{ D }_{ v },{ e }_{ v }^{ ii },st[\iota \mapsto v]>{ \rightarrow  }_{ Dv }<{ e }_{ v }^{ i },{ st }^{ i }> }{ <var\quad x:=a;Dv,{ e }_{ v }st>{ \rightarrow  }_{ Dv }<{ e }_{ v }^{ i },{ st }^{ i }> } 
\end{equation}
\begin{align*}
	where\quad &{ e }_{ v },st\vdash a{ \rightarrow  }_{ A }v \\
	&\iota {= e }_{ v }(next) \\
	&{ e }_{ v }^{ ii }={ e }_{ v }[x\mapsto \iota ][next\mapsto new \iota ]
\end{align*}

\begin{equation}
	{ D }_{ v }::=var x:=a; { D }_{ v } | \varepsilon
\end{equation}


\subsection*{Statements}
The effect of a statement is that the store may change, since a statement may modify the values of variables involved through assignments.
A statement should not modify the variable environment.
We define a BS-semantic for statements (except procedure calls)

\subsubsection{Transitions.}
\begin{equation}
	{ e }_{ v },{ e }_{ p }\vdash <S,st>\rightarrow { st }^{ i }
\end{equation}


\subsubsection{Transition system.}
\begin{equation}
	{((Stm\ast  Sto)\cup Sto \rightarrow Sto) }
\end{equation}


\subsubsection{Assignment.}
\begin{align*}
	&{ e }_{ v },{ e }_{ p }\vdash <x:=a,st>\rightarrow st[\iota \mapsto v]\\
	where\quad &{ e }_{ v },st\vdash a{ \rightarrow  }_{ A }v\quad and\quad  { e }_{ v }(x)=\iota 
\end{align*}


\subsubsection{Skip.}
\begin{equation}
	{ e }_{ v },{ e }_{ p }\vdash <skip,st>\rightarrow st
\end{equation}


\subsubsection{Compare.}
\begin{equation}
	\frac { { e }_{ v },{ e }_{ p }\vdash <{ s }_{ 1 },st>\rightarrow { st }^{ ii }\quad { e }_{ v }{ ,e }_{ p }\vdash <{ S }_{ 2 },{ st }^{ ii }>\rightarrow { st }^{ i } }{ { e }_{ v },{ e }_{ p }\vdash <{ s }_{ 1 };{ s }_{ 2 },st>\rightarrow { st }^{ i } } 
\end{equation}


\subsubsection{If $\top$.}
\begin{equation}
	\frac { { e }_{ v },{ e }_{ p }\vdash <{ s }_{ 1 },st>\rightarrow { st }^{ i } }{ { e }_{ v },{ e }_{ p }\vdash <if\quad b\quad then\quad { S }_{ 1 }\quad else{ \quad S }_{ 2 }>\rightarrow { st }^{ i } } st\vdash b{ \rightarrow  }_{ B }\top 
\end{equation}

\subsubsection{If $\bot$.}
\begin{equation}
	\frac { { e }_{ v },{ e }_{ p }\vdash <{ s }_{ 1 },st>\rightarrow { st }^{ i } }{ { e }_{ v },{ e }_{ p }\vdash <if\quad b\quad then\quad { S }_{ 1 }\quad else{ \quad S }_{ 2 }>\rightarrow { st }^{ i } } st\vdash b{ \rightarrow  }_{ B }\bot 
\end{equation}

\begin{comment}
\subsubsection{While $\top$.}
\begin{equation}

\end{equation}

\subsubsection{While $\bot$.}
\begin{equation}

\end{equation}

\end{comment}