%\gls{gamble} is used to refer to our languange
\chapter{Language Criterias}
When evaluating programming languages it can be difficult to agree on the value of a language's characteristics.
Sebesta have given some criterias which can be evaluated from, and we will take these criterias into account when designing our language.\citep{Sebesta}
The criterias can be seen on \myref{tbl:concepts}.
First a brief description of the criterias will be given, followed by how we rate these criterias for our language according to the information found in \myref{cha:introduction}.
\begin{table}[h]
	\centering
	\colorlet{shadecolor}{gray!40}
    \rowcolors{1}{white}{shadecolor}
	\begin{tabular}{|l|l|c|c|}
	\hline
	\textbf{Language Concepts}										& \textbf{Readability}  & \textbf{Writability} & \textbf{Reliability}   \\ \hline
	Simplicity                                                  	& x 					& x             		 & x           			\\ \hline
	Orthogonality                                               	& x 					& x             		 & x           			\\ \hline
	Data types                                                  	& x 					& x             		 & x           			\\ \hline
	Syntax design                                               	& x 					& x             		 & x           			\\ \hline
	Support for abstraction                                    		&                       & x             		 & x           			\\ \hline
	Expressivity                                                	&                       & x             		 & x           			\\ \hline
	Type checking                                               	&                       &            			 & x           			\\ \hline
	Exception handling                                          	&                       &             			 & x           			\\ \hline
	Restricted aliasing                                         	&                       &             			 & x           			\\ \hline
	\end{tabular}
	\caption{Language evaluation criteria and the characteristics that affect them.}
	\label{tbl:concepts}
\end{table}

For our language the three criterias readability, writability and reliability, are rated as follows.

\begin{enumerate}
	\item Writability
	\item Readability
	\item Reliability
\end{enumerate}

We want it to be easy for programmers and mathmaticians alike to write and read programs in our language.
The focus should be on calculating the problem at hand, without having to conform to weird language specifics, controlling memory etc.
Our language must be simple, which means that the number of basic constructs is kept low. 
This results in both better readability and writability as can be seen on \myref{tbl:concepts}.
The language supports abstraction over the programmer's hardware, by not having to specify where the computations must be run, or having to allocate memory.

\todo{Skal vi gå igennem dem allesammen dybdegående som vi nok ville have gjort sidste semester, eller er sådan en hurtig overskuelig gennemgang fin nok?}

\section{Success criterias}\label{sec:OurCriterias}
We deem our language and compiler to be a good solution for our problemstatement if it upholds to the following criterias:

\begin{itemize}
	\item Calculating matrix and vector calculations on the GPU, without the programmer specifically specifying it to do so.
	\item Can calculate very large matrix operations faster than regular C being run on the CPU.
	\item Expresses good read- and writability, by being simple and having a familiar syntax design.
	\item Has somewhat good reliability, with type and scope checking.
	\item Giving good error messages, for easier debugging, resulting in better writability.
\end{itemize}

Our overall thoughts of our language according to Sebesta's criterias can be found in \myref{sec:phil}.

