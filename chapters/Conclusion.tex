\chapter{Conclusion} % (fold)
\label{cha:conclusion}
For this project the goal was to create a language and a compiler for this language which could perform numerical computations on the gpu.
During the project different aspects of the process of creating a language and a compiler has been presented and different ways of handling these aspects has been discussed aswell as the implementation of these being presented.
The language called \gls{gamble} was designed to make it easier for programmers and mathmaticians alike to perform matrix and vector computations faster by using the power of the \acrshort{gpu}.
Using the operators defined in \myref{subsec:operators} along with matrices or vectors will be generated into OpenCL C code which runs the computations on the \acrshort{gpu}.
This makes the computations fasterm if a reasonably large enough size of the computations are reached, when compared to \gls{gamble} code which is being run on the CPU using ordinary C code instead of OpenCL C.

For the project the parser generator \acrshort{antlr}4 was used, which made a parser for the grammar specified by the project group.
This parser gives a parse tree as output, and this tree is then translated into a \acrshort{ast}.
The \acrshort{ast} makes a lot of what is done in the compiler simpler.
The \acrshort{ast} allowed for easier access to the nodes than the parse tree, while grouping information and using less nodes for information compared to the parse tree.
Scope and type checking is an essential part of the compiler and makes sure to identify many errors in the source code before runtime.

The project showed that there are many aspects in creating a language and a compiler.
The use of the \acrshort{gpu} is worthwhile when the size of the computations become sufficiently large, and creators of new languages or compilers should think of including ways of using this efficient resource.