\chapter{Conclusion} % (fold)
\label{cha:conclusion}
For this project the goal was to create a language and a corresponding compiler which could perform numerical computations on the \acrshort{gpu}.
During the project different aspects of the process of creating a language and a compiler has been presented and different ways of handling these aspects has been discussed as well as the implementation of these presented.
The language \gls{gamble} was designed to make it easier for programmers and mathematicians alike to perform matrix and vector computations faster by using the power of the \acrshort{gpu} without the programmer needing to specify so in the source code.
Using the operators, defined in \myref{subsec:operators}, along with matrices or vectors will compile to OpenCL C code which runs the computations on the \acrshort{gpu}.
As a result of running computations on the \acrshort{gpu} an increase in execution speed can be obtained; an increase in execution speed will only exist if of a significant size; \todo{myref til test af hvornår gamble er improvement} shows what data size is required for \gls{gamble} to provide an increase in execution speed by using the \acrshort{gpu} over the \acrshort{cpu}.
%This increases computation speed if a reasonably large enough size of the computations are reached, when compared to \gls{gamble} code which is being run on the CPU using ordinary C code instead of OpenCL C.

For developing the compiler the parser generator \acrshort{antlr}4 was used, which made a parser for the grammar specified by the project group.
This parser gives a parse tree as output; The parse tree is translated into a \acrshort{ast} by the compiler.
The \acrshort{ast} removes redundant information from the parse tree thus simplifying its traversal.
The \acrshort{ast} allowed for easier access to the nodes than the parse tree, while grouping information and using less nodes for information compared to the parse tree.
Scope and type checking are essential parts of the compiler and identifies errors in the source code before runtime.

The project showed that there are many aspects in creating a language and a compiler.
The use of the \acrshort{gpu} is worthwhile when the size of the computations become sufficiently large, and creators of new languages or compilers should think of including ways of using this resource for such purposes.