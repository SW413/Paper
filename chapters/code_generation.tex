%Phase3
\section{Code Generation}
\todo{Subphases of this is yet unknown, multipass and singlepass should be in this}
%Multipass
%Singlepass
\section{Design}
This section will cover the design of the code generator in the compiler, both the overall structure of the code generator and its requirements, but also its use of the \acrshort{gpu}.
Lastly it will present how the runtime of the programs could be made more efficient.
\subsubsection*{The Code Generator Visitor}
When the code generator is called, the \acrshort{ast} is used to accept a \texttt{CodeGeneratorVisitor} as seen on \myref{fig:CodeGeneratorVisitor}.

\begin{figure}[!ht]
\centering
 \includegraphics[width=0.8\textwidth]{figures/ClassDiagrams/CodeGeneratorCall.pdf}%trim=4cm 0cm 0cm 0cm, clip
\caption{A class diagram of \texttt{CodeGeneratorVisitor} showing the call from the code generator to the visitor which makes the string from the decorated \acrshort{ast}.}\label{fig:CodeGeneratorVisitor}
\vspace{-15pt}
\end{figure}

The visitor builds a string\todo{of object code -- Troels} by traversing the nodes of the tree which will in the end be put into a file by the \texttt{CodeGenerator} class.\todo{Hvad menes der med at det blive puttet i en fil? Det står måske lidt uklart. - Corlin}
Some visit methods like a visit method for a \texttt{ConstantExpressionNode} returns a string while other methods instead directly appends to the string the visitor will return to the class \texttt{CodeGenerator}.
This string is then used as an argument in other visit methods e.g. \texttt{AssignmentNode} which then produces a string which is a statement that can be executed in C. \todo{C eller OpenCL C ? -- Troels}
This string from \texttt{AssignmentNode} is one of the strings appended to the string containing the fully compiled program.
%%%%%%%%%%%%%%%%%%%%%%%%%%%%%%%%%%%%%%%%%%%%%%%%%%%%%%%%%%%%%%%%%%%%%%%%%%%%%%%%%%%%%%%%%%%%%%%%%%

As a result the information bubbles upwards from the leafs of the tree to the statement-nodes.
The information is put in the correct statement-nodes because the visitor pattern makes it possible to specify the route of traversal.
The \texttt{CodeGeneratorVisitor} also contains other methods for producing certain C constructions using the similar \gls{gamble} constructions found in the \acrshort{ast}.
These methods can also be seen on \myref{fig:CodeGeneratorVisitor}.
Some of these methods also make runtime checks of matrices, as matrices needs to be of compatible sizes for some of the operations which can be performed on a matrix, furthermore an index check is implemented such that an out of bounds error will occur if one tries to access memory beyond the bounds of the matrix.
When multiplying two matrices the left matrix of the multiplication has to be a $ N \times M $ while the right one has to be a $ M \times P $ matrix.
So the right matrix must have the same number of columns as the right matrix has rows.
If performing a matrix index multiplication, where every index is multiplied with the corresponding index of the the other matrix, the matrices have to have the same number of rows and columns. \todo{De 3 linjer her virker lidt random. Det hele er bare en stor uorganiseret brødtekst. -- Troels}

These checks will be inserted as a surrounding if statement around the matrix calculations, so they have to pass these checks before the calculation will be done, if it does not pass an error will be printed.\todo{Der kommer da en fejl right ? - Søren}
The reason this is done at run-time instead of at compile-time is because the sizes of matrices can be dynamic, which results in making it impossible to check for this\todo{Fjern ``for this`` ? - Corlin} at compile-time.

When the string is complete it is written to a file called code.c, along with all the other files needed for running the code, e.g. the kernels used for performing computations on the \acrshort{gpu}.
The code.c file is structured so that it is a valid C program.
First off are the libraries included that C uses, then follows a list of prototypes which are all the \gls{gamble} functions translated into C code, both the user made and the ones from libraries.
After the prototypes the main function is made where the body consists of all the statements in the \gls{gamble} sourcecode translated into C.
After the main method all the implementations of the function prototypes are made.

\subsection*{GPU Usage}\label{GPUCode}\change[inline]{This should probably be in the design part of codegen?}
Since \gls{gamble} distances its programmers from directly controlling which computations are performed on the \acrshort{gpu}, determining what code to perform on the \acrshort{gpu} becomes a problem for the compiler to solve.
To make this decision the compiler must know what kind of code performs faster on a \acrshort{gpu} than a \acrshort{cpu}.
From \myref{sec:comparch} it is clear that for it to make sense moving any computation to the \acrshort{gpu} it must be of significant size to make up for the overhead of moving data, and be executable in parallel.
If a computation is reliant on the outcome of other computations, the Fibonacci function as an example, moving it to the \acrshort{gpu} would be a significant decrease in performance compared to on a \acrshort{cpu}.

Any code written in a recursive format will not be run on the \acrshort{cpu} furthermore due to the overhead in data transfer, only computations requiring a significant amount of operations to be performed should be executed on the \acrshort{gpu} as \myref{image:benchmark} shows.
Therefore statements which only contains simple data types, i.e. integers, floats and bools, are performed on the \acrshort{cpu}.
An example could be \texttt{value = value1 + value2}, where all types are integers.
%Therefore statements not containing complex data types, i.e. statements with no vector or matrix arithmetics, are also performed on the \acrshort{gpu}.

However statements that do include matrix or vector arithmetics will be performed on the \acrshort{gpu}.
An example could be matrix multiplication.
Now it is entirely possible to make a matrix multiplication of a 2x2 matrix, which would be so small that the overhead of data transfer is more expensive than simply computing on the \acrshort{cpu}. % however as mentioned in \myref{sec:phil} a computation containing that little data is not what the languange is designed to compute.
The language is meant for larger computations that can actually benefit from using the \acrshort{gpu}, therefore the size of a given matrix or vector is irrelevant when the compiler considers what processor to use. \improvement[inline]{Er vi enige om dette sidste? Jeg syndes lidt vi burde skrive at for simplicitet har vi valgt at alle udregninger på matricer og vectorer sker på gpuen, og at vi forstår dette ikke altid er et godt valg. - Troels}

\subsection*{Optimisation/Runtime Efficiency}
As code is generated considerations of efficiency can be made such that the object code becomes more efficient than other alternatives.
Such optimisations consists of considerations like memory allocation, pipelining, parallelisation and other considerations that may have an impact on execution time.
To do an efficient and complete analysis of how to optimise code, information is required.
This is where \gls{gamble} is at a disadvantage due to \gls{gamble} distances its programmers from controlling where the code is performed and istead does so seamlessly.
As this is seamless, information is lackluster and therefore all computations may not be fully optimised in terms of where to execute, this is one of the tradeoffs \gls{gamble} makes.
If \gls{gamble} were to require more information, the simplicity and seamless use of the \acrshort{gpu} would be lost.

As previously mentioned due the object code being OpenCL C certain considerations pertaining to instruction handling are not of interest for optimisation for this compiler.
Instead the optimisation here resides in when to use the \acrshort{gpu}, optimising the generated C code as well as the OpenCL kernals.

As \gls{gamble} is attempting to seamlessly use the \acrshort{gpu} to increase performance, knowing when the use of the \acrshort{gpu} will actually be a benefit is an important point in the code generation process.
As mentioned to do this effectively would require more information about the computations which are to be done than can be read from the syntax of \gls{gamble}.
Therefore the project group have decided that it is better to be sure that a computation can benefit from the \acrshort{gpu}, rather than risking moving computations that wont benefit as well as those that will.
As such only those operations that the project group knows can benefit from the parallel abilities of the \acrshort{gpu} will be executed on the \acrshort{gpu}.

%Knowing when to use GPU
Even though a computation can be parallelised, does not neccessarily mean it should as is evident in \myref{image:benchmark}.
This is an opportune point for optimising the object code to use the \acrshort{gpu} only when it leads to an increase in execution speed.
A possibility of doing so would be to analyse whether or not an operation is both big enough and compatible with the \acrshort{gpu}.
Performing such an analysis increases the time it takes to do code generation, and even so in a greater part, because of the lackluster information about the source code in the aspect of whether it be efficient on the \acrshort{gpu}.
Because of the difficulty in dicerning not only if there will be an actual increase, but also if any custom functions created by the programmer are fit to run on the \acrshort{gpu} this optimisation is not made.
Instead only vector and matrix operations already defined in the languange will be performed on the \acrshort{gpu}.

%Optimising C code(CPU)

%Optimising OpenCL Kernals
A function that is to be run on the \acrshort{gpu} is in the OpenCL framework called a kernal.
Since kernal code uses explicit memory handling, one must choose what memory space to allocate ones variables in, in the different kernals the better one untilises this memory the faster a kernal can be executed, as a result of memory higher in the memory hierarchy.
\begin{figure}[h!]
\centering
 \includegraphics[width=1\textwidth]{figures/OpenCLOptimisation.png} % trim=4.85cm 15cm 0.85cm 1cm
\caption{Execution speed of a matrix multiplication with different optimisation levels. \citep{CUDAOpenCLOptimisation}}\label{image:OpenCLOptCompare}
\vspace{-15pt}
\end{figure}
As seen on \myref{image:OpenCLOptCompare} some possible optimisations include loop unrolling, common subexpression elimination and loop-invariant code mortion. These are taken as specific examples in this comparison because these optimisations are made in the \acrlong{ptx} code that CUDA compiles.
An OpenCL C compiler also provides the option of doing optimisation upon the code, however it would seem that depending on the \acrshort{gpu} and platform one is working on optimastions must be altered and furthermore the ideal work-group size changes, making universal optimisation a difficult task to take on.
Furthermore OpenCL uses JIT compilation to generate binary code to the appropriate device it is working with.
While this allows it to be used on more platforms than one unlike CUDA, it also results in compiler optimisations being quite time-consuming and increases the total execution time.\citep{CUDAOpenCLOptimisation}
\section{Implementation}
This section will show examples from the \gls{gamble} compiler's source code which generates the object code from a \gls{gamble} program's source code.
This will show the implementation of selected design choices made in the previous chapters.
First an example of how a \texttt{DeclarationNode} is translated into C code will be provided; afterwards the implementation of using the \acrshort{gpu} using OpenCL will be presented.
The compiler starts the code generation by calling an instance of the class \texttt{CodeGenerator} and invoking the method \texttt{GenerateCodeAndWriteToFile}.
This method then makes the \acrshort{ast} accept a CodeGeneratorVisitor and writes its output to a file; while also exporting the object code to a certain directory along with other files needed; the OpenCL kernels used in the program.
The \texttt{outputCode} is a string which starts as an empty string; every visitor then either appends or returns substrings to be appended.
These substrings add appropriate information to the \texttt{outputCode} string as the traversal is ongoing.
\begin{figure}
\centering
\includegraphics[width=0.5\textwidth]{figures/Trees/ASTAlone.PNG}
\caption{An \acrshort{ast} of a declaration \texttt{int a = 5;}}\label{fig:ASTAlone}
\end{figure}

\myref{fig:ASTAlone} shows a \texttt{DeclarationNode} for the expression \texttt{int a = 5;}, in the code generator this node has to be transformed into the declaration written in C. 
The syntax for this is actually the same in C as it is in \gls{gamble} but the node has been through the previous phases which means it has been statically type and scope checked, and therefore is ready to be computed.
The code from the visitor accepting a \texttt{DeclarationNode} can be seen on \myref{lst:DeclarationNodeCodeGen}.
\begin{lstlisting}[float, floatplacement=H!, caption=The visit method for visitting a DeclarationNode in the codegenerator. ,numbers=none,frame=tlrb,label={lst:DeclarationNodeCodeGen}]
@Override
public String VisitDeclarationNode(DeclarationNode node) {
    String expr = "";
    String complexType = "";
    if (node.getExpression() != null){
        resultVarStack.push(node.getVariable());
        expr = visit(node.getExpression());
        resultVarStack.pop();
    }

    if (node.getVariable().isComplex()) {
        ...
    }
    if (expr.indexOf("sclManageArgsLaunchKernel
    	(hardware, software, global_size, local_size") >= 0){
        ...
    }
    
    return complexType.length() > 0 ? complexType + expr : 
    (node.getVariable().toCcode() + " = " + expr + ";");
    }
\end{lstlisting}
Since the example \texttt{int a = 5;} is not a complex type like a matrix or vector the body of the if statement is hidden.
A check is made on the node to see if the expression which the declared variable is assigned from exists. 
Syntactically a matrix or vector can be uninitialised but this actually creates a vector or matrix filled with zeroes.
In the example the expression is not null so it goes into the body.
The result variable in which the result of the expression must be stored is pushed to a stack before visiting the expression.
In the method \texttt{VisitExpresssionNode} the top of this stack is checked to see if the result is a complex datatype or not, in the example it is not and the expression is then evaluated by visiting the nodes of the expression.
The result of the call to \texttt{VisitExpressionNode} is saved to a string \texttt{expr}.
Another if statement checks if the declaration needs a kernel and if the expression needs to be computed on the \acrshort{gpu}.
When returning the result of the call to \texttt{VisitDeclarationNode} a check is made if the complexstring has been made longer or not.
In the example \texttt{int a = 5;} is has not and therefore the string of the datatype and ID is concatenated with the assignment symbol, the substring \texttt{expr} and a semi-colon, before finally being returned.






In the following section an introduction of a library called SimpleOpenCL will take place.
\subsubsection*{Using SimpleOpenCl}
Simple\gls{opencl} is a library for C which simplifies the process of setting up and launching a kernel for \gls{opencl}.
The kernels remain the same, but finding the hardware for executing the kernels and allocating memory for the hardware is simplified.

The \texttt{CodeGeneraterVisitor} starts at the root of the \acrshort{ast} and here the code on \myref{lst:OpenCLSetup} is run.

\begin{lstlisting}[caption=Call to setup Simple\gls{opencl} in the compiler by appending it to a string builder,numbers=none,frame=tlrb,label={lst:OpenCLSetup}]
outputCode.append(filesNstuff.
	 ImportStringFromResource("codesnippets/simpleCLsetup.c") + "\n\n");
\end{lstlisting}
The file simpleCLsetup.c is appended to the code right as the main method of the output file is started.
The file contains the code which can be seen on \myref{lst:OpenCLSetup2}.

\begin{lstlisting}[caption=Simple\gls{opencl} setup in the compiler,numbers=none,frame=tlrb,label={lst:OpenCLSetup2}]
// Simple-\gls{opencl} Hardware setup
	sclHard* allHardware;
	sclHard hardware;
	sclSoft software;
	int found = 0;
	allHardware = sclGetAllHardware( &found );
	hardware = sclGetFastestDevice(allHardware, found);

    size_t local_size[2] = {1, 1};
    size_t global_size[2] = {1, 1}; 

    printf("\n");
// END Hardware setup
\end{lstlisting}

This code creates the elements needed to launch a kernel.
It finds the fastest hardware according to SimpleOpenCl's function calls, which means it finds the device with the most number of compute units, no matter the type of device, be it a \acrshort{cpu} or a \acrshort{gpu}.
For the remaining part of this section the fastest device is a \acrshort{gpu}.
\texttt{global\_size} and \texttt{local\_size} are there to determine the amount of memory needed both globally and locally on the \acrshort{gpu}.
The size of these arrays are initialised to two, because the \gls{gamble} matrices are two-dimensional.
These arrays are then filled out with different numbers corresponding to the columns and rows of the matrices or vectors being calculated upon.
This way of appending templates to the outputCode string is used in different places in the code generation when handling the complex datatypes matrices and vectors.
In fact whenever one of the following operators \texttt{+, -, *, \#, \^{} } are used with matrices or vectors a template is being appended to the outputCode. 
See \myref{tbl:matOps} for a description of what each operator will produce in \gls{gamble}.

When the right side of an assignment or declaration consists of an expression using operators and matrices or vectors, the visitor checks which operator is used and then inputs a template for launching the kernel depending on the operator.
The compiler contains files which have the code for launching the kernel for the specific situation and also for the kernel itself.
If a kernel is used the kernel file is added to the ``codeout'' directory along with the output code itself.

\myref{lst:kernelLaunch} shows one of the kernels being appended to the outputCode.

\begin{lstlisting}[caption=Simple\gls{opencl} launch of a kernel calculating a matrix or vector multiplied with a scalar.,numbers=none,frame=tlrb,label={lst:kernelLaunch}]
//MATRIX §MATRIX_A§ MULTIPLIED WITH A SCALAR §MATRIX_B§
global_size[0] = §MATRIX_A§.rows*§MATRIX_A§.cols;
local_size[0] = 1;
global_size[1] = 1;
local_size[1] = 1;
software = sclGetCLSoftware("matrixMulScalar.cl", "matrixMulScalar", hardware);
§MATRIXTYPE§ scl_scalar_mul§NUM§ = §MATRIX_B§;
// %R means that what is being sent can be read from and written to
// %a means that what is being sent is a non-pointer argument and is constant
sclManageArgsLaunchKernel(hardware, software, global_size, local_size, "%R %a",
    §MATRIX_A§.dataSize, §MATRIX_A§.dataStart, sizeof(§MATRIXTYPE§), &scl_scalar_mul§NUM§);
//END MATRIX SCALAR MULTIPLY
\end{lstlisting}

\texttt{global\_size} is set to be the size of the matrix, and a kernel is then launched for every index in the matrix.
This is decided by setting the indices in \texttt{global\_size}.
If the squared matrix was 2x2, a kernel would be launched with 0, 1, 2 and 3 as the indices.
The implementation of multiplying a matrix with a scalar is made where the matrix is interpreted as a single vector where each row comes after the other.
If the matrix form is needed the rows of the matrix would be placed in \texttt{global\_size[0]} and the columns in \texttt{global\_size[1]}.
If the example of a size 4 matrix is still used, the following sets of kernels would be sent:
\begin{equation}
\{0,0\}, \{0,1\}, \{1,0\}, \{1,1\}
\end{equation}
So a kernel for each index in the matrix.
This can then be used in the kernel to determine which row and which column, the index being sent to the kernel for execution, possess.
The \texttt{software} is where the kernel being launched is set, the file name and the kernel name in the file, the hardware for the execution must also be set.
Then the function \texttt{sclManageArgsLaunchKernel()} handles the launching itself with the variables needed to launch the kernel.
Before this code is appended any string with  \S-signs is is replaced by the corresponding string depending on the situation.
So \texttt{§MATRIX\_A§} is replaced with the id of the left matrix in the expression node.
The code for replacing the strings can be seen on \myref{lst:replaceString}.

\begin{lstlisting}[caption=Code for replacing strings with the corresponding information to be appended to the outputCode.,numbers=none,frame=tlrb,label={lst:replaceString}]
private String matrixKernel(String kernelName, String aID, String bID, String resID, String simpleType) {
    String kernel = filesNstuff.ImportStringFromResource("kernels/" + kernelName + ".cl");
    kernel = kernel.replaceAll("§MATRIXTYPE§", simpleType);
    filesNstuff.WriteToFile(new File("../../../codeout/" + kernelName + ".cl"), kernel);

    String argsNlauch = filesNstuff.ImportStringFromResource("kernelLaunch/" + kernelName + ".c");
    argsNlauch = argsNlauch.replaceAll("§MATRIX_A§", aID);
    argsNlauch = argsNlauch.replaceAll("§MATRIX_B§", bID);
    argsNlauch = argsNlauch.replaceAll("§MATRIX_RES§", resID);
    argsNlauch = argsNlauch.replaceAll("§MATRIXTYPE§", simpleType);
    argsNlauch = argsNlauch.replaceAll("§NUM§", Integer.toString(this.scalarNum));
    argsNlauch = argsNlauch.replaceAll("\\n", "\n" + indent(""));
    return argsNlauch;
}
\end{lstlisting}

However this must also be done for the kernel itself, in the kernels the type is changed depending on the type of the fields in the matrix or vector, which makes it possible to use the same code in the code generator for replacing these strings, since the type is handled dynamically.

The corresponding kernel for \myref{lst:kernelLaunch} can be seen on \myref{lst:kernel}.
\begin{lstlisting}[caption=Kernel code for multiplying a matrix or vector with a scalar.,numbers=none,frame=tlrb,label={lst:kernel}]
__kernel void matrixMulScalar(__global §MATRIXTYPE§ *ma, §MATRIXTYPE§ scalar){
	int global_x = get_global_id( 0);
	ma[ global_x] *=  scalar;
}
\end{lstlisting}

As can be seen the string \texttt{§MATRIXTYPE§} must also be replaced here.
The index which has been sent to the kernel is retrieved by calling \texttt{get\_global\_id(0);}.
The index is then used to access the matrix at the specific index and multiply the value at the index with the scalar.


\subsubsection*{From OpenCL to execution}\label{ssub:makefile}
The OpenCL C code must be compiled by a C-compiler such as the GNU Compiler Collection (GCC) compiler to produce code the \acrshort{cpu} and \acrshort{gpu} can execute.
During the compilation the OpenCL C headers must be accessible, to provide the functions and types used. 
Additionally a standard library is made for the compiler in C, containing functions such as \texttt{matrixToFile} and \texttt{fileToMatrix} which must also be included in the compiled OpenCL C code.
This produces an executable file for the platform targeted by the compiler, this will most often be the platform running, such as x86--64 on Linux.
The same executable file is not usable for another platform, e.g. ARM, nor on another operating system, e.g. Windows or OS X.
To maintain portability a ``MakeFile'' is used; a MakeFile is a file in the syntax which the GNU make utility can execute. 
This means that after running the \gls{gamble} compiler, on a configured computer, the make command can be used to build a binary file.

The kernel can either be compiled at compile-time or \acrshort{jit} compiled at runtime.
The \acrshort{jit} compilation, called online compilation under OpenCL, allows the system to adapt with platform specific performance improvements for the running platform.
If the kernel is compiled at compile-time, called offline compilation under OpenCL, then it might support fewer devices, or will take a long time and increase the size of the executable as it has to contain versions for each hardware for which support is requested. \citep{openclbookjit}
This is an example of a time-memory trade off. 
For \gls{gamble} the online compilation is chosen for its simplicity as it lets the driver of the running system compile the kernel. 