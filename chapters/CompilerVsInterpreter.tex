\chapter{Language runtime}
In this chapter it is analysed whether the source code should be compiled or rather interpreted.
Thereafter our target language is determined.
In the next section it is decided whether \gls{gamble} will utilise an interpreter or a compiler.

\section{Compiler vs Interpreter}
When designing an language an important decision is whether a source program in the language should be compiled to become a executable or the source code should be interpreted and run without compilation.
This is a choice thats need to be settled before the design of the compiler or interpreter can begin.
Both compilers and interpreters has it advantages and disadvantages, it is those that need to be evaluated in the context of the goal of the language.
The pro's and con's expressed in \myref{tbl:compint} does not reflect every difference between a compiler, but rather those aspect deemed important in respect to \gls{gamble}.
\begin{table}[h]
    \centering
    \colorlet{shadecolor}{gray!40}
    \rowcolors{1}{white}{shadecolor}
    \begin{tabular}{|l|l}
    \hline
    \textbf{Compiler}                           & \textbf{Interpreter}           \\ \hline
    Compiles entire source                      & Execute single commands         \\ \hline 
    Most errors are found while compilation     & Error found at run-time\\ \hline 
    Program run in low level language           & Program runs in virtual environment      \\ \hline    
    \caption{Differences between compilers and interpreters in respect to the goals of \gls{gamble}}\label{tbl:compint}
\end{table}
\vspace{-20pt}


\section{Target language}
Something about why we chose openCL C. 
 evt. why not cuda.

 lidt om hvordan openCL c fungerer (kernels og hetrogen)