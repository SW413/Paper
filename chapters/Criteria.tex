%\gls{gamble} is used to refer to our languange   
\chapter{Language Criteria} % (fold)
\label{cha:language_criteria}
When evaluating programming languages one must first agree on the goals of a given language's characteristics.
Robert W. Sebesta has given some criteria which can be evaluated from, we will use these criteria to design our language and to state which criteria are important to the language we are designing.\citep{Sebesta}
The criteria can be seen on \myref{tbl:concepts}.
First a brief description of the criteria will be given, followed by how we rate these criteria for our language according to the information found in \myref{cha:introduction}.
\begin{table}[h]
	\centering
	\colorlet{shadecolor}{gray!40}
    \rowcolors{1}{white}{shadecolor}
	\begin{tabular}{|l|c|c|c|}
	\hline
	\textbf{Language Concepts} & \textbf{Readability}  & \textbf{Writability} & \textbf{Reliability}   \\ \hline
	Simplicity                 & x      		       & x             		  & x           		   \\ \hline
	Orthogonality              & x 				       & x             		  & x           		   \\ \hline
	Data types                 & x 				       & x             		  & x           		   \\ \hline
	Syntax design              & x 				       & x             		  & x           		   \\ \hline
	Support for abstraction    &                       & x             		  & x           		   \\ \hline
	Expressivity               &                       & x             		  & x           		   \\ \hline
	Type checking              &                       &               		  & x           		   \\ \hline
	Exception handling         &                       &               		  & x           		   \\ \hline
	Restricted aliasing        &                       &               		  & x           		   \\ \hline
	\end{tabular}
	\caption{Language evaluation criteria and the characteristics that affect them.}\label{tbl:concepts}
\end{table}

We want it to be easy for programmers and mathematicians alike to write and read programs in our language.
The focus should be on calculating the problem at hand, without having to conform to low level language specifics, controlling memory etc.
Our language must adhere to the concept of simplicity, which means that the number of basic constructs is kept low. 
This results in high levels of both readability and writability as can be seen on \myref{tbl:concepts}.
The language supports abstraction over the programmer's hardware, by not having to specify where the computations must be run, or having to allocate memory.

To get an idea of what each criterion expressed, we will in the following text, describe our thoughts and goals in respect to the several of the  concepts in \myref{tbl:concepts} deemed important to \gls{gamble}.
Simplicity is important in the language of \gls{gamble}.
It is so because of the goals stated in \myref{sec:problem} and based on the other languages examined in \myref{sec:state_of_the_art}, simplicity is important when designing a language where the focus lies in writing code fast and easy, which can take advantage of the \acrshort{gpu}.
It is so because this goal does not require a large language.
Syntax design is also deemed an important criterion, it is so because the language should be familiar and easy to learn for beth experienced but especially newer programmers.
Data types is another important criterion for \gls{gamble}, as stated in \myref{sec:problem} the language should be able to handle a matrix data structure and furthermore is at stated in the problem statement should the language utilise the \acrshort{gpu} without the programmer specifically stating it, in that respect a high level of abstraction is important too. 

Based on the earlier discussions, exception handling is seen as a way of making the language more complex, therefore a goal in \gls{gamble} is to catch errors like type mismatch on compile time, which strongly eliminates the need for exception handling.


\section{Success criteria}\label{sec:OurCriterias}
We deem our language and compiler to be a good solution for our problem statement if it upholds to the following criteria:

\begin{itemize}
	\item Calculating matrix and vector calculations on the GPU, without the programmer specifically specifying it to do so.
	\item Can implicitly calculate matrix operations on the \acrshort{gpu} faster than regular C being run on the \acrshort{cpu}, when sufficient large matrices are provided.
	\item Expresses a high level of read- and writability, by adhering to the concept of simplicity and having a familiar syntax design.
	\item Provides reliability, with type and scope checking.
	\item Gives descriptive and fulfilling error messages, for easier debugging, resulting in better writability.
\end{itemize}

A more precise description of \gls{gamble}'s design principles abiding to Sebesta's criteria will be presented and argued for in the following chapter.
% chapter language_criteria (end)                   