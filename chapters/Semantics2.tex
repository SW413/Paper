\chapter{Semantics}
In this section the semantics for \gls{gamble} will be shown and explained.
Operational semantics is necessary to provide detailed information about the mathematical operations behind the language.
There exists two ways of showing operational semantics for a language.
Natural semantics (big-step semantics) and structural operational semantics (small-step semantics), it is decided to use big-step semantics.
The semantics is organised as follow:

\begin{align*}
	\textbf{a} \in  &Arithmetic\quad expression\\
	\textbf{v} \in  &Variable\\
	\textbf{m} \in  &Matrix\\
	\textbf{vec} \in  &Vector\\
	\textbf{type} \in &Int | Float | Bool
	\textbf{S}	\in &Statement
	\textbf{mtype} \in &Int | Float
	\textbf{Fd}		\in &Function Declaration
\end{align*}

\textbf{v} is meant to store the result, this can be both a value, a vector or a matrix.
Every big-step semantic showing operations on a matrix, is also possible to perform on a vector, which evaluates to a matrix with one column.
Every big-step semantic showing operations on a vector can not be used on a matrix, since it requires the dimensions to fit those of a matrix with one columns.

The variable environment(${ e }_{ v }$) is a function that for each variable defines to which storage location it is bound to. A variable environment corresponds to a table containing symbols which are in that environment.
The store ($st$) is a function that for each storage location in the computer tells us which value is stored in the location. A store corresponds to a complete description of the contents of the memory.\citep{EnvSt_Semantics}

A few of the big-step semantics for \gls{gamble} is included in this chapter, while the rest of the semantics for the rest of the \gls{gamble} is seen in \myref{app:semantics}.
A few of the most important and interesting operational semantics which will be explained are, variable declaration, assignment, matrix declaration, matrix multiplication and scalar product.

\subsection*{Declaring Variables}
Any non-empty variable declaration will modify the variable environment since the new variables will be bound to the new location.
A variable declaration will also modify the store, since the new location will be initialised to contain the initial values of the new variables.

[${VariableDeclaration}_{BSS}$]
\begin{equation}
	\frac { <S,{ e }_{ v }^{ `` },st[\iota \mapsto v]>{ \rightarrow  }_{ S }<{ e }_{ v }^{ ` },{ st }^{ ` }> }{ <type\quad x=a;S,{ e }_{ v }st>{ \rightarrow  }_{ S }<{ e }_{ v }^{ ` },{ st }^{ ` }> } 
\end{equation}
\begin{align*}
	where\quad &{ e }_{ v },st\vdash a{ \rightarrow  }_{ A }v \\
	&\iota {= e }_{ v }(next) \\
	&{ e }_{ v }^{ `` }={ e }_{ v }[x\mapsto \iota ][next\mapsto new \iota ]
\end{align*}

\section*{Statements}
The effect of a statement is that the store may change, since a statement may modify the values of variables involved through assignments.
A statement should not modify the variable environment.
We define a BS-semantic for statements (except procedure calls).

[${Assignment}_{BSS}$]
\begin{align*}
	&{ e }_{ f },{ e }_{ v }\vdash <x=a,st>\rightarrow st[\iota \mapsto v]\\
	where\quad &{ e }_{ v },st\vdash a{ \rightarrow  }_{ A }v\quad and\quad  { e }_{ v }(x)=\iota 
\end{align*}

[${While\top}_{BSS}$]
\begin{align*}
	&\frac { { e }_{ f },{ e }_{ v }\vdash <{ S }_{ 1 },st>\rightarrow { st }^{ `` },\quad { e }_{ v }\vdash <while\quad (b)\quad \{S\},{ st }^{ `` }>\rightarrow { st }^{ ` } }{ { e }_{ v }\vdash <while\quad (b)\quad \{S\},st>\rightarrow { st }^{ ` } }\\
	&if\quad { e }_{ v },st\vdash b{ \rightarrow  }_{ B }\top 
\end{align*}


[${While\bot}_{BSS}$]
\begin{align*}
	&{ e }_{ f },{ e }_{ v }\vdash <while\quad (b)\quad \{S\},\quad st>\rightarrow st\\
	&if\quad { e }_{ v },st\vdash b{ \rightarrow  }_{ B }\bot
\end{align*}

\section*{Matrices and Vectors}
A matrix is referred to as seen below, the superscript refers to name of the matrix, while the subscript is the entry in the matrix.
\begin{align*}
	M^{n}= \begin{bmatrix} { { m }_{ 1,1 }^{ n } } & { { m }_{ 1,2 }^{ n } } & \dots  & { m }_{ 1,k }^{ n } \\
{ { m }_{ 2,1 }^{ n } }  &  { { m }_{ 2,2 }^{ n } } & \dots & { m }_{ 2,k }^{ n }
\\ \vdots  & \vdots & \ddots  & \vdots \\
 { m }_{ j,1 }^{ n } & { m }_{ j,2 }^{ n } & \dots & { m }_{ j,k }^{ n } \end{bmatrix}
\end{align*}

[${MatrixDeclaration}_{BSS}$]
\begin{equation}
	\frac { { e }_{ f },{ e }_{ v },st\vdash j{ \rightarrow  }_{ A }{ v }_{ 1 },\quad { e }_{ f },{ e }_{ v },st\vdash k{ \rightarrow  }_{ A }{ v }_{ 2 } }{ { M }^{ 1 }\quad =\quad matrix<int|float>[{ v }_{ 1 },{ v }_{ 2 }]\rightarrow { st }^{ ` },{ e }_{ v } } ,{ M }^{ 1 }[{ v }_{ 1 },{ v }_{ 2 }]=\begin{bmatrix} { { 0 }_{ 1,1 } } & 0_{ 1,2 } & \dots  & { 0 }_{ { 1,{ v }_{ 2 } } } \\ 0_{ 2,1 } & 0_{ 2,2 } & \dots  & 0_{ 2,v_{ 2 } } \\ \vdots  & \vdots  & \ddots  & \vdots  \\ { 0 }_{ { v }_{ 1 },1 } & 0_{ v_{ 1 },2 } & \dots  & { 0 }_{ { v }_{ 1 },{ v }_{ 2 } } \end{bmatrix}
\end{equation}




[${MatrixMultiplication}_{BSS}$]

Matrix multiplication takes two matrices of the same type, where the size of $m^{1}$ and $m^{2}$satisfy $m^{1}[j,k]*m^{2}[k,l]$.
The operation returns a matrix of with $j$ columns and $l$ rows based on the rules in \myref{hypeproduct}.


\begin{minipage}{1.0\textwidth}
\begin{equation}
\begin{aligned}
	\frac { { e }_{ v },st\vdash { M }^{ 1 }{ \rightarrow  }_{ A }{ v }_{ 1 }\quad { e }_{ v },st\vdash { M }^{ 2 }{ \rightarrow  }_{ A }{ v }_{ 2 } }{ { e }_{ v },st\vdash { M }^{ 1 }*{ M }^{ 2 }{ \rightarrow  }_{ A }{ v } } ,\begin{matrix} { v }_{ 1 }=matrix<int|float>[j,k] \\ { v }_{ 2 }=matrix<int|float>[k,l] \\
	 v=\begin{bmatrix} { { m }_{ 1,1 }^{ 3 } } & { { m }_{ 1,2 }^{ 3 } } & \dots  & { m }_{ 1,l }^{ 3 }\\
{ { m }_{ 2,1 }^{ 3 } } &   +{ { m }_{ 2,2 }^{ 3 } } & \dots & { { m }_{ 2,l }^{ 3 } }
\\ \vdots  & \vdots & \ddots  & \vdots  \\
 { m }_{ j,1 }^{ 3 }& { { m }_{ j,2 }^{ 3 } } & \dots & { m }_{ j,l }^{ 3 } \end{bmatrix} where
\\ \end{matrix}
\end{aligned}
\end{equation}
\begin{equation*}
\begin{aligned}
{ { m }_{ 1,1 }^{ 3 } }  &= { { m }_{ 1,1 }^{ 1 } } *{ { m }_{ 1,1 }^{ 2 } }+ { { m }_{ 1,2 }^{ 1 } } *{ { m }_{ 2,1 }^{ 2 } }+\dots+ { { m }_{ 1,k }^{ 1 } } *{ { m }_{ k,1 }^{ 2 } }\\
{ { m }_{ 1,2 }^{ 3 } }  &= { { m }_{ 1,1 }^{ 1 } } *{ { m }_{ 1,2 }^{ 2 } }+ { { m }_{ 1,2 }^{ 1 } } *{ { m }_{ 2,2 }^{ 2 } }+\dots+ { { m }_{ 1,k }^{ 1 } } *{ { m }_{ k,2 }^{ 2 } }\\
\vdots\\
{ { m }_{ 1,l }^{ 3 } }  &= { { m }_{ 1,1 }^{ 1 } } *{ { m }_{ 1,l }^{ 2 } }+ { { m }_{ 1,2 }^{ 1 } } *{ { m }_{ 2,k }^{ 2 } }+\dots+ { { m }_{ 1,j }^{ 1 } } *{ { m }_{ j,l }^{ 2 } }\\
{ { m }_{ 2,1 }^{ 3 } }  &= { { m }_{ 2,1 }^{ 1 } } *{ { m }_{ 1,1 }^{ 2 } }+ { { m }_{ 2,2 }^{ 1 } } *{ { m }_{ 2,1 }^{ 2 } }+\dots+ { { m }_{ 2,k }^{ 1 } } *{ { m }_{ k,1 }^{ 2 } }\\
{ { m }_{ 2,2 }^{ 3 } }  &= { { m }_{ 2,1 }^{ 1 } } *{ { m }_{ 1,l }^{ 2 } }+ { { m }_{ 2,2 }^{ 1 } } *{ { m }_{ 2,2 }^{ 2 } }+\dots+ { { m }_{ 2,k }^{ 1 } } *{ { m }_{ k,2 }^{ 2 } }\\
&\vdots\\
{ { m }_{ 2,l }^{ 3 } }  &= { { m }_{ 2,1 }^{ 1 } } *{ { m }_{ 1,l }^{ 2 } }+ { { m }_{ 2,2 }^{ 1 } } *{ { m }_{ 2,l }^{ 2 } }+\dots+ { { m }_{ 2,k }^{ 1 } } *{ { m }_{ k,l }^{ 2 } }\\
&\vdots\\
{ { m }_{ j,1 }^{ 3 } }  &= { { m }_{ j,1 }^{ 1 } } *{ { m }_{ 1,1 }^{ 2 } }+ { { m }_{ j,2 }^{ 1 } } *{ { m }_{ 2,1 }^{ 2 } }+\dots+ { { m }_{ j,k }^{ 1 } } *{ { m }_{ k,1 }^{ 2 } }\\
{ { m }_{ j,2 }^{ 3 } }  &= { { m }_{ j,1 }^{ 1 } } *{ { m }_{ 1,2 }^{ 2 } }+ { { m }_{ j,2 }^{ 1 } } *{ { m }_{ 2,2 }^{ 2 } }+\dots+ { { m }_{ j,k }^{ 1 } } *{ { m }_{ k,2 }^{ 2 } }\\
&\vdots\\
{ { m }_{ j,l }^{ 3 } }  &= { { m }_{ j,1 }^{ 1 } } *{ { m }_{ 1,l }^{ 2 } }+ { { m }_{ j,2 }^{ 1 } } *{ { m }_{ 2,l }^{ 2 } }+\dots+ { { m }_{ j,k }^{ 1 } } *{ { m }_{ k,l }^{ 2 } }
\end{aligned}
\end{equation*}
\end{minipage}




[${ScalarMultiplication}_{BSS}$]
A scalar multiplication is a simple operation on a matrix, where every element in the $j,k$ matrix is multiplayed with an Arithmetic expression $a$. 

\begin{equation}
	\frac { { e }_{ v },st\vdash { M }^{ 1 }{ \rightarrow  }_{ A }{ v }_{ 1 }\quad { e }_{ v },st\vdash { a }_{ 1 }{ \rightarrow  }_{ A }{ v }_{ 2 } }{ { e }_{ v },st\vdash { M }^{ 1 }\ast { a }_{ 1 }{ \rightarrow  }_{ A }{ v } } ,\begin{matrix} { v }_{ 1 }=matrix<int|float>[j,k] \\  \\ v=\begin{bmatrix} { { a }_{ 1 }*m }_{ 1,1 }^{ 1 } & { { a }_{ 1 }*m }_{ 1,2 }^{ 1 } & \dots  & { { a }_{ 1 }*m }_{ 1,k }^{ 1 } \\ { { a }_{ 1 }*m }_{ 2,1 }^{ 1 } & { { a }_{ 1 }*m }_{ 2,2 }^{ 1 } & \dots  & { { a }_{ 1 }*m }_{ 2,k }^{ 1 } \\ \vdots  & \vdots  & \ddots  & \vdots  \\ { { a }_{ 1 }*m }_{ j,1 }^{ 1 } & { { a }_{ 1 }*m }_{ j,2 }^{ 1 } & \dots  & { { a }_{ 1 }*m }_{ j,k }^{ 1 } \end{bmatrix} \end{matrix}
\end{equation}