\chapter{Syntax}\label{chap:syntax}

To describe the formal syntax of \gls{gamble}, in order to parse the language, it is necessary to write a \acrfull{cfg}.
This chapter will explain what a \acrfull{cfg} is, and the problems of creating one.
After this explanation our own \acrfull{cfg}, will be presented.

%Our grammar is written using \acrfull{ebnf}, and uses \acrfull{regex} to define terminals such as numbers and names.
%Dette skal nok i afsnittet om vores eget sprog istedet?

\section{Context-Free Grammar}
A \acrshort{cfg} is an area of formal languages which are useful for specifying syntax. 
A \acrshort{cfg} generates a context-free language. 
A \acrshort{cfg} consists of one or several production rules.
On the left side of a production rule is a nonterminal and on the right side are terminals and/or nonterminals, additionally there is a start symbol. % OG/ELLER HYPE!!!
An example of this written in \myref{lst:cfglst1}, where a definition for multiplication and addition with parenthesis is shown.
The grammar is structured such that the multiplication operator have a higher precedence than the addition operator.

\begin{lstlisting}[caption={An example of a \acrshort{cfg} written in \acrshort{ebnf}, with \acrshort{regex} for defining numbers. },frame=tlrb,label={lst:cfglst1},numbers=none]
expression = term | expression "+" term;
term       = factor | term "*" factor;
factor     = constant | "(" expression ")";
constant   = [0-9]+
\end{lstlisting}

It is possible to generate a parse tree for a string which follows the grammar. 
If there exists two or more trees for any given string then the grammar is ambiguous. 
Having an ambiguous grammar can be a problem when parsing.   

\subsection{The dangling else problem}

A common mistake leading to ambiguous grammar is the dangling else problem. \citep{danglingelse}
In many programming languages is it possible to have an if statement and an else statement, and inside the body of these also having if- and else statements. 
A \acrshort{cfg} describing it is shown in \myref{lst:danglingelseex1}.

\begin{lstlisting}[caption={An example of a \acrshort{cfg} describing an if statement. \citep{danglingelse}},frame=tlrb,label={lst:danglingelseex1},numbers=none]
if statement =
    | if clause statement
    | if clause statement else statement

statement =
    | simple statement
    | if statement
    | loop statement
\end{lstlisting}

Given an input where the statement of an if statement contains an if statement this grammar is ambiguous.  
However it can be rewritten to allow if- and else statements in a grammar, however this will in almost all cases cause the size of the grammar to increase. 
A solution to this problem is to observe that there exists two kinds of if statements, open and closed if statements.
An open statement is one which the if statement is not paired with an else, and a closed one is any if statement paired with an else.
A simple statement is also a closed statement.\todo{Hvad er simple :P ?}
A grammar resolving the dangling else problem using this method is shown in \myref{lst:danglingelseex2}.

\begin{lstlisting}[caption={An example of a \acrshort{cfg} describing an if statement, that is not ambiguous. \citep{danglingelse}},frame=tlrb,label={lst:danglingelseex2},numbers=none]
statement =
    | open statement
    | closed statement

open statement =
    | if clause statement
    | if clause closed statement else open statement

closed statement =
    | simple statement
    | if clause closed statement else closed statement
\end{lstlisting}

Another way to resolve this issue is to force statement bodies of if statements, when followed by else statement, to be delimited by explicit blocks, such as \texttt{begin..end} used in Pascal or curly brackets (\texttt{\{...\}}) used in C and derivatives. 

\subsection{Derivations of parse trees}
A parser is a program which takes a string, and parses it into segments according to the rules specified in a grammar.
A more thorough explanation of parsers can be found in \myref{subsec:parser}
Two common strategies to generate a parse tree is leftmost derivation and rightmost derivation. 
A leftmost derivation applies the rules in the grammar by always applying a production rule to the leftmost non-terminal. 
This is the strategy used in a top-down parser, also known as an LL parser.
A rightmost derivation is the reverse, and what is used in a bottom-up parser, also known as a LR parser. 

%For example take the string: ``1 + 3 * 4'', following PEMDAS this results in 13, if one were to simply calculate from left to right the result would be 16.
%A leftmost derivation would be:
%
%\begin{table}
%    \centering
%    \colorlet{shadecolor}{gray!40}
%    \rowcolors{1}{white}{shadecolor}
%    \begin{tabular}{|l|l|l|}
%    \hline
%    \textbf{Step} & \textbf{Sentential Form}           & \textbf{Production Number} \\ \hline
%    1    & \textit{expression}                &                   \\ \hline
%    2    & \textit{expression} ``+'' \textit{term}       & 1                 \\ \hline
%    3    & \textit{term} ``+'' \textit{term}             & 1                 \\ \hline
%    4    & \textit{factor} ``+'' \textit{term}          & 2                 \\ \hline
%    5    & \textit{constant} ``+'' \textit{term}         & 3                 \\ \hline
%    6    & 1 ``+'' \textit{term}                & 4                 \\ \hline
%    7    & 1 ``+'' \textit{term} ``*'' \textit{factor}     & 2                 \\ \hline
%    8    & 1 ``+'' \textit{constant} ``*'' \textit{factor} & 3                 \\ \hline
%    9    & 1 ``+'' 3 ``*'' \textit{constant}      & 3                 \\ \hline
%    10   & 1 ``+'' 3 ``*'' 4             & 4                 \\ \hline    
%    \end{tabular}
%    \caption{Existing GPU supporting languages}
%    \label{tbl:sota}
%\end{table}

\section{Classes of CFGs}
As mentioned there are two common strategies for parsing, leftmost and rightmost. 
A parser also has a lookahead which is the maximum numbers of tokens which are needed to determine what rule should be applied, this is denoted in a parenthesis, so i.e. LL(1) means leftmost derivation using a lookahead of 1 token. 
A lookahead of k means that there is a constant lookahead of a maximum of k tokens for the given parser. 
There also exists the LL(*) which can dynamically change the number of tokens needed to parse by recognising if they follow a \acrshort{regex}.
Combining this with the fact that any LL grammar is a special case of a LR grammar and the different cases of lookahead constructs a hierarchy. 
This hierarchy is shown in \myref{image:hierarchyofgrammars}, LL(*) not included.
This figure is also contains the SLR which is more powerful than a LR(0) grammar, but less than a LR(1), and the LALR(1) which is more powerful than SLR and less than LR(1).
More powerful means that a grammar following it can produce more languages typically a superset of another. 
\begin{figure}[!ht]
\centering
 \includegraphics[width=0.5\textwidth]{figures/classesofgrammars.png} % trim=4.85cm 15cm 0.85cm 1cm
\caption{The hierarchy of \acrlong{cfg}. \citep{NvidiaCUDASeminar} \todo[inline]{TODO: Tikz figur med LL(*) indsat.}}\label{image:hierarchyofgrammars}
\vspace{-15pt}
\end{figure}

