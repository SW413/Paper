\chapter{Compiler Overview}\label{Chp:CompilerOverview}

In this chapter the different phases of the compiler and their goals and tasks will be presented, to give an overview for the chapters to come.

The compiler for \gls{gamble} is separated into three phases: syntax analysis, contextual analysis and code generation.
\myref{fig:phases} shows a state diagram of the phases of the compiler.
Syntax analysis and contextual analysis transforms the source code into an intermediate representation, and verifies the source code according to the specified syntax from the \acrshort{cfg}, and also for type and scope checking.
The choice of language for writing the compiler falls upon Java 1.8, which is also the language the project group have been taught for building compilers in the courses for the semester.
When making a compiler an object-oriented language simplifies many tasks, because of encapsulation, polymorphism and inheritance. 
The paradigm allows for many structural options, and reuse of code when inheriting, which would be impossible using another paradigm like imperative or functional programming.
Java also works across platforms, which is a useful feature.

In the syntax analysis phase the input source code is parsed and seperated into tokens according to the \acrshort{cfg}.
This is the scanner's job, the parser structures these tokens into a tree structure, which can be traversed in the order the source code is to be read.

When the source code has been parsed the tree is then simplified to make the traversal of the tree easier for the compiler, and while removing unnecessary information.

In the contextual analysis phase the tree is used to generate a table, containing all the variables and functions which is declared in the source code.
This is called a symbol table, and it is used to check if the variables and functions called and used in the source code are in scope, and also if they are upholding to the type rules of \gls{gamble}.
This phase results in telling the programmer if a mistake is found in the source code and where the mistake is located, and also what is wrong, but it also results in the tree now containing additional information on the type of the expressions' in the source code.

The last phase of the compiler is code generation and optimizing.
Optimization is any process which will make the generated program faster, e.g. adding constant numbers before the runtime etc, or changing the order of access to matrices to increase locality. 
In the code generation phase the output code is generated from all the information in gathered from the previoes phases of the compiler.

The target language of this compiler is OpenCL C.
To get this to run on any machine it is necessary to compile the output of the \gls{gamble} compiler, with a compiler for OpenCL C, since OpenCL C is not understandable by the computer.
By compiling the OpenCL C code, it is translated into the computer's instruction set and then further into machine code, which the computer understands.
This is an abstraction made by the project group to simplify the process of generating code, as targeting the \acrshort{gpu} using the specific instruction set, not only gives problems targetting more types of GPUs but will will take too much time for the project group to learn as it is very advanced.
Using OpenCL C, the target language is still low level compared to Java or C\#, and C has even been characterised as a portable assembly language.\citep{CPort}


\begin{figure}[ht!]
	\centering
	\begin{tikzpicture}[node distance = 2.5cm]
		\node (invi1) 		[invi,draw=none] {GAMBLE Source code};
		\node (syntax) 		[blockz, below=0.6cm of invi1] {Syntax Analysis};
		\node (contextual) 	[blockz, below of=syntax] {Contextual Analysis};
		\node (codegen) 	[blockz, below of=contextual] {Code Generation};
		\node (opencl) 		[invi,draw=none, below=0.6cm of codegen] {OpenCL C code};

		\node (error1) 		[cloud, right=1cm of syntax] {Error reports};
		\node (error2) 		[cloud, right=1cm of contextual] {Error reports};

		\node (invi2) 		[invi,draw=none, below=0.4cm of syntax] {};
		\node (ast) 		[invi,draw=none, right=-0.2cm of invi2] {Abstract Syntax Tree};

		\node (invi3) 		[invi,draw=none, below=0.4cm of contextual] {};
		\node (dast) 		[invi,draw=none, right=-0.5	cm of invi3] {Abstract Syntax Tree \& Symbol Table};

		\draw [arrow] (invi1) -- (syntax);
		\draw [arrow] (syntax) -- (contextual);
		\draw [arrow] (contextual) -- (codegen);
		\draw [arrow] (codegen) -- (opencl);
		
		\draw [arrow] (syntax) -- (error1);
		\draw [arrow] (contextual) -- (error2);
	\end{tikzpicture}
	\caption{The phases of the compiler.}\label{fig:phases}
\end{figure}
\todo{Change into a state Diagram as we were taught in OOA\&D}
\clearpage

\section{Source Code as Trees}\label{SourceCodeAsTrees}
The source code of a program is parsed by the compiler, but should also saved in some way so it is possible to manipulate the source code, and performing the phases of the compiler.
This is often done by using a tree, as is it in the \gls{gamble} compiler.

The tree structure is useful for this purpose because every node of the tree can contains information, and have children which then makes it possible to express the productions of a grammar by following a path on the tree from the root to a leaf.
A parse tree separates the source code into different productions of the grammar it represents, but also contains all of the syntax from the grammar, such as parenthesis.
The tree structure also makes it possible to traverse the tree in the same path as the source code is written, which means that the tree is able to express the structure of the source code as well as the statements which are found in the program.
An example of parse tree from the declaration \texttt{int a = 5;} is shown in \myref{image:PST}

\begin{figure}
    \centering
    \includegraphics[width=0.5\linewidth]{figures/Trees/PST.PNG}
    \caption{A parse tree from the expression \texttt{int a = 5;} using \glspl{gamble} \acrshort{cfg}.} \label{image:PST}
\end{figure}

The following chapter will explain how the compiler creates a parser to produce these parse trees for the source code, and thus making it possible to use the trees in the compiler.


\chapter{Syntax Analysis}\label{sec:syntaxAnalysis}
Syntax analysis is the first phase in compiling a language.
In this phase it is checked whether the input adheres to the rules of the language.
These rules are defined in a languages' \acrshort{cfg}.
The \acrshort{cfg} of \gls{gamble} is further described in \myref{sec:cfg}.
This analysis can be split up into further sub phases, lexical analysis and parsing, these are described in this chapter.
\section{Design}
%Design
\subsection*{Scanner}
The first stage of syntax analysis is the scanner, also called the lexer which handles the lexical analysis.
The primary function of a scanner is to transform a sequence of characters into a sequence of tokens.
The scanner makes sure that the source code adheres to the grammar rules provided by the CFG.
An example of this, would be that you could use the notation .1 or 0.1 for a decimal number, both being turned into valid tokens by the scanner.
The scanner provided by \acrshort{antlr} groups related tokens into token types such as INT, ID and FLOAT.
In \acrshort{antlr} a token contains at least two pieces of information, the token type and the matched text for the token.

Some examples of our lexical rules for \gls{gamble} can be seen on \myref{lst:token}.
The definition of an integer number on line 3 states that an integer is either a zero or an optional negative sign followed by a single digit from one to nine followed by zero or more numbers from zero to nine.
It is necessary to clearly define tokens for the lexer to read in order to read source code correctly. \citep{Crafting_book}

\begin{lstlisting}[caption=Example of our lexer rules for \acrshort{antlr}4,frame=tlrb,label={lst:token}]
// Integers
INT: 'int' | 'int16' | 'int32' | 'int64' ; // Integers
INTNUM: '0' | SIGN? [1-9][0-9]* ;

// Matrices and vectors
MATRIX: 'matrix' ;
VECTOR: 'vector' ; 

// Whitespace and comments
WS: [ \t ]+ -> skip;
NL: [ \r \n | \n ] -> skip;

COMMENT
    :   '/*' .*? '*/' -> skip
    ;

LINE_COMMENT
    :   '//' ~[\r\n]* -> skip
    ;
\end{lstlisting}
\subsection*{Parser}\label{subsec:parser}
The parser is based on the \acrfull{cfg} of \gls{gamble} written in \acrfull{ebnf}, whose alphabet consists of tokens produced by the scanner.
The parser reads tokens and groups them into phrases according to the \acrshort{cfg}.
The parser verifies that the syntax is correct and upholds to the \acrshort{cfg}, and if a syntax error is found it provides a corresponding error message. \citep{Crafting_book}
By using a parser generator like \acrshort{antlr} or SableCC, handling of syntactic errors and repairs can be done automatically.
A parser can also be written manually but doing so can result in syntactic errors that can prove difficult to find without a tool.
Writing a parser by hand can also take a lot of time, and it can be difficult to go back and change or add new productions to the syntax, which is something the project group will want to do due to the iterative development.
There are many parser generators which can be used like: SableCC, JavaCC, JFlex and many others, but we have chosen to use \acrshort{antlr}.
\acrshort{antlr} has been chosen due to their special use of the ALL(*) grammar, which poses many opportunities for the grammar, and also makes the \acrshort{cfg} simpler to write.
\acrshort{antlr} generates a parser which produces a parse tree that contains information about how the parser have grouped the tokens into more abstract language definitions, such as expressions and statements.

There are different kind of parsers, most common are bottom-up and top-down parsers.
\acrshort{antlr} makes a top-down parser, more specific a recursive descent parser.
A recursive descent parser is a subtype of top-down parser build from a set of mutually recursive procedures where each such procedure implements one of the productions of the grammar.
The structure of the resulting program closely mirrors the grammar it recognizes. \citep{Recursive_programming}
Recursive-descent parsers are a collection of recursive methods, one per rule of the \acrshort{cfg}.
Such a method for an assignment rule may look as shown in \myref{lst:rdpmethod}, where the rule is \texttt{assignment : ID = expr ;}.
So the method expects an ID to be the first token from the tokenstream, then an assignment operator followed by an expression and a semicolon.
Here the expression is a rule itself, and is therefore called on the expected expression.
An error should be returned if anything is not what was expected by the \texttt{match()} call.
\begin{lstlisting}[caption=Example a recursive descent parser method,frame=tlrb,label={lst:rdpmethod}]
// assign : ID ``='' expr ``;'' ;
void assign() { // method generated from rule assign
match(ID); // compare ID to current input symbol then consume
match('=');
expr(); // match an expression by calling expr()
match(';');
}
\end{lstlisting}

%The second stage of the parser is the actual parser.
%The parser is fed a stream of tokens to recognise a sentence structure and in turn outputs the structure to a parse tree.
%The parse tree records how the parser recognises the structure of the input and its components.
%The parse tree that \acrshort{antlr} provides contains information about how the parser have grouped the tokens into more abstract languange definitions such as expressions and statements.
%Where previous versions of \acrshort{antlr} have also implemented the AST, it is not contained in \acrshort{antlr} V4 instead the parse tree provided by \acrshort{antlr} have been used to generate an AST this is discussed in \myref{sec:AST}.
%This tree is a trimmed version of the parse tree, where the less informative data have been removed, this makes it easier to read, and thus easier to use throughout development of the rest of the compiler.

%2nd stage is the actual parser, feeds of tokens to recognize sentence structure
%Parse tree records how the parser recognized structure of input and its component phrases
%Trees provide an easy to walk data structure that will be helpful for the rest of the compiler
%2.2 Implementing Parser - Recursive descent
%Recursive-descent parsers are really just a collection of recursive methods, one per rule.
%Such a rule may look similar to this
%// assign : ID ``='' expr ``;'' ;
%void assign() { // method generated from rule assign
%match(ID); // compare ID to current input symbol then consume
%match('=');
%expr(); // match an expression by calling expr()
%match(';');
%}
%Descent refers to the fact we start from the root and go down to the leaves(tokens)
%Reursive descent is just one form of top-down parsers.					NOTE topdown/bottom up parsing
%The call graph traaced out by invoking methods, mirrors the interior parse tree nodes
%To Build a parse tree manually one would insert ``add new subroot note' operations at the start of each rule, and a ``add new leaf node'' operation to match()
%The assign method checks if all necessary tokens are present and in the right order. When the parser enters assign it doesnt have to choose between more than one alternative. An alternative is one of the choices on the right side of a rule def. A parsing method for such rule would be a switch which looks for what token is present.
% This is called a parising decision or prediction by examining next token
%This is where lookahead comes into play , the lookahead token is the next input token, this can be any token the parser "sniffs" before consuming
%This is one of the places where \acrshort{antlr} is an especially handy tool to use, because \acrshort{antlr} allows for more lookahead than other parser generators.
%Most parsers use a lookahead of one which LL(1) or LR(1), \acrshort{antlr} tones the lookahead up and down depending on what token stream it is trying to decode, as such the \acrshort{antlr} has a lookahead of LL(*)
%\acrshort{antlr} Solves simple ambiguity simply by using the first mentioned rule.
%AST only useful, Parse all artifacts(space, brackets and so on)


\subsection*{Abstract Syntax Tree}
\info[inline]{to be written}

%Implementation
\section{Implementation}

The syntax analysis is implemented through the tool ANTLR (ANother Tool for Languange Recognition), this tool provides some different advantages over other available tools.
ANTLR is build up around a parser technique called LL(*).
LL(*) uses an algorithm to have a varying lookahead when needed.
The LL(*) parser technique allows a bigger class of \acrshort{cfg} than ordinary parser techniques like LL(k).
In the most recent version of ANTLR as time of this publication, ANTLR 4, the underlying algorithm have been extended to a parser technique called Adaptive LL(*) (ALL(*)).
The most important feature of ALL(*) is that it moves grammar analysis to parse time. 
This lets the parse algorithm accept any non-left-recursive productions.
The ALL(*) approach accepts a broader class of grammars than most other parsing methods, one way this is done is to rule out ambiguity by using a rule of precedence.
If a grammar is ambiguous the ALL(*) approach will take the first  available rule in the \acrshort{cfg} and apply it.
This allows for a less strict \acr{cfg} and while most grammars could be rewritten to be unambiguous without applying the precedence rule, this allows the designers of languages grammars to edit existing grammars without rewriting as much a could be required if the precedence rule was not available.	
The idea with the ALL(*) algorithm is that the grammar is analysed dynamically at runtime rather than statically, before executed by the generated parser.
This allows ANTLR access to input sequences while reading through the grammar, meaning not all possible inputs must be considered.
Due to this dynamic analysis ANTLR 4 is able to handle some ambiguous constructs and reduce-reduce conflicts.
As mentioned this allows ANTLR to take care of left-recursion if such is present in the grammar by rewriting it, as such would be the case in \myref{lst:amb}.

\begin{lstlisting}[caption=An ambiguous rule for expr,frame=tlrb,label={lst:amb}]
expr : expr '*' expr 	// match expressions with * operator
     | expr '+' expr 	// match expressions with + operator
     | INT 		// matches simple integer
     ;
\end{lstlisting}

While it may not be obvious from \myref{lst:amb} but this CFG also implements ANTLRs way of representing operator precedence by simply obeying the first alternative in the rule set, as such the multiplication operator (``*'') will have the higher precedence.
The ALL(*) algorithm also means that one can completely disregard lookahead and it will still be able to parse, although one should keep in mind that having more lookahead than necessary will slow down the process.
ANTLR also implements rule element labels in its \gls{cfg} which means one can apply label rules to a construct in a grammar, this allows for conditional steps in the grammar based on the source code being parsed.
Furthermore ANTLR can set up an interface and base implementation of the visitor pattern for the parse tree on a given grammar by running ANTLR with the \texttt{--visitor} flag.\citep{ALLSTAR, LLSTAR, ANTLR4_Book}



\subsubsection*{Traversal of Trees}
When working with trees, traversing those trees is an important part of the process.
For this task different approaches can be taken, a common way is to implement a design pattern, the visitor pattern is particularly popular for tree traversal.
Alternatively one can implement the composite pattern or choose to implement no pattern at all, but simply create a case analysis for each object.
The use of a design pattern is not a requirement for the creation of a compiler.

Design patterns provide solution templates for software problems, each pattern providing its own benefits.
In OOP design patterns are typically aimed at helping object generation and interaction between these objects.
However the most important thing to keep in mind when using a design pattern, is not its exact implementation of classes and methods, but the concept the pattern describes.

The two aforementioned patterns are classified under two different branches of patterns.
The composite pattern is a structural pattern where the visitor pattern is a behavioural one.
A structural pattern provides a way of defining the relations between objects, the composite pattern is used to create a hierarchical recursive tree structure of related objects that may be accessed in a standardised manner.
A behavioural pattern is instead used to define how the objects communicate, the visitor pattern is used to separate a set of structured classes from any functionality that should be performed upon them.
For the compiler the visitor pattern have been implemented for the traversal of trees as such the pattern is described further in \myref{subs:visit}.  

\subsection*{Visitor Pattern}\label{subs:visit}
As mentioned the visitor pattern is but one way of traversing a tree.
The visitor pattern is used not only to traverse the parse tree provided by ANTLR.
The visitor pattern is implemented throughout the compiler, to both create the AST from the parse tree, for the pretty print functionality as well as filling the symbol table.
As such the visitor pattern defines the structure of the compiler, and thus understanding what is gained from using this pattern is important.
The visitor pattern is a Gang of Four, authors of ``Design Patterns: Elements of Reusable Object-Oriented Software'', design pattern.
Its description says ``The visitor pattern is a design pattern that separates a set of structured data from the functionality that may be performed upon it.''. \citep{GOF}

The pattern is a behavioural pattern i.e. it defines how communication between classes and entities are handled.
In the tree walk for the ANTLR generated parse tree, the visitor should convert the entirety into a new AST.
This entails that each different node in the parse tree must be visited and create an equivelent node for the AST.

Through use of the visitor pattern the functionality is seperated from the classes they are performed upon.
Instead the functionality is on a interface that each visitor implements.
The classes have an accept method that allows them to call the visitor in question with itself as an argument.
This allows the ability of adding new operations without changing the original data structure, an invaluable feature when doing iterative development.
\myref{image:visitor} shows a UML diagram of the visitor pattern.
This diagram is from a C\# representation, and while the idea is the same the exact implementation is not identical to the one used in the compiler for GAMBLE.
The most important things to take note of are the classes ``ConcreteElement'' and ``ConcreteVisitor''.
The ``ConcreteElement'' represent the different kinds of nodes in a given tree.
The ``ConcreteVisitor'' represent the different kind of visitors implemented, this being one for AST, parse tree, symbol table, code generation and pretty print, all these implements an interface that contains visit methods for each ``ConcreteElement''.

\begin{figure}[h!]
\centering
 \includegraphics[width=1\textwidth]{figures/VisitorPattern.png} % trim=4.85cm 15cm 0.85cm 1cm
\caption{A UML diagram for the implementation of the visitor pattern in a C\# environment}\label{image:visitor}
\vspace{-15pt}
\end{figure}



\todo{
Gang of four beskriver også denne benefit - men er ikke helt sikker på hvad den benefit rent faktisk betyder for os?
Another benefit is that a single visitor object is used to visit all classes.
This visitor can maintain state between calls to individual data objects. <----- dis is what im not sure of.
}

\subsection{Creating the \acrshort{ast}}\label{CreatingAst}

The goal for the \acrshort{ast} was to decrease the information in the tree, while making fields on the classes for easier access to the information.
Another way could be having all the nodes connected as children on the tree, but instead of running through the trees children, and looking for specific children, the compiler instead directly access different fields on the nodes instead.
All the nodes of the \acrshort{ast} have been designed with this in mind.
For example on figure \myref{image:ASTDecl} a class structure can be seen, which consists of all the classes needed to express a declaration on the tree.
To use a previous example a declaration could be \texttt{int a = 5;}.

\begin{figure}[!ht]
\centering
 \includegraphics[width=1\textwidth]{figures/ClassDiagrams/ASTDeclarationNodeMoreInfo.pdf} % trim=4.85cm 15cm 0.85cm 1cm
\caption{A UML class diagram of the classes used for a DeclarationNode on the \acrshort{ast}.}\label{image:ASTDecl}
\vspace{-15pt}
\end{figure}

This class contains a lot of different information which is used depending on which class the instance of the \texttt{Variable} is connected to.
If \texttt{Variable} is connected to a VariableExpressionNode, the only fields used on the variable class is ValueType and Id, while the booleans IsFunction and IsComplex are set to false.
But in the example \texttt{int a = 5;} the tree structure looks like the AST on \myref{image:AST}.
If \texttt{Variable} is used when a function is on the right side of an assignment or declaration, an example being \texttt{int a = foo(5);} the fields used on \texttt{Variable} are Id, ValueType, arguments, and the boolean IsFunction is instead set to true.
The printargument field is used when the a functioncall to \texttt{Print();}is made, and entrance, size and IsComplex is used when dealing with the complex types, vectors and matrices.
\texttt{TopNode} sets the structure of a \gls{gamble} program as described in \myref{subsec:Struc}. 
The full Classdiagram can be seen on \todo{Indsæt bilag med hele klassediagrammet brugt til AST.}



%Subphases
\chapter{Contextual Analysis}
Once the source code has passed through the syntax analysis, and thereby is correct in regard to the \acrshort{cfg}, it must be checked for contextual errors.
In this analysis phase semantic errors are the ones being checked for, this includes type errors such as type mismatch, e.g. adding a bool to a float, which may well be using the correct grammar, but is not a valid arithmetic expression.
The design and implementation of this analysis as well as how such errors are handled are described in this chapter.
\info[inline]{meta for this, phases are done in their own subsection files, we will talk about the following Filling of symbol Table -> Scope Checking -> Type Checking}
\subsection*{Symbol Table}
In a compiler it is useful to store information about the identifiers, variables and functions in a data structure. 
This information can be useful for scope checking and type checking.

There are two common ways of having a Symbol Table, either have one single table for every identifier or have one for each scope. 
If constrained by memory then having a single symbol table can be beneficial, however having multiple symbol tables can simply the code at a memory cost. 

In \gls{gamble} there exists scopes, and for each of these scopes there is a corresponding symbol table. 
Scopes inherit from each other so a scope can enclose another. 
The outermost scope, the global scope, is where functions are declared, every scope is either directly enclosed by this scope or recursively.
This means that every function can be called from anywhere within the source code. 

In \gls{gamble} the class SymbolTable represents the symbol table.
The core constituent of this class is the ArrayList of the Scope class, called allScopes, meaning that every scope is stored in this ArrayList.
Every scope contains Map of Symbols and strings as keys, and information about the scope such what scope it is enclosed by. 
The key represents the name of the symbol, and must be unique to the scope and not found in an enclosing scope, as this could cause an ambiguity to arise. 
A symbol is either a variable or a function, in the class Symbol its data type, name and scope is stored. 


\section{Scope Checking}
As a part of the contextual analysis the compiler must ensure that every reference is valid in their given scope.
The validity of such reference is specific to the rules of the language, this section describes how the compiler upholds the rules listed in \myref{subsec:Scope}.
\subsection*{Design}
In \myref{subsec:Scope} the scope of identifiers, variables and functions, are defined for \gls{gamble}.
A variable is in scope from its declaration until the end of the block it is declared in.
An inner scope inherits the identifiers declared in the outer scopes. 

During contextual analysis it is important to verify that each variable and function used are in scope and if not it should produce a descriptive error message.
The error message should indicate which identifier is not in scope, and the line this identifier is introduced on.

To check this all references to identifiers must be checked to see if they match an identifier in the symbol table of the current scope, and recursively the scope which encloses it. 
Furthermore it is important that any usage of an identifier only exists after its declaration.
In the compiler for \gls{gamble}, the symbol table is filled while the compiler is also performing the scope checking.
This is because it reduces the amount of traversals through the tree, scope checking and filling the symbol table is also of similar concept, e.g. a declaration creates an entry in the symbol table, while expressions simply use lookups in this table.

The scope checker can produce two errors: redeclaration errors and undeclared errors defined in \myref{subsec:Scope}.\todo{reference nødvendig igen? also: can produce. MP}
A redeclaration error is produced when an attempt to declare a variable while it is already declared in scope is made.\todo{klumset formulering. MP}
An undeclared error is produced when an attempt to use a variable which is not declared in the current scope or any enclosing scopes are made. 
Examples are shown in \myref{lst:scopeErrors}.

\begin{lstlisting}[caption=Examples of scope errors in \gls{gamble}, numbers=none,frame=tlrb,label={lst:scopeErrors}]
/* [...] */
int a = 1;
float a = 2.2;   /* Redeclaration error */
int a = 2;       /* Redeclaration error */ 

b = 2;           /* Undeclared error */   
int b = 0;
b = foo();       /* Redeclaration error and undeclared error */ 
/* [...] */
\end{lstlisting}
\todo{Evt lav en UML diagram der viser de klasser der bruges til at lave symbol tabellet, det gør i hvert fald det nemmere at forstå implementation. Men et diagram hører til i design afsnittet.}

\subsection*{Implementation}
\info[inline]{Todo: Listing or/eller figur i denne del.}
In the \gls{gamble} compiler, the class \texttt{SymbolTable} represents a collection of scopes.
The core constituent of this class is the ArrayList of the \texttt{Scope} class, called \texttt{allScopes}, meaning that every scope is stored in this ArrayList.
Every scope contains a hashed map with \texttt{Symbol}s as values and strings as keys.
Furthermore all scopes contain information about the particular scope such as enclosing scope and a unique id. 
The key in the hashed maps represents the name of the symbol, and must be unique to the scope and not found in an enclosing scope, as this could cause an ambiguity to arise.
This ambiguity is not allowed in \gls{gamble} and therefore throws a redeclaration error.
\todo{Meget forvirrende at holde styr på uden en figur tbh. - Søren}

To determine whether or not a declaration is a redeclaration, the compiler looks up the name of the variable in the \texttt{symbolMap} of the current scope.
If no entry is found, the search continues in the \texttt{symbolMap} of the enclosing scope and so on recursively.\todo{Må man ikke redeclare et variable i et indre scope ??, så hvis jeg bruger tmp ude i global scope, så må jeg ikke bruge den inde i en funktion ?? - Søren - Ikke ifølge hvad der står her - Marc}
The same lookup process is executed when a variable is used e.g in an expression or assignment.
Hereby all enclosing scopes are checked for the declaration of the variable; making redeclaration of variables in scope, and usage of undeclared variables impossible in \gls{gamble}, as defined in \myref{cha:language_design}.
A \texttt{symbol} is an encapsulation of a variable and relevant peripheral data.
The peripheral data consists of a boolean, used for checking if a declared variable is used, and an integer with the line number if the declaration of the variable.
The boolean describing if a declared variable is used, is set to true, if the previously described lookup process for a variable in use, finds a declared variable from the unique id (the name of the variable).
This boolean also makes it possible for the \gls{gamble} compiler to find unused variables and then prompt the user with a relevant warning.

In order for all of the above to be implemented an instance of the \texttt{SymbolTable} class is passed via the constructor, to a visitor which traverses the \acrshort{ast}.
This visitor, the \texttt{SymbolTableFillVisitor} then fills the referenced \texttt{symbolTable} with scopes and their symbols.\todo{Hedder den stadig SymbolTableFillVisitor?}
Every time the visitor meets the start of a new scope e.g. the block of statements within a loop construct.
A new instance of the \texttt{Scope} class is pushed to the \texttt{scopeStack}, hereby making it possible to fill the relevant scope when the block of statements is visited.
At the end of a scope the top element on the \texttt{scopeStack} is popped, and as a consequence of this the top of the \texttt{scopeStack} is back to the enclosing scope.

\section{Type Checking}
\subsection*{Type Checking}
The second important part of the contextual analysis phase for the compiler is the type checking which enforces the type system of \gls{gamble}.
As \gls{gamble} is statically typed it is necessary to check if all references to identifiers and constant values fit into the context they exist in. 
Since implicit conversion between floating point and integer types is not a part of gamble an error must be issued everywhere they are used wrongly. 
It is however possible to convert between integer and floating point types internally e.g. from int16 to int64. 

The symbol table is used as a reference for which type the variable is, and therefore the type checking happens after the symbol table have been filled and therefore after scope checking is completed. 
Type checking is done in many parts of the code, one or more times for each line is common. 
For every operator it must be checked if its types match and if it results in an assignment if that also matches.
Every function call must match the formal parameters of the function. 

The errors produced by the type checker is: ArgumentsError and TypeMismatchError.
An ArgumentsError indicates that the number of arguments does not match the ones given.
A TypeMismatchError is any error which is caused by a value or identifier not being compatible with the function parameters, operator used etc.
Examples are shown in \myref{lst:typeErrors}.

\begin{lstlisting}[caption=Examples of type errors in \gls{gamble},numbers=none,frame=tlrb,label={lst:typeErrors}]
/* [...] */
int a = 1 + 2;      /* Valid */
float b = 2.2 + 1;  /* TypeMismatchError */
float c = 2;        /* TypeMismatchError */ 

a = 2.2;            /* TypeMismatchError */
b = foo(1);         /* ArgumentsError (Takes more or fewer arguments) */ 
/* [...] */
\end{lstlisting}

This check is done as a part of decorating the AST and uses a visitor to traverse the AST. \todo{implementering? ...}

\section{Error Handling}
\subsection*{Design}\label{subsec:DesignErrorHandling}
The source code given to the \gls{gamble} compiler can contain errors and warnings, these are found in the various stages of the compiler.
Most central are scope and type related errors, these are found in the contextual analysis phase of the compiler. 
The following errors and warnings are reported by the \gls{gamble} compiler:
\info[inline]{Itemize yay or nay?}
\begin{itemize}
	\item Arguments error: The arguments of a function call doesn't match the ones used.
	\item Redeclaration error: A variable or function with the same name is already declared.
	\item Type mismatch error: The types used are incompatible with the operator. 
	\item Undeclared error: Attempted use of an undefined variable or function.
	\item Unused variable warning: A warning signaling that a variable or function is declared but never invoked.
\end{itemize} 
The errors in the compiler should give useful information about: Where the error is in the source code and what variable(s) or function(s) was wrongly used.
This include line numbers which then have to be carried over from the parse tree to the abstract syntax tree, this is expanded in \todo{ref til afsnit som ikke findes i denne branch}.

\subsection*{Implementation}\label{subsec:ImplementationErrorHandling}
In the \gls{gamble} compiler this is implemented by having a class \texttt{LanguageError} of which each specific error type inherits from.
The \texttt{LanguageError} superclass has information about what type the error, an enumeration which is either \texttt{Error} or \texttt{Warning}. 
This is so because a program which has warning but no errors, should still compile, however any program with errors should not. 
Compared to simply printing any error when it was discovered and stopping compilation, this model allows the compiler to go through the entire \acrshort{ast} and discover every error. 
It also unifies how errors are printed and shown to the user. \todo{Denne og forrige sætning er måske lidt svage? Skal der være mere ellere mindre om det. På plus siden er det jo en vurdering og det som viser forståelse osv.}
There is also an integer indicating which line number the error is contained on.
This is to enhance the error reporting process, making it easier for users to debug. 
\todo{Måske skriv noget om at dette kræver informationen er med i ASTet fra PTet ?}
As an example the class for \texttt{UnDeclaredError} is shown in \myref{lst:undeclarederrorclass}.
First there are variable which contains information needed to report the error to users. 
Then there is a constructor which simply assigns all values given to it to the new objects fields.
Lastly there is the \texttt{toString} method which is overridden from the implementation in the \texttt{LanguageError} class. 
An example of an \texttt{UnDeclaredError} is: \texttt{Error[line   42]-> Undeclared variable towel in scope Local}. %Sneaky H2G2 reference 

\begin{lstlisting}[caption=The UnDeclaredError class in the \gls{gamble} compiler,numbers=none,frame=tlrb,label={lst:undeclarederrorclass}]
public class UnDeclaredError extends LanguageError {
    private Variable unDeclaredVariable;
    private Scope scope;

    public UnDeclaredError(Variable unDeclaredVariable, Scope scope, int lineNum) {
        this.unDeclaredVariable = unDeclaredVariable;
        this.scope = scope;
        this.lineNum = lineNum;
        this.errorType = ErrorType.ERROR;
    }

    @Override
    public String toString() {
        String type = unDeclaredVariable.isFunction() ? "function " : "variable ";
        return super.toString() + String.format("Undeclared %6$s %1$s%4$s%3$s in scope %2$s%5$s%3$s",
                ANSI_RED, ANSI_BLUE, ANSI_RESET,
                unDeclaredVariable, scope, type);
    }
}
\end{lstlisting}

This error is found in the \texttt{CheckIfUndeclared} method in the \texttt{SymbolTableFillVisitor} shown in \myref{lst:CheckIfUndeclared}.
This method is called whenever a variable or function is invoked, and returns the appropriate information. \todo{Teknisk set returnerer den ikke informationen men sætter det gennem en reference, skal dette skrives eksplicit? Kunne ikke lige formulere det ordenligt.}
If the variable or function invoked is not declared then an error is added to a global list of errors. 
\begin{lstlisting}[caption=The CheckIfUndeclared method in the SymbolTableFillVisitor class in the \gls{gamble} compiler,numbers=none,frame=tlrb,label={lst:CheckIfUndeclared}]
public Void CheckIfUndeclared(Variable variable, StatementNode node) {
    Symbol tmpSymbol = symbolTable.currentScope().resolve(variable.getId());
    if (tmpSymbol != null) {
        /* Sets appropriate information to the 'variable' variable */
    } else {
        errors.add(new UnDeclaredError(variable, symbolTable.currentScope(), node.getLineNumber()));
        symbolTable.currentScope().define(variable);
    }

    return null;
}
\end{lstlisting}


%Phase3
\section{Code Generation}
\todo{Subphases of this is yet unknown, multipass and singlepass should be in this}
%Multipass
%Singlepass
\subsection*{GPU Usage}\label{GPUCode}\change[inline]{This should probably be in the design part of codegen?}
Since \gls{gamble} distances its programmers from directly controlling which computations are performed on the \acrshort{gpu}, determining what code to perform on the \acrshort{gpu} becomes a problem for the compiler to solve.
To make this decision the compiler must know what kind of code performs faster on a \acrshort{gpu} than a \acrshort{cpu}.
From \myref{sec:comparch} it is clear that for it to make sense moving any computation to the \acrshort{gpu} it must be of significant size to make up for the overhead of moving data, and be executable in parallel.
If a computation is reliant on the outcome of other computations, the Fibonacci function as an example, moving it to the \acrshort{gpu} would be a significant decrease in performance compared to on a \acrshort{cpu}.

Any code written in a recursive format will not be run on the \acrshort{cpu} furthermore due to the overhead in data transfer, only computations requiring a significant amount of operations to be performed should be executed on the \acrshort{gpu} as \myref{image:benchmark} shows.
Therefore statements which only contains simple data types, i.e. integers, floats and bools, are performed on the \acrshort{cpu}.
An example could be \texttt{value = value1 + value2}, where all types are integers.
%Therefore statements not containing complex data types, i.e. statements with no vector or matrix arithmetics, are also performed on the \acrshort{gpu}.

However statements that do include matrix or vector arithmetics will be performed on the \acrshort{gpu}.
An example could be matrix multiplication.
Now it is entirely possible to make a matrix multiplication of a 2x2 matrix, which would be so small that the overhead of data transfer is more expensive than simply computing on the \acrshort{cpu}. % however as mentioned in \myref{sec:phil} a computation containing that little data is not what the languange is designed to compute.
The language is meant for larger computations that can actually benefit from using the \acrshort{gpu}, therefore the size of a given matrix or vector is irrelevant when the compiler considers what processor to use. \improvement[inline]{Er vi enige om dette sidste? Jeg syndes lidt vi burde skrive at for simplicitet har vi valgt at alle udregninger på matricer og vectorer sker på gpuen, og at vi forstår dette ikke altid er et godt valg. - Troels}

