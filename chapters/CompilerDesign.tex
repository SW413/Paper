\chapter{Compiler Design}\todo{Evenetuelt et nyt navn hvis dette er mere teori end hvordan vi implementere vores compiler}
A compiler can be seperated into three phases; syntax analysis, contextual analysis and code generation.
The syntax analysis phase checks whether or not the source code adheres to the rules for the languange, such as statement constructs.
The contextual analysis phase checks whether or not the languange is used correctly, such as type checking.
The code generation translates the source code into the target code once the syntax and contextual analysis phases have accepted the source code.
This chapter examines each phase of the compiler and what these entail.
How each phase is constructet for GAMBLE will also be gone over in this chapter. %This may or may not change

\info[inline]{A few thing such as symbol table i am unaware where as to put, but possibly in the Metatext for compiler design, or in metatext for syntax analysis}
\improvement[inline]{Metatext}
%Antlr
\section{Implementation}

The syntax analysis is implemented through the tool ANTLR (ANother Tool for Languange Recognition), this tool provides some different advantages over other available tools.
ANTLR is build up around a parser technique called LL(*).
LL(*) uses an algorithm to have a varying lookahead when needed.
The LL(*) parser technique allows a bigger class of \acrshort{cfg} than ordinary parser techniques like LL(k).
In the most recent version of ANTLR as time of this publication, ANTLR 4, the underlying algorithm have been extended to a parser technique called Adaptive LL(*) (ALL(*)).
The most important feature of ALL(*) is that it moves grammar analysis to parse time. 
This lets the parse algorithm accept any non-left-recursive productions.
The ALL(*) approach accepts a broader class of grammars than most other parsing methods, one way this is done is to rule out ambiguity by using a rule of precedence.
If a grammar is ambiguous the ALL(*) approach will take the first  available rule in the \acrshort{cfg} and apply it.
This allows for a less strict \acr{cfg} and while most grammars could be rewritten to be unambiguous without applying the precedence rule, this allows the designers of languages grammars to edit existing grammars without rewriting as much a could be required if the precedence rule was not available.	
The idea with the ALL(*) algorithm is that the grammar is analysed dynamically at runtime rather than statically, before executed by the generated parser.
This allows ANTLR access to input sequences while reading through the grammar, meaning not all possible inputs must be considered.
Due to this dynamic analysis ANTLR 4 is able to handle some ambiguous constructs and reduce-reduce conflicts.
As mentioned this allows ANTLR to take care of left-recursion if such is present in the grammar by rewriting it, as such would be the case in \myref{lst:amb}.

\begin{lstlisting}[caption=An ambiguous rule for expr,frame=tlrb,label={lst:amb}]
expr : expr '*' expr 	// match expressions with * operator
     | expr '+' expr 	// match expressions with + operator
     | INT 		// matches simple integer
     ;
\end{lstlisting}

While it may not be obvious from \myref{lst:amb} but this CFG also implements ANTLRs way of representing operator precedence by simply obeying the first alternative in the rule set, as such the multiplication operator (``*'') will have the higher precedence.
The ALL(*) algorithm also means that one can completely disregard lookahead and it will still be able to parse, although one should keep in mind that having more lookahead than necessary will slow down the process.
ANTLR also implements rule element labels in its \gls{cfg} which means one can apply label rules to a construct in a grammar, this allows for conditional steps in the grammar based on the source code being parsed.
Furthermore ANTLR can set up an interface and base implementation of the visitor pattern for the parse tree on a given grammar by running ANTLR with the \texttt{--visitor} flag.\citep{ALLSTAR, LLSTAR, ANTLR4_Book}


\todo{skal flytte}
%Phase1
\section{Syntax Analysis}
\todo{meta for this, phases are done in their own subsection files}
%Subphases
\subsection*{Scanner}
The first stage of syntax analysis is the scanner, also called the lexer which handles the lexical analysis.
The primary function of a scanner is to transform a sequence of characters into a sequence of tokens.
The scanner makes sure that the source code adheres to the grammar rules provided by the CFG.
An example of this, would be that you could use the notation .1 or 0.1 for a decimal number, both being turned into valid tokens by the scanner.
The scanner provided by \acrshort{antlr} groups related tokens into token types such as INT, ID and FLOAT.
In \acrshort{antlr} a token contains at least two pieces of information, the token type and the matched text for the token.

Some examples of our lexical rules for \gls{gamble} can be seen on \myref{lst:token}.
The definition of an integer number on line 3 states that an integer is either a zero or an optional negative sign followed by a single digit from one to nine followed by zero or more numbers from zero to nine.
It is necessary to clearly define tokens for the lexer to read in order to read source code correctly. \citep{Crafting_book}

\begin{lstlisting}[caption=Example of our lexer rules for \acrshort{antlr}4,frame=tlrb,label={lst:token}]
// Integers
INT: 'int' | 'int16' | 'int32' | 'int64' ; // Integers
INTNUM: '0' | SIGN? [1-9][0-9]* ;

// Matrices and vectors
MATRIX: 'matrix' ;
VECTOR: 'vector' ; 

// Whitespace and comments
WS: [ \t ]+ -> skip;
NL: [ \r \n | \n ] -> skip;

COMMENT
    :   '/*' .*? '*/' -> skip
    ;

LINE_COMMENT
    :   '//' ~[\r\n]* -> skip
    ;
\end{lstlisting}
\subsection*{Parser}\label{subsec:parser}
The parser is based on the \acrfull{cfg} of \gls{gamble} written in \acrfull{ebnf}, whose alphabet consists of tokens produced by the scanner.
The parser reads tokens and groups them into phrases according to the \acrshort{cfg}.
The parser verifies that the syntax is correct and upholds to the \acrshort{cfg}, and if a syntax error is found it provides a corresponding error message. \citep{Crafting_book}
By using a parser generator like \acrshort{antlr} or SableCC, handling of syntactic errors and repairs can be done automatically.
A parser can also be written manually but doing so can result in syntactic errors that can prove difficult to find without a tool.
Writing a parser by hand can also take a lot of time, and it can be difficult to go back and change or add new productions to the syntax, which is something the project group will want to do due to the iterative development.
There are many parser generators which can be used like: SableCC, JavaCC, JFlex and many others, but we have chosen to use \acrshort{antlr}.
\acrshort{antlr} has been chosen due to their special use of the ALL(*) grammar, which poses many opportunities for the grammar, and also makes the \acrshort{cfg} simpler to write.
\acrshort{antlr} generates a parser which produces a parse tree that contains information about how the parser have grouped the tokens into more abstract language definitions, such as expressions and statements.

There are different kind of parsers, most common are bottom-up and top-down parsers.
\acrshort{antlr} makes a top-down parser, more specific a recursive descent parser.
A recursive descent parser is a subtype of top-down parser build from a set of mutually recursive procedures where each such procedure implements one of the productions of the grammar.
The structure of the resulting program closely mirrors the grammar it recognizes. \citep{Recursive_programming}
Recursive-descent parsers are a collection of recursive methods, one per rule of the \acrshort{cfg}.
Such a method for an assignment rule may look as shown in \myref{lst:rdpmethod}, where the rule is \texttt{assignment : ID = expr ;}.
So the method expects an ID to be the first token from the tokenstream, then an assignment operator followed by an expression and a semicolon.
Here the expression is a rule itself, and is therefore called on the expected expression.
An error should be returned if anything is not what was expected by the \texttt{match()} call.
\begin{lstlisting}[caption=Example a recursive descent parser method,frame=tlrb,label={lst:rdpmethod}]
// assign : ID ``='' expr ``;'' ;
void assign() { // method generated from rule assign
match(ID); // compare ID to current input symbol then consume
match('=');
expr(); // match an expression by calling expr()
match(';');
}
\end{lstlisting}

%The second stage of the parser is the actual parser.
%The parser is fed a stream of tokens to recognise a sentence structure and in turn outputs the structure to a parse tree.
%The parse tree records how the parser recognises the structure of the input and its components.
%The parse tree that \acrshort{antlr} provides contains information about how the parser have grouped the tokens into more abstract languange definitions such as expressions and statements.
%Where previous versions of \acrshort{antlr} have also implemented the AST, it is not contained in \acrshort{antlr} V4 instead the parse tree provided by \acrshort{antlr} have been used to generate an AST this is discussed in \myref{sec:AST}.
%This tree is a trimmed version of the parse tree, where the less informative data have been removed, this makes it easier to read, and thus easier to use throughout development of the rest of the compiler.

%2nd stage is the actual parser, feeds of tokens to recognize sentence structure
%Parse tree records how the parser recognized structure of input and its component phrases
%Trees provide an easy to walk data structure that will be helpful for the rest of the compiler
%2.2 Implementing Parser - Recursive descent
%Recursive-descent parsers are really just a collection of recursive methods, one per rule.
%Such a rule may look similar to this
%// assign : ID ``='' expr ``;'' ;
%void assign() { // method generated from rule assign
%match(ID); // compare ID to current input symbol then consume
%match('=');
%expr(); // match an expression by calling expr()
%match(';');
%}
%Descent refers to the fact we start from the root and go down to the leaves(tokens)
%Reursive descent is just one form of top-down parsers.					NOTE topdown/bottom up parsing
%The call graph traaced out by invoking methods, mirrors the interior parse tree nodes
%To Build a parse tree manually one would insert ``add new subroot note' operations at the start of each rule, and a ``add new leaf node'' operation to match()
%The assign method checks if all necessary tokens are present and in the right order. When the parser enters assign it doesnt have to choose between more than one alternative. An alternative is one of the choices on the right side of a rule def. A parsing method for such rule would be a switch which looks for what token is present.
% This is called a parising decision or prediction by examining next token
%This is where lookahead comes into play , the lookahead token is the next input token, this can be any token the parser "sniffs" before consuming
%This is one of the places where \acrshort{antlr} is an especially handy tool to use, because \acrshort{antlr} allows for more lookahead than other parser generators.
%Most parsers use a lookahead of one which LL(1) or LR(1), \acrshort{antlr} tones the lookahead up and down depending on what token stream it is trying to decode, as such the \acrshort{antlr} has a lookahead of LL(*)
%\acrshort{antlr} Solves simple ambiguity simply by using the first mentioned rule.
%AST only useful, Parse all artifacts(space, brackets and so on)


\subsection*{Abstract Syntax Tree}
\info[inline]{to be written}
\subsection*{Visitor Pattern}\label{subs:visit}
As mentioned the visitor pattern is but one way of traversing a tree.
The visitor pattern is used not only to traverse the parse tree provided by ANTLR.
The visitor pattern is implemented throughout the compiler, to both create the AST from the parse tree, for the pretty print functionality as well as filling the symbol table.
As such the visitor pattern defines the structure of the compiler, and thus understanding what is gained from using this pattern is important.
The visitor pattern is a Gang of Four, authors of ``Design Patterns: Elements of Reusable Object-Oriented Software'', design pattern.
Its description says ``The visitor pattern is a design pattern that separates a set of structured data from the functionality that may be performed upon it.''. \citep{GOF}

The pattern is a behavioural pattern i.e. it defines how communication between classes and entities are handled.
In the tree walk for the ANTLR generated parse tree, the visitor should convert the entirety into a new AST.
This entails that each different node in the parse tree must be visited and create an equivelent node for the AST.

Through use of the visitor pattern the functionality is seperated from the classes they are performed upon.
Instead the functionality is on a interface that each visitor implements.
The classes have an accept method that allows them to call the visitor in question with itself as an argument.
This allows the ability of adding new operations without changing the original data structure, an invaluable feature when doing iterative development.
\myref{image:visitor} shows a UML diagram of the visitor pattern.
This diagram is from a C\# representation, and while the idea is the same the exact implementation is not identical to the one used in the compiler for GAMBLE.
The most important things to take note of are the classes ``ConcreteElement'' and ``ConcreteVisitor''.
The ``ConcreteElement'' represent the different kinds of nodes in a given tree.
The ``ConcreteVisitor'' represent the different kind of visitors implemented, this being one for AST, parse tree, symbol table, code generation and pretty print, all these implements an interface that contains visit methods for each ``ConcreteElement''.

\begin{figure}[h!]
\centering
 \includegraphics[width=1\textwidth]{figures/VisitorPattern.png} % trim=4.85cm 15cm 0.85cm 1cm
\caption{A UML diagram for the implementation of the visitor pattern in a C\# environment}\label{image:visitor}
\vspace{-15pt}
\end{figure}



\todo{
Gang of four beskriver også denne benefit - men er ikke helt sikker på hvad den benefit rent faktisk betyder for os?
Another benefit is that a single visitor object is used to visit all classes.
This visitor can maintain state between calls to individual data objects. <----- dis is what im not sure of.
}

%Phase2
\info[inline]{meta for this, phases are done in their own subsection files, we will talk about the following Filling of symbol Table -> Scope Checking -> Type Checking}
%Subphases
\subsection*{Contextual Constraints}
\info[inline]{this includes cope rules, type rules and possibly semantics}
\subsection*{Symbol Table}
In a compiler it is useful to store information about the identifiers, variables and functions in a data structure. 
This information can be useful for scope checking and type checking.

There are two common ways of having a Symbol Table, either have one single table for every identifier or have one for each scope. 
If constrained by memory then having a single symbol table can be beneficial, however having multiple symbol tables can simply the code at a memory cost. 

In \gls{gamble} there exists scopes, and for each of these scopes there is a corresponding symbol table. 
Scopes inherit from each other so a scope can enclose another. 
The outermost scope, the global scope, is where functions are declared, every scope is either directly enclosed by this scope or recursively.
This means that every function can be called from anywhere within the source code. 

In \gls{gamble} the class SymbolTable represents the symbol table.
The core constituent of this class is the ArrayList of the Scope class, called allScopes, meaning that every scope is stored in this ArrayList.
Every scope contains Map of Symbols and strings as keys, and information about the scope such what scope it is enclosed by. 
The key represents the name of the symbol, and must be unique to the scope and not found in an enclosing scope, as this could cause an ambiguity to arise. 
A symbol is either a variable or a function, in the class Symbol its data type, name and scope is stored. 

\subsection*{Decorated Abstract Syntax Tree}
\todo{to be written}
%Phase3
\section{Code Generation}
\todo{Subphases of this is yet unknown, multipass and singlepass should be in this}
%Multipass
%Singlepass

%	Syntaks Analyse
%		Scannar > Parser > ANTLR > AST - Visitor til AST
%		Implementation
%
%	Kontekstuel Analyse
%		Symboltable
%		Hvad tester vi for? Og hvorfor gør vi det?
%		Implementation
%
%	Code Generation
%		What do? How do?
%		How did we do ?
%
%
%
%
%
%
%
%
