\chapter{Compiler Design}\todo{Evenetuelt et nyt navn hvis dette er mere teori end hvordan vi implementere vores compiler}
A compiler can be seperated into three phases; syntax analysis, contextual analysis and code generation.
The syntax analysis phase checks whether or not the source code adheres to the rules for the languange, such as statement constructs.
The contextual analysis phase checks whether or not the languange is used correctly, such as type checking.
The code generation translates the source code into the target code once the syntax and contextual analysis phases have accepted the source code.
This chapter examines each phase of the compiler and what these entail.
How each phase is constructet for GAMBLE will also be gone over in this chapter. %This may or may not change

\info[inline]{A few thing such as symbol table i am unaware where as to put, but possibly in the Metatext for compiler design, or in metatext for syntax analysis}
\improvement[inline]{Metatext}
%Antlr
\section{Implementation}\label{sec:ANTLR}
The syntax analysis is implemented through the tool \acrfull{antlr}, this tool provides some advantages over other available tools.
\acrshort{antlr} is based upon a parser technique called LL(*).
LL(*) uses an algorithm to have a varying lookahead when needed.
The LL(*) parsers, which is a parser that upholds an LL(*) grammar, does not allow a bigger class of \acrshort{cfg}s than other parsers like LL(k), but can change the number of tokens needed as lookahead dynamically. 
In the most recent version of \acrshort{antlr} as time of this publication, \acrshort{antlr}4, the underlying algorithm have been extended to a parser technique called Adaptive LL(*) (ALL(*)).
An important feature of ALL(*) is it moves grammar analysis to parse time and thereby lets the algorithm accept any non-left-recursive productions.
On top of that \acrshort{antlr} allows simple left recursions by rewriting them before parse time.

The \acrshort{antlr} approach accepts a broader class of grammars than most other parsing methods, one way this is done is to rule out ambiguity by using a rule of precedence.
If a grammar is ambiguous the ALL(*) approach will take the first available rule in the \acrshort{cfg} and apply it.
This allows for more opportunities in the \acrshort{cfg} and while most grammars could be rewritten to be unambiguous without applying the precedence rule.
The idea with the ALL(*) algorithm is that the grammar is analysed at parse-time, and requires no static analysis of the grammar. 
This means that the undecidability of static LL(*) grammar analysis is avoided and instead it is possible to make correct parsers for any non-left-recursive \acrshort{cfg}.
This allows \acrshort{antlr} access to input sequences while reading through the grammar, meaning not all possible inputs must be considered.
Due to this dynamic analysis \acrshort{antlr}4 is able to handle some ambiguous constructs and reduce-reduce conflicts.
As mentioned this allows \acrshort{antlr} to take care of left-recursion if such is present in the grammar by rewriting it, as such would be the case in \myref{lst:amb}.

\begin{lstlisting}[caption=An ambiguous rule for expr, which ANTLR handles by applying the first rule of the production if possible,frame=tlrb,label={lst:amb}]
expr : expr '*' expr 	#MulExpression // match expressions with * operator
     | expr '+' expr 	#AddExpression// match expressions with + operator
     | INT 		// matches simple integer
     ;
\end{lstlisting}
\myref{lst:amb} implements \acrshort{antlr}s way of representing operator precedence by simply obeying the first alternative in the rule set, as such the multiplication operator (``*'') will have the higher precedence.
The ALL(*) algorithm also means that one can completely disregard lookahead and it will still be able to parse, although one should keep in mind that having more lookahead than necessary will slow down the process of parsing.
The scanner provided by \acrshort{antlr} groups related tokens into token types such as INT, ID and FLOAT.
In \acrshort{antlr} a token contains at least two pieces of information, the token type and the matched text for the token.
\acrshort{antlr} also implements rule element labels in its \acrfull{cfg} which means one can apply label rules to a construct in a grammar, this allows for conditional steps in the grammar based on the source code being parsed.
The labels on \myref{lst:amb} are \texttt{\#MulExpression} and \texttt{\#AddExpression}.
Furthermore \acrshort{antlr} can set up an interface and base implementation of the visitor pattern for the parse tree on a given grammar by running \acrshort{antlr} with the \texttt{--visitor} flag. \citep{ALLSTAR, LLSTAR, antlr4_Book}
\todo{skal flytte}
%Phase1
\section{Syntax Analysis}

\info[inline]{meta for this, phases are done in their own subsection files}
%Subphases
\subsection{Scanner}
The first stage of syntax analysis is the scanner, also called the lexer which handles the lexical analysis.
The primary function of a scanner is to transform a sequence of characters into a sequence of tokens.
The scanner makes sure that the source code adheres to the grammar rules provided by the CFG.
An example of this, would be that you could use the notation .1 or 0.1 for a decimal number, both being turned into valid tokens by the scanner.
The scanner provided by ANTLR groups related tokens into token types such as INT, ID and FLOAT.
In ANTLR a token contains at least two pieces of information, the token type and the matched text for the token.

Some examples of our lexical rules for \gls{gamble} can be seen on \myref{lst:token}.
The definition of an integer number on line 3 states that an integer is either a zero or an optional negative sign followed by a single digit from one to nine followed by zero or more numbers from zero to nine.
It is necessary to clearly define tokens for the lexer to read in order to read source code correctly. \citep{Crafting_book}

\begin{lstlisting}[caption=Example of our lexer rules for ANTLR4,frame=tlrb,label={lst:token}]
// Integers
INT: 'int' | 'int16' | 'int32' | 'int64' ; // Integers
INTNUM: '0' | SIGN? [1-9][0-9]* ;

// Matrices and vectors
MATRIX: 'matrix' ;
ROWVECTOR: 'rowvector' | 'rvec' ;
COLVECTOR: 'colvector' | 'cvec' ;  

// Whitespace and comments
WS: [ \t ]+ -> skip;
NL: [ \r \n | \n ] -> skip;

COMMENT
    :   '/*' .*? '*/' -> skip
    ;

LINE_COMMENT
    :   '//' ~[\r\n]* -> skip
    ;
\end{lstlisting}
\subsection*{Parser}\label{subsec:parser}
The parser is based on the \acrfull{cfg} of \gls{gamble} written in \acrfull{ebnf}, whose alphabet consists of tokens produced by the scannar.
The parser reads tokens and groups them into phrases according to the \acrshort{cfg}.
The parser verifies that the syntax is correct and upholds to the \acrshort{cfg}, and if a syntax error is found it provides a corresponding error message. \citep{Crafting_book}
By using a parser generator like \acrshort{antlr} or SableCC, handling of syntactic errors and repairs can be done automatically.
A parser can also be written manually but doing so can result in syntactic errors that is hard to locate or solve.
Writing a parser by hand can also take a lot of time, and it can be difficult to go back and change or add new productions to the syntax, which is something the project group will want to do due to the iterative development.
There are many parser generators which can be used like: SableCC, JavaCC, JFlex and many others, but we have chosen to use \acrshort{antlr}.
\acrshort{antlr} has been chosen due to their special use of the ALL(*) grammar, which poses many opportunities for the grammar, and also makes the \acrshort{cfg} easier to write.
\acrshort{antlr} generates a parser which produces a parse tree that contains information about how the parser have grouped the tokens into more abstract language definitions, such as expressions and statements.

There are different kind of parsers, most common are bottom-up and top-down parsers.
\acrshort{antlr} makes a top-down parser, more specific a recursive descent parser.
A recursive descent parser is a subtype of top-down parser build from a set of mutually recursive procedures where each such procedure implements one of the productions of the grammar.
The structure of the resulting program closely mirrors the grammar it recognizes. \citep{Recursive_programming}
Recursive-descent parsers are a collection of recursive methods, one per rule of the \acrshort{cfg}.
Such a method for an assignment rule may look as shown in \myref{lst:rdpmethod}, where the rule is \texttt{assignment : ID = expr ;}.
So the method expects an ID to be the first token from the tokenstream, then an assignment operator followed by an expression and a semicolon.
Here the expression is a rule itself, and is therefore called on the expected expression.
An error should be returned if anything is not what was expected.
\begin{lstlisting}[caption=Example a recursive descent parser method,frame=tlrb,label={lst:rdpmethod}]
// assign : ID ``='' expr ``;'' ;
void assign() { // method generated from rule assign
match(ID); // compare ID to current input symbol then consume
match('=');
expr(); // match an expression by calling expr()
match(';');
}
\end{lstlisting}

%The second stage of the parser is the actual parser.
%The parser is fed a stream of tokens to recognise a sentence structure and in turn outputs the structure to a parse tree.
%The parse tree records how the parser recognises the structure of the input and its components.
%The parse tree that \acrshort{antlr} provides contains information about how the parser have grouped the tokens into more abstract languange definitions such as expressions and statements.
%Where previous versions of \acrshort{antlr} have also implemented the AST, it is not contained in \acrshort{antlr} V4 instead the parse tree provided by \acrshort{antlr} have been used to generate an AST this is discussed in \myref{sec:AST}.
%This tree is a trimmed version of the parse tree, where the less informative data have been removed, this makes it easier to read, and thus easier to use throughout development of the rest of the compiler.

%2nd stage is the actual parser, feeds of tokens to recognize sentence structure
%Parse tree records how the parser recognized structure of input and its component phrases
%Trees provide an easy to walk data structure that will be helpful for the rest of the compiler
%2.2 Implementing Parser - Recursive descent
%Recursive-descent parsers are really just a collection of recursive methods, one per rule.
%Such a rule may look similar to this
%// assign : ID ``='' expr ``;'' ;
%void assign() { // method generated from rule assign
%match(ID); // compare ID to current input symbol then consume
%match('=');
%expr(); // match an expression by calling expr()
%match(';');
%}
%Descent refers to the fact we start from the root and go down to the leaves(tokens)
%Reursive descent is just one form of top-down parsers.					NOTE topdown/bottom up parsing
%The call graph traaced out by invoking methods, mirrors the interior parse tree nodes
%To Build a parse tree manually one would insert ``add new subroot note' operations at the start of each rule, and a ``add new leaf node'' operation to match()
%The assign method checks if all necessary tokens are present and in the right order. When the parser enters assign it doesnt have to choose between more than one alternative. An alternative is one of the choices on the right side of a rule def. A parsing method for such rule would be a switch which looks for what token is present.
% This is called a parising decision or prediction by examining next token
%This is where lookahead comes into play , the lookahead token is the next input token, this can be any token the parser "sniffs" before consuming
%This is one of the places where \acrshort{antlr} is an especially handy tool to use, because \acrshort{antlr} allows for more lookahead than other parser generators.
%Most parsers use a lookahead of one which LL(1) or LR(1), \acrshort{antlr} tones the lookahead up and down depending on what token stream it is trying to decode, as such the \acrshort{antlr} has a lookahead of LL(*)
%\acrshort{antlr} Solves simple ambiguity simply by using the first mentioned rule.
%AST only useful, Parse all artifacts(space, brackets and so on)


\subsection*{Abstract Syntax Tree}
\todo{to be written}
%Phase2
\section{Contextual Analysis}
\info[inline]{meta for this, phases are done in their own subsection files}
%Subphases
\subsection*{Contextual Constraints}
\info[inline]{this includes cope rules, type rules and possibly semantics}
\subsection*{Decorated Abstract Syntax Tree}
\todo{to be written}
%Phase3
\chapter{Code Generation}
Code generation is the phase in which the object code is generated, the process and considerations this entails are covered in this chapter.
Object code is the output of the compiler.
Once the source code has passed through syntax analysis and contextual analysis without errors it has been validated and the compiler can proceed with generating the object code.
In the case of this compiler the object code is OpenCL C, as a result one or more compilers beyond this one are required to eventually end up with machine code that can be executed.\todo{more compilers? - e.g. vi bruger vores egen og gcc - Marc}
The object code is OpenCL C this means that tasks such as instruction selection and scheduling as well as register allocation will be handled by the compiler which will compile the OpenCL C code rather than \gls{gamble}.\todo{Håndterer compileren scheduling ????? - Søren - Instruction scheduling var en af de nævnte ting fra SPO kurset som man på lavniveau håndtere ja - Marc .. instruction scheduling er jo hvilken rækkefølge en given sekvens af instruktioner udføres i, ikke hvilken process har adgang til CPUen på et givent tidspunkt. -- Troels}
Furthermore in the code generation phase optimisation of the object code also takes place, for \gls{gamble} this means to create object code which is quickly executed and utilises the \acrshort{gpu} for calculation that benefit from its use.\todo{er det ok at bruge optimisation her, vi code gen'er jo bare, det er jo ikke rigtig en optimering. ? :) - Søren - Men området er referet til som optimisation, det er også derfor der efter står hvad betydningen af optimisation er for gamble - Marc }
A state diagram showing the sub-phases of the code generation can be seen in \myref{fig:flowCodegen}.

\vspace{10pt}
\begin{figure}[h]
    \centering
    \begin{tikzpicture}[node distance = 3cm, auto]
        %\node (invi1) [invi, draw=none] {};
        %\node (ast) [lille, below=-0.35cm of invi1] {Abstract Syntax Tree};
        %\node (symboltable) [lille, minimum width=6.75cm, minimum height=2.4cm, right=2cm of invi1, fill=blue!10, label={[xshift=0cm, yshift=-1cm]Symbol Table}] {};
        %\node (scope) [lille, right=1.1cm of ast] {Scopechecker};
        %\node (type) [lille, right=0.7cm of scope] {Typechecker};
        %\node (dast) [lille, right=1.1cm of type] {Decorated Abstact Syntax Tree};

        %\node (error) [cloud, below=1cm of symboltable] {Error report};

        %\draw [arrow] (ast) -- (scope);
        %\draw [arrow] (scope) -- (type);
        %\draw [arrow] (type) -- (dast);
        %\draw [arrow,dashed] (scope) -- (error);
        %\draw [arrow,dashed] (type) -- (error);
        %\draw [arrow,dashed] (symboltable) -- (error);

        \node (dast) [lille, align=left] {Contextual \\Analysis Phase};
        \node (cgv) [lille, right=0.7cm of dast, align=left] {Code Generation \\Visitor};
        \node (copy) [lille, right=0.7cm of cgv] {Output \gls{opencl} C code};
        \node (error) [invi, draw=none, minimum width=2cm, right=1cm of copy, label={[xshift=40pt, yshift=-17pt]Finished Compilation}] {};

        \draw[black,fill=black, above=1cm of parser] (10.9,0) circle (1ex);
        \draw[black, above=1cm of parser] (10.9,0) circle (1.3ex); 

        \draw [arrow] (dast) -- (cgv);
        \draw [arrow] (cgv) -- (copy);
        \draw [arrow] (copy) -- (error);
    \end{tikzpicture}
    \caption{State diagram showing the modules of the code generation. } 
    \label{fig:flowCodegen}
\end{figure}
\vspace{-20pt}


%	Syntaks Analyse
%		Scannar > Parser > ANTLR > AST - Visitor til AST
%		Implementation
%
%	Kontekstuel Analyse
%		Symboltable
%		Hvad tester vi for? Og hvorfor gør vi det?
%		Implementation
%
%	Code Generation
%		What do? How do?
%		How did we do ?
%
%
%
%
%
%
%
%
