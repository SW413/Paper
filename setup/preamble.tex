%  A simple AAU report template.
%  2014-09-13 v. 1.1.0
%  Copyright 2010-2014 by Jesper Kjær Nielsen <jkn@es.aau.dk>
%
%  This is free software: you can redistribute it and/or modify
%  it under the terms of the GNU General Public License as published by
%  the Free Software Foundation, either version 3 of the License, or
%  (at your option) any later version.
%
%  This is distributed in the hope that it will be useful,
%  but WITHOUT ANY WARRANTY; without even the implied warranty of
%  MERCHANTABILITY or FITNESS FOR A PARTICULAR PURPOSE.  See the
%  GNU General Public License for more details.
%
%  You can find the GNU General Public License at <http://www.gnu.org/licenses/>.
%
\documentclass[11pt,twoside,a4paper,openright]{memoir}
%%%%%%%%%%%%%%%%%%%%%%%%%%
%% COMMAND DEACTIVATION %%
%%%%%%%%%%%%%%%%%%%%%%%%%%

\let\added\undefined
\let\deleted\undefined


%%%%%%%%%%%%%%
%% PACKAGES %%
%%%%%%%%%%%%%%


%%% Initial things %%%
% Fix various issues with LaTeX2e
\usepackage{fixltx2e}
% Font package
\usepackage{fourier}
% Index
\usepackage{makeidx}
\makeindex


%%% Translations and character encodings %%%
% Enable use of several characters, including æ, ø and å
\usepackage[utf8]{inputenc}
%  language
\usepackage[english]{babel}
% Use PostScript fonts instead of bitmap ones. Also does other stuff.
\usepackage[T1]{fontenc}
% Various LaTeX symbols
\usepackage{latexsym}
% Wider selection of colours
\usepackage[pdftex,dvipsnames,table]{xcolor}  % Coloured text etc.% Improved element justification
\usepackage{ragged2e}
% Font improvements
\usepackage{fix-cm}
% Enables inclusion of PDF files
\usepackage{pdfpages}
% Enables various forms of lines, like double-underlining (\uuline{})
\usepackage[normalem,normalbf]{ulem}
% Sets the tolerance for distance between words, determining when to hyphenate.
\pretolerance=2500


%%% Figures and tables (Floats) %%%
% Ensures that floats won't appear -before- the place where they're added
\usepackage{flafter}
% Enable multi-rows and -columns
\usepackage{multirow}
\usepackage{multicol}
% Double, horizontal lines
\usepackage{hhline}
% Enables coloured tables
\usepackage{colortbl}
% Gives improved control over placement of floats
% \begin{figure}[!h] % Won't be floating
\usepackage{here}
% Wrap text around figures
\usepackage{wrapfig}
% Wrap text around tables
\usepackage{floatflt}
% Enables the \FloatBarrier command
\usepackage{placeins}
% Rotation of figures
\usepackage{rotating}
% Framed boxes
\usepackage{framed}
% Booktabs - Fancy tables
\usepackage{booktabs}
% Enables inclusion of PDF documents of version 1.6+
%\pdfoptionpdfminorversion=6


%%% Mathematic formulas %%%
% AMS math
%\usepackage{amsmath}
%\usepackage{amssymb}
% Extra fonts (for math, I think)
\usepackage{stmaryrd}
% Access text symbols
\usepackage{textcomp}
% Extend AMS
%\usepackage{mathtools}
%\usepackage{cancel}
% Use theorems in your document
% The ntheorem package is also used for the example environment
% When using thmmarks, amsmath must be an option as well. Otherwise \eqref doesn't work anymore.
%\usepackage[framed,amsmath,thmmarks]{ntheorem}
% Pretty fractions, just because
%\usepackage{nicefrac}


%%% Graphics %%%
% Various image-commands
\usepackage{eso-pic}
% Use JPEG and PNG images
\usepackage{array,graphicx}
\DeclareGraphicsExtensions{.pdf,.png,.jpg}
\graphicspath{{./images/}}
%Wrapping Images
\usepackage{wrapfig}


%%% Text stuff %%%
% Filler text
%\usepackage{lipsum}
% Page counting
%\usepackage{totpages}
% Acronyms
%\usepackage{acronym}
%avoid widows
%\widowpenalty10000


%%% Source Code Stuff %%%
% Adds \lstinline!code there!, where !! are delimeters not used in the code
% Adds the environment: lstlisting
% Adds command \lstinputlisting[options]{filename.ext}
% More info in manual.
%\usepackage{listings}
%\lstloadlanguages{[Sharp]C,XML,SQL}
%\lstset{numbers=left,
%        numberstyle=\tiny,
%        stepnumber=2,
%        numbersep=5pt,
%        frame=tb,
%        inputencoding=utf8,
%        tabsize=2,
%        extendedchars=true,
%        language=[Sharp]C}

\usepackage{color}
\definecolor{bluekeywords}{rgb}{0.13,0.13,1}
\definecolor{greencomments}{rgb}{0,0.5,0}
\definecolor{redstrings}{rgb}{0.9,0,0}

\usepackage{courier}

\usepackage{listings}

\lstdefinelanguage{JavaScript}{
  keywords={typeof, new, true, false, catch, function, return, null, catch, switch, var, if, in, while, do, else, case, break},
  keywordstyle=\color{blue}\bfseries,
  ndkeywords={class, export, boolean, throw, implements, import, this},
  ndkeywordstyle=\color{darkgray}\bfseries,
  identifierstyle=\color{black},
  sensitive=false,
  comment=[l]{//},
  morecomment=[s]{/*}{*/},
  commentstyle=\color{purple}\ttfamily,
  stringstyle=\color{red}\ttfamily,
  morestring=[b]',
  morestring=[b]"
}


\lstset{language=[Sharp]C,
  captionpos=b,            
  columns=fixed,     
  numbers=left,
  numberstyle=\tiny,
  showspaces=false,
  showtabs=false,
  breaklines=true,
  inputencoding=utf8,
  showstringspaces=false,
  breakatwhitespace=true,
  escapeinside={(*@}{@*)},
  commentstyle=\color{greencomments},
  keywordstyle=\color{bluekeywords},
  stringstyle=\color{redstrings},
  basicstyle=\ttfamily\small,
  literate={ø}{{\o}}1
  		   {æ}{{\ae}}1
  		   {å}{{\aa}}1
  		   {Ø}{{\O}}1
  		   {Æ}{{\AE}}1
  		   {Å}{{\AA}}1
         {§}{{\S}}1
}

\lstdefinestyle{make}{tabsize=4}

%%% References, bibtex and URLs %%%
% Post URLs. Allows breaking at hyphens to help avoid long links.
\usepackage[hyphens]{url}
% Better cross references
\usepackage[danish]{varioref}
% Enable natbib citation styles
\usepackage[numbers]{natbib}
% Define a new 'leo' style for URL package, that will use a smaller font
\makeatletter
\def\url@leostyle{%
  \@ifundefined{selectfont}{\def\UrlFont{\sf}}{\def\UrlFont{\small\ttfamily}}
}
\makeatother
% And of course, use this new style
\urlstyle{leo}




%%% Floats %%%
% Not entirely sure why I need this yet
\let\newfloat\relax
\usepackage{float}
% Enables usage of \subcaption, \subtop and \subbottom
\newsubfloat{figure}


%%% Todo Stuff %%%
% Insert needed corrections with \fixme{..}, which will cause an error during compile, if any are present once 'draft'
% is replaced with 'final'
\usepackage[footnote,draft,english,silent,nomargin]{fixme}
\fxsetup{layout={footnote,marginclue,index},innerlayout={inline,index}}


%%% Changes Markup %%%
% Markup changes of varying types.
% Adds the commands:
%  - \added[id=(author id), remark={remark text}]{new text}
%  - \deleted[id=(author id), remark={remark text}]{old text}
%  - \replaced[id=(author id), remark={remark text}]{new text}{old text}
\usepackage[xcolor,authormarkup=footnote]{changes}
% Adds the commands:
%  - \cbstart
%  - \cbend
%  - \cbdelete
%  - Environment: changebar
\usepackage[outerbars,xcolor]{changebar}
\cbcolor{red}



%%%%%%%%%%%%%%%%%%%%%%%
%% DOCUMENT SETTINGS %%
%%%%%%%%%%%%%%%%%%%%%%%


%%% Margins %%%
% \setlrmarginsandblock{binding}{edge}{ratio}
\setlrmarginsandblock{2.0cm}{2.0cm}{*}
% \setulmarginsandblock{top}{bottom}{ratio}
\setulmarginsandblock{2.0cm}{2.0cm}{*}
% Performs various calculations and makes several non-Memoir things work with the Memoir class
\checkandfixthelayout 
% Correct todonotes placement
\reversemarginpar


%%% Paragraph formatting %%%
% Size of paragraph indentation
\setlength{\parindent}{0mm}
% Distance between paragraphs (double enter)
\setlength{\parskip}{3mm}
% Line distance
\linespread{1,1}


%%% Bibliography %%%
% Defines parameters for the bibliography, such as the parenthesis and separators
%%%% OLD STYLE!
%\bibpunct{[}{]}{,}{a}{}{;}
% Bibliography style
%%%% OLD STYLE!
%\bibliographystyle{bibtex/harvard}
\bibliographystyle{plainnat}


%%% Table of contents %%%
% Depth of numbered headlines
\setsecnumdepth{subsubsection}
% Changing the document class' limit for number-depth
\maxsecnumdepth{subsection}
% Define the depth included in the table of contents
\settocdepth{section}
% Use letters instead of Roman numerals in TOC
\renewcommand{\thepart}{\Alph{part}}


%%% Text stuff %%%
% Removes distance between items in itemize
%\let\olditemize=\itemize
%\def\itemize{\olditemize\setlength{\itemsep}{-1ex}}
%% Removes distance between items in enumerate
%\let\oldenumerate=\enumerate
%\def\enumerate{\oldenumerate\setlength{\itemsep}{-1ex}}
\usepackage{enumitem}


%%% Changes (Language strings) %%%
\addto\captionsdanish{
  \def\listofchangesname{Ændringer i dokumentet}
  \def\summaryofchangesname{Ændringer}
  \def\changesaddname{Tilføjet}
  \def\changesdeletename{Slettet}
  \def\changesreplacename{Erstattet}
  \def\changesauthorname{Skribent}
  \def\changesanonymousname{anonym}
  \def\changesnoloc{Listen af ændringer tilgængelig efter næste \LaTeX\ kørsel.}
  \def\changesnosoc{Opsummering af ændringer tilgængelig efter næste \LaTeX\ kørsel.}
}


%%% Visual references %%%
% Enables clickable hyperlinks

%%% Add hidelinks to remove boxes around links

\usepackage[colorlinks,backref=page]{hyperref}
% General setup of hyperlinks package
\hypersetup{
    breaklinks = true,
    colorlinks = false,
    linkcolor = black,
    anchorcolor = black,
    citecolor = black
}


%%% Colour definitions %%%
% Defines: gray
\definecolor{gray}{gray}{0.80}
% Defines: numbercolor
\definecolor{numbercolor}{gray}{0.7}
% Defines: shadecolor
\definecolor{shadecolor}{RGB}{33,26,82}
% Defines: aaublue
\definecolor{aaublue}{RGB}{33,26,82}


%%% Figure and table texts setup %%%
% Font definition for the 'Figure' or 'Table' displays.
\captionnamefont{\small\bfseries\itshape}
% Font definition for the numbering
\captiontitlefont{\small}
% Delimiter between number and figure text
\captiondelim{. }
% Left justify multi-line figure texts below one another
\hangcaption
% Width of figure text
\captionwidth{\linewidth}
% Distance below figure text
\setlength{\belowcaptionskip}{10pt}
% Fix space between figure number and name
\setlength{\cftfigurenumwidth}{14mm}


%%% Page header and footer %%%
% Define width of header and footer
\setlength{\headwidth}{\textwidth}
% Create pagestyle for pages with and without a new chapter
\makepagestyle{reportPlain}
\makepagestyle{reportChapter}
% Pagestyle for chapter pages (Only a footer, of course)
\makefootrule{reportChapter}{\headwidth}{\normalrulethickness}{\footruleskip}
\makeevenfoot{reportChapter}{\thepage}{}{}
\makeoddfoot{reportChapter}{}{}{\thepage}
% Pagestyle for regular pages
\makerunningwidth{reportPlain}{\headwidth}
\makeheadposition{reportPlain}{flushright}{flushleft}{flushright}{flushleft}
\makeevenhead{reportPlain}{\leftmark}{}{}
\makeoddhead{reportPlain}{}{}{\rightmark}
\makeevenfoot{reportPlain}{\thepage}{}{}
\makeoddfoot{reportPlain}{}{}{\thepage}
\makeheadrule{reportPlain}{\headwidth}{\normalrulethickness}
\makefootrule{reportPlain}{\headwidth}{\normalrulethickness}{\footruleskip}
% Use pagestyles
\pagestyle{reportPlain}
\aliaspagestyle{chapter}{reportChapter}
\aliaspagestyle{part}{reportChapter}
\aliaspagestyle{title}{empty}
% Do not stretch pages
\raggedbottom


%%% Naming %%%
% Define various names for captions and such
%\addto\captionsdanish{
 % \renewcommand\appendixname{Bilag}
%  \renewcommand\contentsname{Indholdsfortegnelse} 
%  \renewcommand\appendixpagename{Bilag}
%  \renewcommand\cftchaptername{\chaptername~}
%  \renewcommand\cftappendixname{\appendixname~}
%  \renewcommand\appendixtocname{Bilag}
%}


%%% Appendix setup %%%
% Appendix setup. Might need some settings here
\usepackage{appendix}


%%% Chapter look and feel %%%
% Define style: jenor
\newif\ifchapternonum
\makechapterstyle{jenor}{
  \renewcommand\printchaptername{}
  \renewcommand\printchapternum{}
  \renewcommand\printchapternonum{\chapternonumtrue}
  \renewcommand\chaptitlefont{\fontfamily{pbk}\fontseries{db}\fontshape{n}\fontsize{25}{35}\selectfont\raggedleft}
  \renewcommand\chapnumfont{\fontfamily{pbk}\fontseries{m}\fontshape{n}\fontsize{1in}{0in}\selectfont\color{numbercolor}}
  \renewcommand\printchaptertitle[1]{%
    \noindent
    \ifchapternonum
    \begin{tabularx}{\textwidth}{X}
    {\let\\\newline\chaptitlefont ##1\par}
    \end{tabularx}
    \par\vskip-2.5mm\hrule
    \else
    \begin{tabularx}{\textwidth}{Xl}
    {\parbox[b]{\linewidth}{\chaptitlefont ##1}} & \raisebox{-15pt}{\chapnumfont \thechapter}
    \end{tabularx}
    \par\vskip2mm\hrule
    \fi
  }
}
% Use style: jenor
%\chapterstyle{jenor}
\usepackage{titlesec, blindtext}
\definecolor{gray75}{gray}{0.75}
\newcommand{\hsp}{\hspace{20pt}}
\titleformat{\chapter}[hang]{\Huge\bfseries\vspace{-40pt}}{\thechapter\vspace{-20pt}\hsp\textcolor{gray75}{|}\hsp}{0pt}{\Huge\bfseries\vspace{-20pt}}

%%% Misc stuff %%%
% Use regular numbers for pages
%\pagenumbering{arabic}
% Word and letter counts
\newcommand{\wordcount}{\input{preamble/sums/wordcount.sum}}
\newcommand{\charcount}{\input{preamble/sums/charcount.sum}}
\newcommand{\lettercount}{\charcount}
\newif\ifcounts
% Italicized quote-environment
\newenvironment{italicquote}{\begin{quote}\itshape}{\end{quote}}
% Acronyms in list or not (true if in list)
\newif\ifAcroList
\AcroListfalse


%%% Left-aligning bibliography %%%
%\renewcommand*{\bibfont}{\raggedright}


%%%% these patches ensure that the backrefs point to the actual occurrences of the citations in the text, not just the page or section in which they appeared
%%%% http://tex.stackexchange.com/questions/54541/precise-back-reference-target-with-hyperref-and-backref
%%%% BEGIN BACKREF DIRECT PATCH, apply these AFTER loading hyperref package with appropriate backref option
% The following options are provided for the patch, currently with a poor interface!
% * If there are multiple cites on the same (page|section) (depending on backref mode),
%   should we show only the first one or should we show them all?
\newif\ifbackrefshowonlyfirst
\backrefshowonlyfirstfalse
%\backrefshowonlyfirsttrue
%%%% end of options
%
% hyperref is essential for this patch to make any sense, so it is not unreasonable to request it be loaded before applying the patch
\makeatletter
% 1. insert a phantomsection before every cite, so hyperref has something to target
%    * in case natbib is loaded. hyperref provides an appropriate hook so this should be safe, and we don't even need to check if natbib is loaded!
\let\BR@direct@old@hyper@natlinkstart\hyper@natlinkstart
\renewcommand*{\hyper@natlinkstart}{\phantomsection\BR@direct@old@hyper@natlinkstart}% note that the anchor will appear after any brackets at the start of the citation, but that's not really a big issue?
%    * if natbib isn't used, backref lets \@citex to \BR@citex during \AtBeginDocument
%      so just patch \BR@citex
\let\BR@direct@oldBR@citex\BR@citex
\renewcommand*{\BR@citex}{\phantomsection\BR@direct@oldBR@citex}%

% 2. if using page numbers, show the page number but still hyperlink to the phantomsection instead of just the page!
\long\def\hyper@page@BR@direct@ref#1#2#3{\textit{\hyperlink{#3}{Page #1}}}

% check which package option the user loaded (pages (hyperpageref) or sections (hyperref)?)
\ifx\backrefxxx\hyper@page@backref
    % they wanted pages! make sure they get our re-definition
    \let\backrefxxx\hyper@page@BR@direct@ref
    \ifbackrefshowonlyfirst
        %\let\backrefxxxdupe\hyper@page@backref% test only the page number
        \newcommand*{\backrefxxxdupe}[3]{#1}% test only the page number
    \fi
\else
    \ifbackrefshowonlyfirst
        \newcommand*{\backrefxxxdupe}[3]{#2}% test only the section name
    \fi
\fi

% 3. now make sure that even if there is no numbered section, the hyperref's still work instead of going to the start of the document!
\RequirePackage{etoolbox}
\patchcmd{\Hy@backout}{Doc-Start}{\@currentHref}{}{\errmessage{I can't seem to patch backref}}
\makeatother
%%%% END BACKREF PATCHES


% Enable figures like checkmarks
\usepackage{pifont}
\newcommand{\cmark}{\ding{51}}%
\newcommand{\xmark}{\ding{55}}%

%%%%%%%%%%%%%%%%%%%%%%%%%%%%%%%%%%%%%%%%%%%%%%%%
%Flowchart
\usepackage{tikz}
\usetikzlibrary{shapes.geometric, arrows, positioning}
%%%%%%%%%%%%%%%%%%%%%%%%%%%%%%%%%%%%%%%%%%%%%%%%

%\usepackage{bera}% optional: just to have a nice mono-spaced font

\newcommand*\rot{\rotatebox{90}}

\colorlet{punct}{red!60!black}
\definecolor{background}{HTML}{EEEEEE}
\definecolor{delim}{RGB}{20,105,176}
\colorlet{numb}{magenta!60!black}

\lstdefinelanguage{json}{
    basicstyle=\normalfont\ttfamily,
    numbers=left,
    numberstyle=\scriptsize,
    stepnumber=1,
    numbersep=8pt,
    showstringspaces=false,
    breaklines=true,
    frame=lines,
    backgroundcolor=\color{background},
    literate=
     *{:}{{{\color{punct}{:}}}}{1}
      {,}{{{\color{punct}{,}}}}{1}
      {\{}{{{\color{delim}{\{}}}}{1}
      {\}}{{{\color{delim}{\}}}}}{1}
      {[}{{{\color{delim}{[}}}}{1}
      {]}{{{\color{delim}{]}}}}{1},
}


%% SOME MACROS HYPE

\newcommand{\class}[1] {\texttt{#1}}


%%%%%%%%%%%%%%%%%%%%%%%%%%%%%%%%%%%%%%%%%%%%%%%%
% Misc
%%%%%%%%%%%%%%%%%%%%%%%%%%%%%%%%%%%%%%%%%%%%%%%%
% Add bibliography and index to the table of
% contents
% Add the command \pageref{LastPage} which refers to the
% page number of the last page
%\usepackage[
% disable, %turn off todonotes
%colorinlistoftodos, %enable a coloured square in the list of todos
%textwidth=\marginparwidth, %set the width of the todonotes
%textsize=scriptsize, %size of the text in the todonotes
%prependcaption
%]{todonotes}
%\usepackage{xargs}                      % Use more than one optional parameter in a new commands

\usepackage{tocbibind}

\usepackage[english]{cleveref}
\crefname{exa}{eksempel}{eksempler}
\newcommand{\myref}[1]{\autoref{#1}}
\newcommand{\Myref}[1]{\Autoref{#1}}


% GLOSARIEIASI
\usepackage[acronym]{glossaries}
\usepackage{xparse}

\makeglossaries

%\newglossaryentry{gpu}{name=GPU, description={Graphics Processing Unit}}
%\newglossaryentry{cpu}{name=CPU, description={Central Processing Unit}}
%\newglossaryentry{gpgpu}{name=GPGPU, description={General-Purpose computing on Graphics Processing Units}}

% \gls{CUDA}
% \glspl{CUDAs}
% \Gls{<label>}
\newglossaryentry{cuda}{name=CUDA, description={Compute Unified Device Architecture: a parallel computing platform and programming model created by NVIDIA.}}
\newglossaryentry{gamble}{name=GAMBL, description={The name of the programming language made in this report.}}
\newglossaryentry{opencl}{name=OpenCL, description={Open Computing Language: a framework for writing programs that execute across heterogeneous platforms.}}
\newglossaryentry{heterogeneous}{name=heterogeneous, description={Systems with more than one kind of processor can be described as a heterogenous system.}}
\newglossaryentry{Parallel Thread Execution}{name=PTX, description={Pseudo-assembly languange used by Nvidia and OpenCL as the target languange when compiling kernals}}

% \acrfull == CentralProcessingUnit(CPU)
% \acrlong == CentralProcessingUnit
% \acrshort == CPU
\newacronym{cpu}{CPU}{Central Processing Unit}
\newacronym{gpu}{GPU}{Graphics Processing Unit}
\newacronym{gpgpu}{GPGPU}{General-Purpose computing on Graphics Processing Units}
\newacronym{simd}{SIMD}{Single Instruction Multiple Data}
\newacronym{ebnf}{EBNF}{Extended Backus–Naur Form}
\newacronym{cfg}{CFG}{Context-Free Grammar}
\newacronym{regex}{regex}{regular expression}
\newacronym{ast}{AST}{Abstract Syntax Tree}
\newacronym{antlr}{ANTLR}{ANother Tool for Languange Recognition}
\newacronym{jit}{JIT}{Just-In-Time}
\newacronym{ptx}{PTX}{Parallel Thread Execution}


\usepackage{xargs}                      % Use more than one optional parameter in a new commands
\usepackage[colorinlistoftodos,prependcaption,textsize=tiny]{todonotes}

%\usepackage[colorinlistoftodos,prependcaption,textsize=tiny]{todonotes}
\newcommandx{\unsure}[2][1=]{\todo[linecolor=red,backgroundcolor=red!25,bordercolor=red,#1]{#2}}
\newcommandx{\change}[2][1=]{\todo[linecolor=blue,backgroundcolor=blue!25,bordercolor=blue,#1]{#2}}
\newcommandx{\info}[2][1=]{\todo[linecolor=OliveGreen,backgroundcolor=OliveGreen!25,bordercolor=OliveGreen,#1]{#2}}
\newcommandx{\improvement}[2][1=]{\todo[linecolor=Plum,backgroundcolor=Plum!25,bordercolor=Plum,#1]{#2}}
\newcommandx{\thiswillnotshow}[2][1=]{\todo[disable,#1]{#2}}



\usepackage{tikz}

\usetikzlibrary{shapes.geometric, arrows}

\tikzstyle{lille} = [rectangle, minimum width=2cm, minimum height=1cm,text centered, draw=black, fill=red!30]
\tikzstyle{main} = [rectangle, minimum width=2cm, minimum height=2cm,text centered, draw=black, fill=red!30]
\tikzstyle{hoj} = [rectangle, minimum width=1cm, minimum height=2cm,text centered, draw=black, fill=red!30]
\tikzstyle{cloud} = [draw, ellipse,fill=yellow!20, node distance=0.7cm, minimum height=2em]
\tikzstyle{invi} = [draw, ellipse, node distance=3cm, minimum height=2em]
\tikzstyle{line} = [draw, -latex']
\tikzstyle{arrow} = [thick,->,>=stealth]
\tikzstyle{blockz} = [rectangle, minimum width=4cm, minimum height=1cm,text centered, draw=black, fill=blue!30]




\usepackage{pdflscape}