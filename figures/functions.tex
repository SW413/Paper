\begin{table}[h]
    \centering
    \colorlet{shadecolor}{gray!40}
    \rowcolors{1}{white}{shadecolor}
    \begin{tabular}{|l|p{8.5cm}|}
    \hline
    \textbf{Function}                                               &\textbf{Description}                                                                               \\ \hline
    \texttt{print(\textit{any*})}                                   &Takes the parameter(s) and print it in the console.                                                 \\ \hline 
    \parbox[t]{5cm}{\texttt{rows(matrix m)}\\\texttt{rows(vector v)}} &Return the number of rows for the given vector or matrix.                                          \\ \hline
    \parbox[t]{5cm}{\texttt{cols(matrix m)}\\\texttt{cols(vector v)}} &Return the number of cols for the given vector or matrix.                                          \\ \hline    
    \texttt{fileToMatrix(string s)}                                 &Reads the input file of \texttt{s} and return it as a matrix.                                        \\ \hline
    \texttt{matrixToFile(matrix m , string s)}                      &Takes matrix and saves it in the given path.                                                       \\ \hline  
    %\texttt{MakeIntMatrix(int r, int c, int[] vals)}                &Creates an int matrix with the specified rows and columns and the value desired at each entry.     \\ \hline
    %\texttt{MakefloatMatrix(int r, int c, float[] vals)}            &Creates an float matrix with the specified rows and columns and the value desired at each entry.   \\ \hline
    \end{tabular}
    \caption{Available functions in \gls{gamble}, their input parameters and a short description for each. *the print function takes any parameter including strings.}\label{tbl:funcs}
\end{table}
%\vspace{-20pt}