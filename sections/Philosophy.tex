%GAMBLE is used to refer to our languange
\section{Design philosophy}
For complex computations or large datasets one must acknowledge the computing power required for such computations.
The main focus for GAMBLE is to use the computational powers of the GPU to handle such computations without it being inconvenient for the programmer.
From \myref{sec:state_of_the_art} we know that several libraries and languages allows the programmer to explicitly designate workloads to the GPU however this often requires explicit memory handling as well.
GAMBLE is meant to be used for computations which has a need for more computational power; due to this the language should not be too unfamiliar.
Keeping GAMBLE familiar to other languages reduces the time required for users to familiarize themselves with GAMBLE.
This is done by using a C-like syntax, while stripping GAMBLE of features we deem not necessary for a laungauge which main focus lies in easy acces to GPGPU programming and linear algebra calculations.
These choises will be documented in the rest of this chapter.
As the GPU is the resource being used to achieve more computing power, the data computed must also be applicable to the niche of the GPU, i.e. the data must be parallelisable as explained in \myref{sec:comparch}.
This basic need will influence how data is represented, and also puts focus onto matrices and vector calculations, which as mentioned before often can be paralellised.
As it follows from this philosophy, some criteria for GAMBLE exists.
\begin{itemize}
	\item Allow the programmer to use the GPU without it being inconvenient.
	\item Avoid implementing unnecessary data types and features.
	\item Let the language be somewhat familiar to read and use.
	\item Let the language promote read- and write-ability.
\end{itemize}

\textbf{Allow the programmer to use the GPU without it being inconvenient}

Due to GAMBLE being focused on numerical computations, allowing the programmer to focus on managing the mathematical aspects is key.
Therefore having the programmer control the runtime architecture seems an unnecessary distraction.
GAMBLE takes care of designating the computations to the right processor, whether it be the GPU or the CPU, as such any inconvenience in that process is removed from the programmer whose focus can be solely on the mathematics.

\textbf{Avoid implementing unnecessary data types and features}

When deciding what any language should offer its user, it is important to keep in mind what is the purpose of the language.
As GAMBLEs purpose is to use the GPU for calculations which can be parallelised implementing features or data types that do not hold any regard to this aspect would clutter the language.
Additionally GAMBLE should not try to adapt itself towards purposes for which it is not designed, and example of excluding such features is the fact that strings are not part of the language, this choice and others like it are further documented in this chapter.

\textbf{Let the language be somewhat familiar to read and use}

As mentioned the main purpose of GAMBLE is to use the GPU for computations, and is focused on doing computations, not developing new software.
As such GAMBLE would most often be used where this niche is required.
It may even be likely that it is not used when developing an algorithm to do computations, but first used once the algorithm is complete, and can be applied to bigger sets of data.
Therefore to use the niche that GAMBLE proclaims, having the language be familiar makes it easier to use for its pure computational aspect.

\textbf{Let the language promote read- and write-ability}

Read- and write-ability directly influences the reliability of a language. 
The easier a program is to write, the more likely it is to be correct.%\citep{Sebesta}
This is gained through using a familiar syntax, as mention above, as well as sticking to certain language characteristics shown in \myref{tbl:concepts}.
It makes sense for GAMBLE to adopt a C-like syntax, as several of the most popular languages also uses a C-like syntax, the difference will be in how code is structured.\citep{TIOBE}
\begin{table}[h]
\begin{tabular}{|l|l|c|c|}
\hline
\begin{tabular}[c]{@{}l@{}}Language\\ Concepts\end{tabular} & Readability            & \multicolumn{1}{l|}{Write-ability} & \multicolumn{1}{l|}{Reliability} \\ \hline
Simplicity                                                   & \multicolumn{1}{c|}{x} & x                                 & x                                \\ \hline
Orthogonality                                                 & \multicolumn{1}{c|}{x} & x                                 & x                                \\ \hline
Data types                                                   & \multicolumn{1}{c|}{x} & x                                 & x                                \\ \hline
Syntax design                                                & \multicolumn{1}{c|}{x} & x                                 & x                                \\ \hline
Support for abstraction                                      &                        & x                                 & x                                \\ \hline
Expressivity                                                 &                        & x                                 & x                                \\ \hline
Type checking                                                &                        & \multicolumn{1}{l|}{}             & x                                \\ \hline
Exception handling                                           &                        & \multicolumn{1}{l|}{}             & x                                \\ \hline
Restricted aliasing                                           &                        & \multicolumn{1}{l|}{}             & x                                \\ \hline
\caption{Language evaluation criteria and the characteristics that affect them.\citep{Sebesta}}
	\label{tbl:concepts}
\end{tabular}
\end{table
