\section{Control Flow}\label{subsec:control-flow}
Control flow is any part of the language in which a branch occurs. 
A branch is in this case a change in which instruction happens next. 
These come in 3 variations: primitives, choice and loops.
C, C\#, Java etc contain some or all of these elements to control the flow of a program:

\begin{itemize}
    \item \texttt{if/if-else}
    \item \texttt{?:} (Ternary operator)
    \item \texttt{switch}
    \item \texttt{while}
    \item \texttt{for}
    \item \texttt{foreach}
    \item \texttt{do..while}
    \item \texttt{goto}
\end{itemize}

If a programming language has if, goto and label then the other constructs can be made.
For example a while-loop can be translated to a if and goto, this is shown in \myref{ifgotowhile1} and \myref{ifgotowhile2} written in C. 

\noindent\begin{minipage}{.45\textwidth}
\begin{lstlisting}[caption=Loop made with while.,frame=tlrb, label=ifgotowhile1, numbers=none]{Name}
while ( condition )
{
    statement();
}
\end{lstlisting}
\end{minipage}\hfill
\begin{minipage}{.45\textwidth}
\begin{lstlisting}[caption=The same loop with if and goto.,frame=tlrb, label=ifgotowhile2, numbers=none]{Name}
label startOfLoop:
if ( condition )
{
    statement();
    goto startOfLoop;
}
\end{lstlisting}
\end{minipage}

The same can be done with the \texttt{for} and \texttt{do..while} loops. 
This means that the only need control structures should be \texttt{if} and \texttt{goto}. 
However the use of goto is considered harmful, because it is easy to make mistakes when using it. \citep{DijkstraGoto}
Even though \texttt{for} and \texttt{while} are very similar, they are often used in different contexts.
The for loop is very useful for iterating over a collection or knowing exactly how many time the code is executed. 
Therefore we have decided to include both \texttt{for} and \texttt{while} in GAMBLE, but not \texttt{foreach} and \texttt{do..while}, this is to simplify the number of control structures.

We have decided to keep the traditional \texttt{if} and \texttt{if-else} constructs. 
These are very useful, and can achieve very much the same as a \texttt{switch}, by chaining if-else statements, so we have excluded it. 
The ternary operator, also known as the inline if, written as \texttt{?:} in C and many other languages, is useful for compacting code to make it more readable, therefore we include it in GAMBLE.

\subsection{if, if-else and ?:} \todo{Jeg blander meget rundt i statement og construct, kan ikke helt beslutte mig. }
Examples of if/if-else GAMBLE are shown in \myref{iflst} and \myref{ifelselst}. 
The condition is enclosed in a parenthesis this is to make the if construct similar to the function call, in which the parameters are in a parenthesis. 
All the conditional code must be enclosed in a curly bracket, this is to make the code more readable. 
The use of curly brackets is to make them similar to functions to make the language more alike \todo{mangler et godt fagord her.}.

\noindent\begin{minipage}{.45\textwidth}
\begin{lstlisting}[caption=An if statement in GAMBLE.,frame=tlrb, label=iflst, numbers=none]{Name}
if ( condition )
{
    statement();
}
\end{lstlisting}
\begin{lstlisting}[caption=A use of \texttt{?:} in GAMBLE.,frame=tlrb, label=terlst, numbers=none]{Name}
condition ? ifTrue() : ifFalse();
\end{lstlisting}

\end{minipage}\hfill
\begin{minipage}{.45\textwidth}
\begin{lstlisting}[caption=An if-else statement in GAMBLE.,frame=tlrb, label=ifelselst, numbers=none]{Name}
if ( condition )
{
    statement();
} else {
    alternative_statement();
}
\end{lstlisting}
\end{minipage}

An example of the ternary operator \texttt{?:} is shown in \myref{terlst}. 
If the boolean statement before the question mark is true, then the first statement is executed otherwise the second is. 
The ternary operator can also be used for an assignment. 

\subsection{while} 
\todo{Er ikke sikker på om den metode anvendt til if er fik og/eller den vi vil bruge. Venter med at skrive dette.}
\subsection{for} 
\todo{Er ikke sikker på om den metode anvendt til if er fik og/eller den vi vil bruge. Venter med at skrive dette.}