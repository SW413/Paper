\chapter{Design}
\label{cha:Design}
\section{Philosophy of the language}


\section{Generel}
\subsection{Character set}

\subsection{Comments}

\subsection{Case Sensitivity}

\subsection{Ignored characters}

\subsection{Scope}

\subsection{Operators}

\section{Type and variables}\label{Types}
\subsection{Integers}

\subsection{Floating points}

\subsection{Booleans}

\section{Functions}
This section will describe GAMBLE's use of functions. 
Like in many other languages it is possible to declare your own functions in GAMBLE.
This is useful for organising code, and reusing code in sourcecode.

\subsection{Function identifiers}
When identifying a function you also need to write the function's body.
If function identifiers were spread out all over the document it could be hard to find where certain elements of the code were.
In C it is possible to make prototypes of functions in the top of the document, and then declare the bodies of the functions in other places of the document. 
This is not possible in GAMBLE because of the increase in readability.
A function identifier is exactly like C, with the \texttt{return type} then the \texttt{functionname}, \texttt{(formal parameters)} and finally \texttt{\{body\}}.
An example of a function identifier can be seen on \myref{functionID}.
Other syntaxes could have been made, but not only is this the way C does it, it is also widely used in many other languages like C\# and Java.


\begin{lstlisting}[label=functionID]
int functionName(int a, float b){
	functionbody..
}
\end{lstlisting}

\subsection{Function calls}
When you have made a function, you need to call the function as well.
This is done again exactly like in C, where you call a function by writing its name followed by the formal parameters in parenthesis.
This is a widely used way of calling functions in many languages just like the identification was.
An example can be seen on \myref{functionCall}.

\begin{lstlisting}[label=functionCall]
functionName(4, 3.4);
\end{lstlisting}


\subsection{Return value}
Functions have a return value which can be seen on \myref{functionId}.
The return values are all the types in GAMBLE mentioned in \myref{Types} and then void. 
A void function will not return anything, but instead can be seen as a procedure which manipulates its input, and maybe prints something.
This choice was made because GAMBLE should to be able to return all the different types in the language from a function, but also be able to perform these procedures which a void function may do.
GAMBLE assign return values from a function just like it assigns other values to variables.
An example can be seen on \myref{returnFunction}.

\begin{lstlisting}[label=returnFunction]
int a = 0;

a = functionName(4, 3.4);
\end{lstlisting}

\subsection{Premade functions (rename)}\todo{This}


\section{Control Flow}
\subsection{For-loop}

\subsection{While-loop}

\subsection{If \& If else}
