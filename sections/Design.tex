\chapter{Design}
\label{cha:Design}
%Metatext for design chapter
In this chapter the design of the languange will be represented.
First the philosophy for the languange will be introduced to clarify what this languange is attempting to acheive.
The philosophy will lead to some attributes being more important than others, the choices we deem the more impactful ones will also be documented in this chapter.

\input{sections/philosophy.tex}

\section{Generel}
\textbf{Character set}
Source code must be in Unicode.
This is chosen even though all variables and types is in ACSI, because Unicode gives the programmer of different origins the possibility to use special characters of theirs first language.
An example of this would be the danish %\"\Æ,\Ø,\Å\".

\textbf{Comments}
Comments are done as in other large languages i.e. C and Java, with the use of \"\/\/\".
The compiler ignores everything from \"\/\/\" to the end of the line.
The decision  to use already existing solutions is to make sure the language is easy to learn if the programmer already have some coding experience.
Below on \myref{lst:CommSemi} a snippet of an comment in use can be seen.

\textbf{Semicolon}
A semicolon \"\;\" is used as a terminator and are required after every statement.
This is again based on other already existing languages such as Java and the C family.
The use of terminators in a language is of less importance, and therefore we decide that a semicolon is required to terminate each statement for easier readability.
The use of semicolon can also be found on \myref{lst:CommSemi}.


\begin{lstlisting}[caption={Comment and Semicolon},label={lst:CommSemi}]
int a = 1;		//initialization with specified type.
\end{lstlisting}

\textbf{Case Sensitivity}
The language is case sensitive to make sure the programmer uses whatever convention he is most use to, and pleased with.
C\# uses case sensitivity to i.e. allow the programmer to separate class names and instances.
Having a case sensitive language provide a higher readability and writability since the programmer are use to writing English or other spoken language which are case sensitivity.

\textbf{Ignored characters}
The language is a free-form language, which means all whitespace is ignored.
Whitespace includes, space, tabs and line feed.
Whitespace is ignored so it easier to organise the source code to achieve a higher level of readability.

\textbf{Scope}
\todo{Scope rules?}

\textbf{Operators}
Available operators in the language can be found on \ref{tbl:operators} with a short description.
\begin{table}[h]
    \centering
    \colorlet{shadecolor}{gray!40}
    \rowcolors{1}{white}{shadecolor}
    \begin{tabular}{|c|l|l|l|l|l|}
    \hline
    \textbf{Operator}  & \textbf{Integer}                   & \textbf{Floating point number}    & \textbf{Boolean}      & \textbf{Matrix}       & \textbf{Vector}     \\ \hline
    +                  & Addition (infix)                   & Addition (infix)                  & Type error            & Addition (infix)      & Addition (infix) \\ \hline 
    -                  & Subtraction (infix)                & Subtraction (infix)               & Type error            & Subtraction (infix)   & Subtraction (infix) \\ \hline 
    ++                 & Increment (unary,pre-, postfix)    & Increment(unary,pre-, postfix)    & Type error            & Type error            & Type error \\ \hline    
    --                 & Decrement (unary,pre-, postfix)    & Decrement(unary,pre-, postfix)    & Type error            & Type error            & Type error \\ \hline
    *                  & Multiplication (infix)             & Multiplication(infix)             & Type error            & Type error            & Type error      \\ \hline
    \%                 & Remainder (infix)                  & Modulo(infix)                     & Type error            & Type error            & Type error  \\ \hline
    /                  & Division (infix)                   & Division (infix)                  & Type error            & Type error            & Type error \\ \hline
    \&\&               & Type error                         & Type error                        & Logical and (infix)   & Type error            & Type error \\ \hline 
    ||                 & Type error                         & Type error                        & Logical or (infix)    & Type error            & Type error \\ \hline 
    ==                 & Equality (infix)                   & Equality* (infix)                 & Type error            & Type error            & Type error \\ \hline 
    !=                 & Inequality (infix)                 & Inequality* (infix)               & Type error            & Type error            & Type error \\ \hline
    <=                 & Less than or equal to (infix)      & Less than or equal to* (infix)    & Type error            & Type error            & Type error \\ \hline
    >=                 & More than or equal to (infix)      & More than or equal to* (infix)    & Type error            & Type error            & Type error \\ \hline
    <                  & Less than (infix)                  & Less than (infix)                 & Type error            & Type error            & Type error \\ \hline
    >                  & More than (infix)                  & More than (infix)                 & Type error            & Type error            & Type error\\ \hline
    \end{tabular}
    \caption[List of operators on primitive data types in GAMBLE.]{List of operators on primitive data types in GAMBLE.\@*Equality comparisons on floating point number may be incorrect.}\label{tbl:operators}
\end{table}
\vspace{-20pt}  

\textbf{Structure} 


\section{Type and variables}\label{sec:Types}
%http://www.informit.com/articles/article.aspx?p=2103809&seqNum=3
GAMBLE uses three primitive data types; integers(int), floating points(float) and boolean values(bool), as well as two composite data types; matrices and vector.

\textbf{Integers and floating points}
Integers default to 32bit and the same goes for floating points, it is possible to declare them of another bit size; 16bit and 64bit are valid alternatives in case such should be needed.%Gør vi faktisk dette - im not sure?
The time sink of using the GPU is the overhead data transfer from the CPU to the GPU, as such having the data types declarable in smaller bit sizes allows the programmer to determine what bit size is really needed.
Thus awards the programmer the ability to optimize their computations while taking the data transfer into account.

\textbf{Booleans}
In a languange such as C, booleans are represented by integers where 0 evaluates to false.
To enhance readability and avoid the booleans acting as integers, they are given their own data type, where the meaning is clear by it being either true or false.

\textbf{Matrices}
The two composite data types provided by GAMBLE are representations of matrices and vectors from linear algebra.
A matrice can be of size n by m where n represents rows and m represents columns.
Each row correlates to a row vector, vectors exist both as row and column vectors.
As such a matrice can be seen as a number n of stacked row vectors, which happen to have the same representation as arrays do in C-like languanges i.e vectors can be seen as a type of array, and matrices as multiple arrays.

\textbf{Vectors}
The vectors are a subset of matrices, as such they are represented in a similar fashion.
%Det kunne eventuelt give mening at lave et table over hvordan vi repræsentere data(bitwise)??

\section{Functions}
This section will describe GAMBLE's use of functions. 
Like in many other languages it is possible to declare your own functions in GAMBLE.
This is useful for organising code, and reusing parts of the sourcecode.

\subsection{Function identifiers}
When identifying a function is is required to write the body of the function.
If function identifiers were spread out all over the document it could be hard to find where certain elements of the code were.
In C it is possible to make prototypes of functions in the top of the document, and then declare the bodies of the functions in other places of the document. 
This is not possible in GAMBLE because of the increase in readability.
A function identifier is exactly like C, with the \texttt{return type} then the \texttt{functionname}, \texttt{(formal parameters)} and finally \texttt{\{body\}}.
The function body can contain a number of statements and various controlflows.
Furthermore it can contain calls to other functions or calls to the function itself, and hereby recursion is a possibility in GAMBLE
An example of a function identifier can be seen on \myref{functionID}.
Other syntaxes could have been used, but not only is this the way C does it, it is also widely used in many other languages like C\# and Java.

\begin{lstlisting}[label=functionID]                                                             
int add(int a, int b){
	return a + b;
}
\end{lstlisting}

\subsection{Function calls}
When you have made a function, you need to call the function as well.
This is done again exactly like in C, where you call a function by writing its name followed by the formal parameters in parenthesis.
This is a widely used way of calling functions in many languages just like the identification was.
An example can be seen on \myref{functionCall}.

\begin{lstlisting}[label=functionCall]
add(4, 3);
\end{lstlisting}


\subsection{Return value}
Functions have a return value which can be seen on \myref{functionID}.
The return values are all the types in GAMBLE mentioned in \myref{sec:Types} and then void. 
A void function will not return anything, but instead can be seen as a procedure which manipulates its input, and perhaps prints something.
This choice was made because GAMBLE should to be able to return all the different types in the language from a function, but also be able to perform these procedures which a void function may do.
GAMBLE assigns return values from a function just like it assigns other values to variables.
The value or type to be returned is preceded by the keyword \texttt{return}, just like it is in C and other C-like languages.
An example can be seen on \myref{returnFunction}.

\begin{lstlisting}[label=returnFunction]
int a = 0;

a = add(4, 3);
\end{lstlisting}

\subsection{Premade functions (rename)}\todo{This} 
%\section{Control Flow}
%\textbf{For-loop}
%\textbf{While-loop}
%\textbf{If \& If else}

%% OLD
%
\section{Control Flow}
Control flow is the parts of the language which will make ``decitions''.
Often these are separated into two primary groups: iteration and condition. 
An iteration is a statement which gives the program the possibility to iterate over the same code a given number of times. 
Common loops in programming languages are: do..while, while, for and foreach. 
The iteration of a loop can be broken in our language, to jump out of the loop, by a statement often called \texttt{break}, and the ability to skip to start of the next iteration by the keyword \texttt{continue}.

The conditionals in many programming languages are: if, ternary operator (?:) and switch.
The combination of a conditional and a jump, often seen as a \texttt{goto} can produce an iteration. 
The goto statement is not in our language, it is considered harmful by many, and in all but a few cases useless, if other conditionals are in the programming language. \citep{DijkstraGoto}
The switch statement is not included in our language. \todo{Skal vi have switch med?}

\subsection{While-loop}
The while-loop is a very common language construct, it iterates once or more times \textit{while} a given boolean condition is true. 
In our language blocks are delimited by curly brackets (``\texttt{\{}'' and ``\texttt{\}}''), and so is the body of the while-loop. 
Unlike languages like C, C\# and Java etc. in our language it is not allowed to omit the curly brackets if one wishes to only to have a single statement in the while-loop, this is a syntax error. 
This is to increase the consistency of the code thus making it more readable. 
In \myref{lst:whileExample} is an example of a while-loop in our language. 
An variation of the while-loop called the do..while-loop, in which a single execution a done before checking the boolean condition is not included in our language. 
This is to reduce the amount of built-in control structures. 

\begin{figure}[h]
\begin{lstlisting}[caption=An example of a while-loop, label=lst:whileExample]
while ( boolean_condition )
{
    example_statement();
}
\end{lstlisting}
\end{figure}
\subsection{For-loop}
A for-loop is a loop like the while-loop, with additional features. 
The traditional for-loop has three fields, the initialization, the condition and the increment/decrement. 
Any of these can be omitted, if the conditional is omitted then the loop is an infinite loop, but it can still be broken by the \texttt{break} statement.
Like for while-loops, curly brackets around the body of for-loops are enforced. 
Any traditional for-loop can be translated into a while-loop, by putting the initialization before the while-loop and the increment/decrement in the end of the body. 
We have the tradition for-loop in our language, an example of it is shown in \myref{lst:whileExample}. 
For-loops are often used to iterate over a collection, and it is easier to see what the for-loop does than a while-loop, as the three fields are in one place. 
This can be further simplified by having a foreach-loop, which iterate over each element in a collection.
However this is not included in our language, in order to reduce the scope of the language and simplify its development. \todo{Skal vi have foreach?}
\begin{figure}[h]
\begin{lstlisting}[caption=An example of a for-loop, label=lst:forExample]
for ( initialization ; boolean_condition ; increment/decrement )
{
    example_statement();
}
\end{lstlisting}
\end{figure}


\subsection{Conditionals}
The if-statement is a very common language construct, it is just like a while-loop except it does not iterate and only executes once if the condition is true.
Like for while-loops, curly brackets around the if-statement are enforced. 
Optionally an else statement can be added to the if-statement, this occurs only if the boolean condition of the previous if was false. 
A grouping of several if and else statements are allowed and is called an if-else-chain. 

The \texttt{?:} operator known as the conditional-operator or ternary if, is also in our language. 
Before the question mark is a boolean condition if it is true, then the statement after the question mark is executed otherwise the one after the colon. 
An example of both the if, if-else and ternary if are shown in \myref{lst:condExample}.
\begin{figure}[h]
\begin{lstlisting}[caption=An example of the conditional operators., label=lst:condExample]
if ( boolean_condition )
{
    example_statement();
} else {
    // executed only if the boolean_condition is false
    other_statement;
}

// ternary example, if the boolean_condition is true,
// value is alternative_one, else it is alternative_two
value = boolean_condition ? alternative_one : alternative_two;

\end{lstlisting}
\end{figure}

%% New
\section{Control Flow}\label{subsec:control-flow}
Control flow is any part of the language in which a branch occurs. 
A branch is in this case a change in which instruction happens next. 
These come in 3 variations: primitives, choice and loops.
C, C\#, Java etc contain some or all of these elements to control the flow of a program:

\begin{itemize}
    \item \texttt{if/if-else}
    \item \texttt{?:} (Ternary operator)
    \item \texttt{switch}
    \item \texttt{while}
    \item \texttt{for}
    \item \texttt{foreach}
    \item \texttt{do..while}
    \item \texttt{goto}
\end{itemize}

If a programming language has if, goto and label then the other constructs can be made.
For example a while-loop can be translated to a if and goto, this is shown in \myref{ifgotowhile1} and \myref{ifgotowhile2} written in C. 

\noindent\begin{minipage}{.45\textwidth}
\begin{lstlisting}[caption=Loop made with while.,frame=tlrb, label=ifgotowhile1, numbers=none]{Name}
while ( condition )
{
    statement();
}
\end{lstlisting}
\end{minipage}\hfill
\begin{minipage}{.45\textwidth}
\begin{lstlisting}[caption=The same loop with if and goto.,frame=tlrb, label=ifgotowhile2, numbers=none]{Name}
label startOfLoop:
if ( condition )
{
    statement();
    goto startOfLoop;
}
\end{lstlisting}
\end{minipage}

The same can be done with the \texttt{for} and \texttt{do..while} loops. 
This means that the only need control structures should be \texttt{if} and \texttt{goto}. 
However the use of goto is considered harmful, because it is easy to make mistakes when using it. \citep{DijkstraGoto}
Even though \texttt{for} and \texttt{while} are very similar, they are often used in different contexts.
The for loop is very useful for iterating over a collection or knowing exactly how many time the code is executed. 
Therefore we have decided to include both \texttt{for} and \texttt{while} in GAMBLE, but not \texttt{foreach} and \texttt{do..while}, this is to simplify the number of control structures.

We have decided to keep the traditional \texttt{if} and \texttt{if-else} constructs. 
These are very useful, and can achieve very much the same as a \texttt{switch}, by chaining if-else statements, so we have excluded it. 
The ternary operator, also known as the inline if, written as \texttt{?:} in C and many other languages, is useful for compacting code to make it more readable, therefore we include it in GAMBLE.

\subsection{if, if-else and ?:} \todo{Jeg blander meget rundt i statement og construct, kan ikke helt beslutte mig. }
Examples of if/if-else GAMBLE are shown in \myref{iflst} and \myref{ifelselst}. 
The condition is enclosed in a parenthesis this is to make the if construct similar to the function call, in which the parameters are in a parenthesis. 
All the conditional code must be enclosed in a curly bracket, this is to make the code more readable. 
The use of curly brackets is to make them similar to functions to make the language more alike \todo{mangler et godt fagord her.}.

\noindent\begin{minipage}{.45\textwidth}
\begin{lstlisting}[caption=An if statement in GAMBLE.,frame=tlrb, label=iflst, numbers=none]{Name}
if ( condition )
{
    statement();
}
\end{lstlisting}
\begin{lstlisting}[caption=A use of \texttt{?:} in GAMBLE.,frame=tlrb, label=terlst, numbers=none]{Name}
condition ? ifTrue() : ifFalse();
\end{lstlisting}

\end{minipage}\hfill
\begin{minipage}{.45\textwidth}
\begin{lstlisting}[caption=An if-else statement in GAMBLE.,frame=tlrb, label=ifelselst, numbers=none]{Name}
if ( condition )
{
    statement();
} else {
    alternative_statement();
}
\end{lstlisting}
\end{minipage}

An example of the ternary operator \texttt{?:} is shown in \myref{terlst}. 
If the boolean statement before the question mark is true, then the first statement is executed otherwise the second is. 
The ternary operator can also be used for an assignment. 

\subsection{while} 
\todo{Er ikke sikker på om den metode anvendt til if er fik og/eller den vi vil bruge. Venter med at skrive dette.}
\subsection{for} 
\todo{Er ikke sikker på om den metode anvendt til if er fik og/eller den vi vil bruge. Venter med at skrive dette.}

