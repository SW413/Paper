\chapter{Design}
\label{cha:Design}
\section{Philosophy of the language}


\section{Generel}
\textbf{Character set}
Source code must be in Unicode.
This is chosen even though all variables and types is in ACSI, because Unicode gives the programmer of different origins the possibility to use special characters of theirs first language.
An example of this would be the danish %\"\Æ,\Ø,\Å\".

\textbf{Comments}
Comments are done as in other large languages i.e. C and Java, with the use of \"\/\/\".
The compiler ignores everything from \"\/\/\" to the end of the line.
The decision  to use already existing solutions is to make sure the language is easy to learn if the programmer already have some coding experience.
Below on \todo{myref} a snippet of an comment in use can be seen.

\textbf{Semicolon}
A semicolon \"\;\" is used as a terminator and are required after every statement.
This is again based on other already existing languages such as Java and the C family.
The use of terminators in a language is of less importance, and therefore we decide that a semicolon is required to terminate each statement for easier readability.
The use of semicolon can also be found on \todo{myref}


\begin{lstlisting}[caption={Comment and Semicolon}]
int a = 1;		//initialization with specified type.
\end{lstlisting}

\textbf{Case Sensitivity}
The language is case sensitive to make sure the programmer uses whatever convention he is most use to, and pleased with.
C\# uses case sensitivity to i.e. allow the programmer to separate class names and instances.
Having a case sensitive language provide a higher readability and writability since the programmer are use to writing English or other spoken language which are case sensitivity.

\textbf{Ignored characters}
The language is a free-form language, which means all whitespace is ignored.
Whitespace includes, space, tabs and line feed.
Whitespace is ignored so it easier to organise  the source code to achieve a higher level of readability.

\textbf{Scope}
\todo{Scope rules?}

\textbf{Operators}
Available operators in the language can be found on \ref{tbl:operators} with a short description.
\begin{table}[h]
    \centering
    \colorlet{shadecolor}{gray!40}
    \rowcolors{1}{white}{shadecolor}
    \begin{tabular}{|c|l|l|l|l|l|}
    \hline
    \textbf{Operator}  & \textbf{Integer}                   & \textbf{Floating point number}    & \textbf{Boolean}      & \textbf{Matrix}       & \textbf{Vector}     \\ \hline
    +                  & Addition (infix)                   & Addition (infix)                  & Type error            & Addition (infix)      & Addition (infix) \\ \hline 
    -                  & Subtraction (infix)                & Subtraction (infix)               & Type error            & Subtraction (infix)   & Subtraction (infix) \\ \hline 
    ++                 & Increment (unary,pre-, postfix)    & Increment(unary,pre-, postfix)    & Type error            & Type error            & Type error \\ \hline    
    --                 & Decrement (unary,pre-, postfix)    & Decrement(unary,pre-, postfix)    & Type error            & Type error            & Type error \\ \hline
    *                  & Multiplication (infix)             & Multiplication(infix)             & Type error            & Type error            & Type error      \\ \hline
    \%                 & Remainder (infix)                  & Modulo(infix)                     & Type error            & Type error            & Type error  \\ \hline
    /                  & Division (infix)                   & Division (infix)                  & Type error            & Type error            & Type error \\ \hline
    \&\&               & Type error                         & Type error                        & Logical and (infix)   & Type error            & Type error \\ \hline 
    ||                 & Type error                         & Type error                        & Logical or (infix)    & Type error            & Type error \\ \hline 
    ==                 & Equality (infix)                   & Equality* (infix)                 & Type error            & Type error            & Type error \\ \hline 
    !=                 & Inequality (infix)                 & Inequality* (infix)               & Type error            & Type error            & Type error \\ \hline
    <=                 & Less than or equal to (infix)      & Less than or equal to* (infix)    & Type error            & Type error            & Type error \\ \hline
    >=                 & More than or equal to (infix)      & More than or equal to* (infix)    & Type error            & Type error            & Type error \\ \hline
    <                  & Less than (infix)                  & Less than (infix)                 & Type error            & Type error            & Type error \\ \hline
    >                  & More than (infix)                  & More than (infix)                 & Type error            & Type error            & Type error\\ \hline
    \end{tabular}
    \caption[List of operators on primitive data types in GAMBLE.]{List of operators on primitive data types in GAMBLE.\@*Equality comparisons on floating point number may be incorrect.}\label{tbl:operators}
\end{table}
\vspace{-20pt}   


\section{Type and variables}
\textbf{Integers}
\textbf{Floating points}
\textbf{Booleans}

\section{Functions}
\textbf{Function identifiers}
\textbf{Function calls}
\textbf{Return value}
\textbf{Premade functions (rename)}

\section{Control Flow}
\textbf{For-loop}
\textbf{While-loop}
\textbf{If \& If else}